% Особенности инструкций AVX-512.
\subsection{Особенности набора инструкций AVX-512}

Набор инструкций AVX-512 представляет собой расширение 256-битных инструкций AVX архитектуры Intel x86.
Этот набор инструкций поддержан в семействах микропроцессоров Intel Xeon Phi второго поколения (Knight Landing, KNL) и Intel Xeon Skylake.

Инструкции AVX-512 работают с 512-битными векторными регистрами (zmm), которые могут содержать целочисленные или вещественные данные.
Каждый zmm регистр способен вместить, например, 8 вещественных значений двойной точности (double) или 16 вещественных значений одинарной точности (float).
Набор инструкций AVX-512 реализует множество операций с векторными аргументами, среди которых арифметические операции, операции сравнения, операции чтения из памяти и записи в память, транцендентные операции, комбинированные операции вида $\pm a \cdot b \pm c$, операции перестановки элементов векторов и другие.

Для поддержки выборочного применения операций над упакованными данными к конкретным элементам векторов большинство инструкций AVX-512 использует специальные регистры-маски в качестве аргументов.
Всего таких регистров 8 (k0-k7).
При выполнении векторной операции элемент результирующего вектора будет вычислен только если бит маски с соответствующим номером выставлен в единицу, в противном случае выполнение операции для данных элементов будет проигнорировано.
Данная уникальная возможность набора инструкций AVX-512 обеспечивает реализацию предикатного режима исполнения \cite{Volkonsky2003}, который поддержан в таких архитектурах, как ARM или «Эльбрус» \cite{Kim2013}.
Наличие предикатного режима исполнения позволяет применять оптимизацию слияния ветвей исполнения и, таким образом, избавляться от лишних операций передачи управления, что помогает создавать высокоэффективный параллельный код.

Из других важных особенностей набора инструкций AVX-512 можно отметить операции множественного чтения элементов векторов, расположенных в памяти с произвольными смещениями от базового адреса, а также аналогичные операции записи элементов векторов в
память с произвольными смещениями (операции gather/scatter).
Хотя эти операции крайне медленные, они в некоторых случаях помогают существенно упростить логику векторизованного кода.
Также следует отметить большое разнообразие различных операций перестановки, перемешивания, дублирования, пересылки элементов векторов, что позволяет произвольным образом менять порядок обработки данных.
Также существенное ускорение способны принести комбинированные операции, объединяющие операцию умножения и сложения в одну операцию.

Для упрощения применения векторных инструкций при оптимизации программного кода для компилятора icc разработаны специальные функции-интринсики (они определены в заголовочном файле immintrin.h) \cite{IntelIntrinsicsGuide}.
Эти функции покрывают не все множество инструкций AVX-512, однако избавляют от необходимости вручную
писать ассемблерный код.
Вместо этого предоставляется возможность оперировать встроенными типами данных для 512-битных векторов и использовать их при работе с функциями-инстринсиками как обычные базовые типы (при построении компилятором исполняемого кода для этих типов данных будут использованы регистры zmm).
Некоторые функции-инстринсики соответствуют не одной отдельной команде, а целой последовательности, как например группа функций reduce, другие же просто раскрываются в вызов библиотечной функции (например, тригонометрические функции или функция hypot).
