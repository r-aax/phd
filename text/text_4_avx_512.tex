% Особенности инструкций AVX-512.
\subsection{Особенности набора инструкций AVX-512}

Набор инструкций AVX-512 представляет собой расширение 256-битных инструкций AVX архитектуры Intel x86.
Этот набор инструкций поддержан в семействах микропроцессоров Intel Xeon Phi второго поколения (Knight Landing, KNL) и Intel Xeon Skylake.
До появления набора инструкций AVX-512 сопроцессоры Intel Xeon Phi KNC также поддерживали 512-битные инструкции, которые схожи с AVX-512, однако исполняемый код для Intel Xeon Phi KNC не был бинарно совместимым с исполняемым кодом для Intel Xeon.

Инструкции AVX-512 работают с 512-битными векторными регистрами (zmm), которые могут содержать целочисленные или вещественные данные.
Каждый zmm регистр способен вместить, например, 8 вещественных значений двойной точности (double) или 16 вещественных значений одинарной точности (float).
Набор инструкций AVX-512 реализует множество операций с векторными аргументами, среди которых арифметические операции, операции сравнения, операции чтения из памяти и записи в память, транцендентные операции, комбинированные операции вида $\pm a \cdot b \pm c$, операции перестановки элементов векторов и другие.

Множество инструкций AVX-512 состоит из следующих подмножеств: AVX-512F (Foundation) -- основной набор векторных инструкций с поддержкой маскирования, AVX-512PF (PreFetch) -- инструкции предварительной подкачки данных из памяти, AVX-512ER (Exponential and Reciprocal) -- команды для вычисления экспоненты и обратных значений, AVX-512CD (Conflict Detection) -- инструкции для определения конфликтов, которые помогают эффективно применять векторизацию кода, а также наборы AVX-512BW и AVX-512DQ, поддержанные в Skylake.
В следующих поколениях процессоров (Intel Xeon Phi Knights Mill, Intel Cannonlake, Intel Ice Lake) набор инструкций AVX-512 расширяется еще больше, в него входят команды для работы с 52-битными целочисленными значениями, специальные команды для работы с нейросетями и AES шифрованием, реализация арифметики полей Галуа, имплементация специальных битовых операций, а также новый класс комбинированных операций, позволяющий еще больше повысить пиковую производительность процессоров.

Для поддержки выборочного применения операций над упакованными данными к конкретным элементам векторов большинство инструкций AVX-512 использует специальные регистры-маски в качестве аргументов.
Всего таких регистров 8 (k0-k7).
Маски используются в командах для осуществления условной операции над элементами упакованных данных (если соответствующий бит выставлен в 1, то операция выполняется, а точнее результат операции записывается в соответствующий элемент вектора назначения) или для слияния элементов данных в регистр назначения.
Также маски могут использоваться для выборочного чтения из памяти и запись в память элементов векторов, для аккумулирования результатов логических операций над элементами векторов.
Данная уникальная возможность набора инструкций AVX-512 обеспечивает реализацию предикатного режима исполнения \cite{Volkonsky2003}, который поддержан в таких архитектурах, как ARM или «Эльбрус» \cite{Kim2013}.
Наличие предикатного режима исполнения позволяет применять оптимизацию слияния ветвей исполнения и, таким образом, избавляться от лишних операций передачи управления, что помогает создавать высокоэффективный параллельный код.

Из других важных особенностей набора инструкций AVX-512 можно отметить операции множественного чтения элементов векторов, расположенных в памяти с произвольными смещениями от базового адреса, а также аналогичные операции записи элементов векторов в
память с произвольными смещениями (операции gather/scatter).
Хотя эти операции крайне медленные, они в некоторых случаях помогают существенно упростить логику векторизованного кода.
Также следует отметить большое разнообразие различных операций перестановки, перемешивания, дублирования, пересылки элементов векторов, что позволяет произвольным образом менять порядок обработки данных.
Также существенное ускорение способны принести комбинированные операции, объединяющие операцию умножения и сложения в одну операцию.

По схеме работы можно выделить несколько групп операций AVX-512.
Упакованные операции с одним операндом zmm (512-битный вектор) и одним результатом zmm получают на вход один вектор и применяют к каждому его элементу конкретную функцию, получая результат того же размера, который по маске записывается в выходной вектор. Примерами таких операция является получение абсолютного значения, извлечение корня, округление, операции сдвигов и другие.
Упакованные операции с двумя операндами zmm и одним результатом zmm отличаются только тем, что применяемая функция является бинарной.
К данной группе относятся операции поэлементного сложения, вычитания, умножения, деления, сдвига на переменное количество разрядов и другие.
Упакованные операции с двумя операндами zmm и результатом маской выполняют поэлементное сравнение двух векторов.
Операции конвертации предназначены для преобразования элементов вектора из одного формата в другой, к ним относятся наборы команд cvt и pack.
Упакованные комбинированные операции принимают на вход сразу три zmm вектора a, b, c и поэлементно вычисляют значения вида $\pm a \cdot b \pm c$, которые по маске записываются в выходной вектор.
Операции перестановок не выполняют арифметических действий, а только переставляют части вектора в произвольном порядке, определяемом типом операции и дополнительными параметрами.
Данная группа представлена большим набором разнообразных операций unpck, shuf, align, blend, perm.
Операции пересылок предназначены для перемещения последовательных данных между регистрами, а также между памятью и регистром.
Поддержаны также операции пересылки элементов данных, расположенных не последовательно, а с произвольными смещениями от заданного базового адреса в памяти (операции gather и scatter), а также операции пересылки с дублированием элементов, позволяющие переместить одно значение сразу в несколько элементов вектора.
Операции предварительной подкачки данных используются для того, чтобы увеличить вероятность того, что к моменту исполнения команды данные уже будут в кэше.
Кроме того, поддержаны другие операции с более сложной логикой, среди которых определение класса вещественного числа, реализация логических функций от трех аргументов, операции определения конфликтов и другие

Для упрощения применения векторных инструкций при оптимизации программного кода для компилятора icc разработаны специальные функции-интринсики (они определены в заголовочном файле immintrin.h) \cite{IntelIntrinsicsGuide}.
Эти функции покрывают не все множество инструкций AVX-512, однако избавляют от необходимости вручную писать ассемблерный код.
Вместо этого предоставляется возможность оперировать встроенными типами данных для 512-битных векторов и использовать их при работе с функциями-инстринсиками как обычные базовые типы (при построении компилятором исполняемого кода для этих типов данных будут использованы регистры zmm).
Некоторые функции-инстринсики соответствуют не одной отдельной команде, а целой последовательности, как например группа функций reduce, другие же просто раскрываются в вызов библиотечной функции (например, тригонометрические функции или функция hypot).
Из множества интринсиков можно выделить следующие группы функций, схожие по структуре.
Функции swizzle, shuffle, permute и permutevar осуществляют перестановку элементов вектора и раскрываются в последовательность операций, в которой присутвует shuf и пересылка по маске.
Для большего числа операций AVX-512 реализованы соответствующие инстринсики, раскрывающиеся в одну конкретную операцию.
Среди них арифметические операции, побитовые операции, операции чтения из памяти и записи в память, операции конвертации, слияние двух векторов, нахождение обратных значений, получение минимума и максимума из двух значений, операции сравнения, операции с масками, комбинированные операции и другие.
Некоторые инстринсики, особенно предназначенные для выполнения упакованных трансцендентных операций, раскрываются просто в вызов библиотечной функции (например, \texttt{\_mm512\_log\_ps}, \texttt{\_mm512\_hypot\_ps}, тригонометрические функции). 
