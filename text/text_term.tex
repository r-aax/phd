% Предметный указатель.
\newpage

\section*{Предметный указатель}
\addcontentsline{toc}{section}{Предметный указатель}

алгоритм Гаусса-Жордана, \pageref{term:alg_gauss_zhordan}

алгоритм декомпозиции генетический, \pageref{term:alg_decomp_gen}

алгоритм декомпозиции иерархического дробления, \pageref{term:alg_decomp_hierarch}, \pageref{term:alg_decomp_hierarch2}, \pageref{term:alg_decomp_hierarch3}

алгоритм декомпозиции инкрементальный, \pageref{term:alg_decomp_inc}

алгоритм декомпозиции линейный, \pageref{term:alg_decomp_linear}, \pageref{term:alg_decomp_linear2}

алгоритм декомпозиции пузырькового роста, \pageref{term:alg_decomp_bubble}

алгоритм декомпозиции с помощью наращивания доменов, \pageref{term:alg_decomp_rgrow}, \pageref{term:alg_decomp_rgrow2}, \pageref{term:alg_decomp_rgrow3}

алгоритм декомпозиции случайный, \pageref{term:alg_decomp_random}, \pageref{term:alg_decomp_random2}

алгоритм декомпозиции Фархата, \pageref{term:alg_decomp_farhat}

алгоритм сглаживания границ между доменами, \pageref{term:alg_smooth_domains_border}, \pageref{term:alg_smooth_domains_border2}

базовая точка в задаче распределения вычислительной нагрузки, \pageref{term:distr_base_point}

быстрое преобразование Фурье, \pageref{term:furier_transform}

векторизация вычислений, \pageref{term:vectorization}

вычислительная молекула, см. вычислительная окрестность ячейки

вычислительная окрестность ячейки, \pageref{term:cell_calc_template}, \pageref{term:cell_calc_template2}, \pageref{term:cell_calc_template3}

вычислительный кластер гетерогенный, \pageref{term:cluster_getero}, \pageref{term:cluster_getero2}

вычислительный кластер гомогенный, \pageref{term:cluster_gomo}, \pageref{term:cluster_gomo2}

вычислительный шаблон ячейки, см. вычислительная окрестность ячейки

генотип, \pageref{term:genotype}, \pageref{term:genotype2}

геометрическое место точек, \pageref{term:gmt}

градиентный спуск, \pageref{term:gradient_spusk}

грань блока, \pageref{term:block_facet}

граф вычислительного кластера, \pageref{term:graph_cluster}

граф дуальный, \pageref{term:dual_graph}, \pageref{term:dual_graph2}, \pageref{term:dual_graph3}

граф кубический, \pageref{term:graph_cubic}

граф расчетной задачи, \pageref{term:graph_task}

двухцветный цикл, \pageref{term:bicolor_cycle}

деформация системы линейных неравенств, \pageref{term:deform_sys_lin_neravenstv}, \pageref{term:deform_sys_lin_neravenstv2}

домен, \pageref{term:domain}, \pageref{term:domain2}, \pageref{term:domain3}, \pageref{term:domain4}, \pageref{term:domain5}

интерфейс касания блоков расчетной сетки, \pageref{term:block_interface}

критический путь исполнения, \pageref{term:critical_path}, \pageref{term:critical_path2}

метод Годунова, \pageref{term:godunov_method}, \pageref{term:godunov_method2}

метод многослойного перестроения поверхности, \pageref{term:method_remesh_multi}

метод окрестностей перестроения поверхности, \pageref{term:method_remesh_okr}

метод пирамид перестроения поверхности, \pageref{term:method_remesh_pyramid}

метод погруженной границы, \pageref{term:immersed_boundary_method}, \pageref{term:immersed_boundary_method2}

метод призм перестроения поверхности, \pageref{term:method_remesh_prism}

метод прямоугольников перестроения поверхности, \pageref{term:method_remesh_rect}

метод Тонга перестроения поверхности, \pageref{term:method_remesh_tong}

метод трапеций перестроения поверхности, \pageref{term:method_remesh_trap}

метод устранения самопересечений с восстановлением, \pageref{term:method_selfint_repare}

метод устранения самопересечений с дроблением, \pageref{term:method_selfint_cut}

мутация, \pageref{term:mutation}

область блока, \pageref{term:block_scope}

окрестность, \pageref{term:okrestnost}

опортная вершина в алгоритме декомпозиции, \pageref{term:opor_point}, \pageref{term:opor_point2}

показатель максимальной длины границы между доменами, \pageref{term:decomp_maxbord}, \pageref{term:decomp_maxbord2}, \pageref{term:decomp_maxbord3}, \pageref{term:decomp_maxbord4}, \pageref{term:decomp_maxbord5}

показатель неравномерность декомпозиции, \pageref{term:decomp_neravn}, \pageref{term:decomp_neravn2}, \pageref{term:decomp_neravn3}, \pageref{term:decomp_neravn4}, \pageref{term:decomp_neravn5}, \pageref{term:decomp_neravn6}, \pageref{term:decomp_neravn7}

показатель суммарной длины границы между доменами, \pageref{term:decomp_sumbord}, \pageref{term:decomp_sumbord2}, \pageref{term:decomp_sumbord3}, \pageref{term:decomp_sumbord4}, \pageref{term:decomp_sumbord5}

популяция, \pageref{term:population}

предикатное исполнение, \pageref{term:predicate_execution}

раскраска Тейта, \pageref{term:coloring_tait}

расчетная сетка адаптивная локально-измельчающаяся, \pageref{term:mesh_adaptive}

расчетная сетка блочно-структурированная, \pageref{term:mesh_block_struct2}, \pageref{term:mesh_block_struct3}, \pageref{term:mesh_block_struct4}, \pageref{term:mesh_block_struct5}

расчетная сетка декартова, \pageref{term:mesh_descartes}, \pageref{term:mesh_descartes2}

расчетная сетка несогласованная, \pageref{term:mesh_nesoglas}

расчетная сетка неструктурированная поверхностная, \pageref{term:unstruct_surf_calc_mesh}, \pageref{term:unstruct_surf_calc_mesh2}, \pageref{term:unstruct_surf_calc_mesh3}, \pageref{term:unstruct_surf_calc_mesh4}, \pageref{term:unstruct_surf_calc_mesh5}, \pageref{term:unstruct_surf_calc_mesh6}

расчетная сетка простая, \pageref{term:mesh_simple}

расчетная сетка согласованная, \pageref{term:mesh_soglas}

ребро кроссдоменное, \pageref{term:edge_cross}, \pageref{term:edge_cross2}, \pageref{term:edge_cross3}, \pageref{term:edge_cross4}

ребро междоменное, см. ребро кроссдоменное

решатель римановский, \pageref{term:riemann_solver}, \pageref{term:riemann_solver2}, \pageref{term:riemann_solver3}

самопересечение расчетной сетки, \pageref{term:mesh_self_intersect}

свертывание системы линейных неравенств, \pageref{term:method_svert_sys_neravenstv}

сглаживание в нулевом пространстве, \pageref{term:smooth_null}

сглаживание высот, \pageref{term:smooth_height}

сглаживание нормалей, \pageref{term:smooth_norm}

сильное масштабирование вычислений, \pageref{term:strong_scale}

скрещивание, \pageref{term:crossover}

стратегия распараллеливания CHUNKS, \pageref{term:parallel_strategy_chunks}

стратегия распараллеливания INTERLEAVE, \pageref{term:parallel_strategy_interleave}

стратегия распараллеливания RACE, \pageref{term:parallel_strategy_race}

схема Стегера-Уорминга, \pageref{term:steger_warming_scheme}

теневой слой, \pageref{term:block_shadow_layer}, \pageref{term:block_shadow_layer2}

ускорение при распаралленивании на общей памяти, \pageref{term:shr_speedup}

ускорение при распараллеливании с передачей сообщений, \pageref{term:msg_speedup}, \pageref{term:msg_speedup2}, \pageref{term:msg_speedup3}

функции-интринсики, \pageref{term:intrinsic}

шаблон аппроксимации в методе погруженных границы, \pageref{term:ibm_template}

шаблон сглаживания границы между доменами, \pageref{term:smooth_template}

штрафная функция, \pageref{term:penalty_function}

эффективность распараллеливания на общей памяти, \pageref{term:shr_eff}

эффективность распараллеливания с передачей сообщений, \pageref{term:msg_eff}, \pageref{term:msg_eff2}, \pageref{term:msg_eff3}

ячейка блока или домена внутреняя, \pageref{term:cell_block_inner}, \pageref{term:cell_block_inner2}

ячейка блока или домена граничная, \pageref{term:cell_block_border}, \pageref{term:cell_block_border2}

ячейка блока интерфейсная, \pageref{term:cell_block_interface}

ячейка блока или домена фиктивная, \pageref{term:cell_block_ghost}, \pageref{term:cell_block_ghost2}

ячейка блока или домена MPI обмена, \pageref{term:cell_block_mpi}, \pageref{term:cell_block_mpi2}

ячейка внешняя в методе погруженной границы, \pageref{term:cell_ibm_outer}, \pageref{term:cell_ibm_outer2}

ячейка внутренняя в методе погруженной границы, \pageref{term:cell_ibm_innner}, \pageref{term:cell_ibm_innner2}

ячейка граничная в методе погруженной границы, \pageref{term:cell_ibm_border}, \pageref{term:cell_ibm_border2}

ячейка пересечения при поиске самопересечений, \pageref{term:cell_intersect}

ячейка скрытая при поиске самопересечений, \pageref{term:cell_hidden}

ячейка статическая при поиске самопересечений, \pageref{term:cell_static}

ячейка фиктивная в методе погруженной границы, \pageref{term:cell_ibm_ghost}, \pageref{term:cell_ibm_ghost2}, \pageref{term:cell_ibm_ghost3}
