\% Предметный указатель.
\newpage

\section*{Предметный указатель}
\addcontentsline{toc}{section}{Предметный указатель}

базовая точка в задаче распределения вычислительной нагрузки, \pageref{term:distr_base_point}

вычислительная молекула ячейки, см. вычислительная окрестность ячейки

вычислительная окрестность ячейки, \pageref{term:cell_calc_template}

вычислительный кластер гетерогенный, \pageref{term:cluster_getero}, \pageref{term:cluster_getero2}

вычислительный кластер гомогенный, \pageref{term:cluster_gomo}, \pageref{term:cluster_gomo2}

вычислительный шаблон ячейки, см. вычислительная окрестность ячейки

геометрическое место точек, \pageref{term:gmt}

градиентный спуск, \pageref{term:gradient_spusk}

грань блока, \pageref{term:block_facet}

граф вычислительного кластера, \pageref{term:graph_cluster}

граф расчетной задачи, \pageref{term:graph_task}

деформация системы линейных неравенств, \pageref{term:deform_sys_lin_neravenstv}, \pageref{term:deform_sys_lin_neravenstv2}

домен, \pageref{term:domain}

дульный граф, \pageref{term:dual_graph}

интерфейс касания блоков расчетной сетки, \pageref{term:block_interface}

метод многослойного перестроения поверхности, \pageref{term:method_remesh_multi}

метод окрестностей перестроения поверхности, \pageref{term:method_remesh_okr}

метод пирамид перестроения поверхности, \pageref{term:method_remesh_pyramid}

метод погруженной границы, \pageref{term:immersed_boundary_method}, \pageref{term:immersed_boundary_method2}

метод призм перестроения поверхности, \pageref{term:method_remesh_prism}

метод прямоугольников перестроения поверхности, \pageref{term:method_remesh_rect}

метод Тонга перестроения поверхности, \pageref{term:method_remesh_tong}

метод трапеций перестроения поверхности, \pageref{term:method_remesh_trap}

метод устранения самопересечений с восстановлением, \pageref{term:method_selfint_repare}

метод устранения самопересечений с дроблением, \pageref{term:method_selfint_cut}

область блока, \pageref{term:block_scope}

окрестность, \pageref{term:okrestnost}

показатель максимальной длины границы между доменами, \pageref{term:decomp_maxbord}

показатель неравномерность декомпозиции, \pageref{term:decomp_neravn}, \pageref{term:decomp_neravn2}

показатель суммарной длины границы между доменами, \pageref{term:decomp_sumbord}

расчетная сетка адаптивная локально-измельчающаяся, \pageref{term:mesh_adaptive}

расчетная сетка блочно-структурированная, \pageref{term:mesh_block_struct}, \pageref{term:mesh_block_struct2}, \pageref{term:mesh_block_struct3}

расчетная сетка декартова, \pageref{term:mesh_descartes}, \pageref{term:mesh_descartes2}

расчетная сетка несогласованная, \pageref{term:mesh_nesoglas}

расчетная сетка неструктурированная поверхностная, \pageref{term:unstruct_surf_calc_mesh}, \pageref{term:unstruct_surf_calc_mesh2}

расчетная сетка простая, \pageref{term:mesh_simple}

расчетная сетка согласованная, \pageref{term:mesh_soglas}

ребро кроссдоменное, \pageref{term:edge_cross}

ребро междоменное, см. ребро кроссдоменное

самопересечение расчетной сетки, \pageref{term:mesh_self_intersect}

свертывание системы линейных неравенств, \pageref{term:method_svert_sys_neravenstv}

сглаживание в нулевом пространстве, \pageref{term:smooth_null}

сглаживание высот, \pageref{term:smooth_height}

сглаживание нормалей, \pageref{term:smooth_norm}

сильное масштабирование вычислений, \pageref{term:strong_scale}

теневой слой блока, \pageref{term:block_shadow_layer}

ускорение при распараллеливании с передачей сообщений, \pageref{term:msg_speedup}

шаблон аппроксимации в методе погруженных границы, \pageref{term:ibm_template}

эффективность распараллеливания с передачей сообщений, \pageref{term:msg_eff}

ячейка блока внутреняя, \pageref{term:cell_block_inner}

ячейка блока граничная, \pageref{term:cell_block_border}

ячейка блока интерфейсная, \pageref{term:cell_block_interface}

ячейка блока фиктивная, \pageref{term:cell_block_ghost}

ячейка блока MPI обмена, \pageref{term:cell_block_mpi}

ячейка внешняя в методе погруженной границы, \pageref{term:cell_ibm_outer}, \pageref{term:cell_ibm_outer2}

ячейка внутренняя в методе погруженной границы, \pageref{term:cell_ibm_innner}, \pageref{term:cell_ibm_innner2}

ячейка граничная в методе погруженной границы, \pageref{term:cell_ibm_border}, \pageref{term:cell_ibm_border2}

ячейка пересечения при поиске самопересечений, \pageref{term:cell_intersect}

ячейка скрытая при поиске самопересечений, \pageref{term:cell_hidden}

ячейка статическая при поиске самопересечений, \pageref{term:cell_static}

ячейка фиктивная в методе погруженной границы, \pageref{term:cell_ibm_ghost}, \pageref{term:cell_ibm_ghost2}, \pageref{term:cell_ibm_ghost3}
