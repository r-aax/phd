% Предметный указатель.
\newpage

\section*{Предметный указатель}
\addcontentsline{toc}{section}{Предметный указатель}

алгоритм Гаусса-Жордана, \pageref{term:alg_gauss_zhordan}

алгоритм декомпозиции генетический, \pageref{term:alg_decomp_gen}, \pageref{term:alg_decomp_gen2}

алгоритм декомпозиции иерархического дробления, \pageref{term:alg_decomp_hierarch}, \pageref{term:alg_decomp_hierarch2}, \pageref{term:alg_decomp_hierarch3}

алгоритм декомпозиции инкрементальный, \pageref{term:alg_decomp_inc}, \pageref{term:alg_decomp_inc2}

алгоритм декомпозиции линейный, \pageref{term:alg_decomp_linear}, \pageref{term:alg_decomp_linear2}

алгоритм декомпозиции пузырькового роста, \pageref{term:alg_decomp_bubble}, \pageref{term:alg_decomp_bubble2}

алгоритм декомпозиции с помощью наращивания доменов, \pageref{term:alg_decomp_rgrow}, \pageref{term:alg_decomp_rgrow2}, \pageref{term:alg_decomp_rgrow3}

алгоритм декомпозиции случайный, \pageref{term:alg_decomp_random}, \pageref{term:alg_decomp_random2}

алгоритм декомпозиции Фархата, \pageref{term:alg_decomp_farhat}

алгоритм сглаживания границ между доменами, \pageref{term:alg_smooth_domains_border}, \pageref{term:alg_smooth_domains_border2}

базовая точка в задаче распределения вычислительной нагрузки, \pageref{term:distr_base_point}

быстрое преобразование Фурье, \pageref{term:furier_transform}

векторизация вычислений, \pageref{term:vectorization}, \pageref{term:vectorization2}, \pageref{term:vectorization3}, \pageref{term:vectorization4}, \pageref{term:vectorization5}, \pageref{term:vectorization6}

векторная маска, \pageref{term:vector_mask}, \pageref{term:vector_mask2}, \pageref{term:vector_mask3}, \pageref{term:vector_mask4}, \pageref{term:vector_mask5}, \pageref{term:vector_mask6}, \pageref{term:vector_mask7}, \pageref{term:vector_mask8}

векторный предикат, см. векторная маска

вероятность ребра CFG, \pageref{term:edge_prob}

вычислительная молекула, см. вычислительная окрестность ячейки

вычислительная окрестность ячейки, \pageref{term:cell_calc_template}, \pageref{term:cell_calc_template2}, \pageref{term:cell_calc_template3}

вычислительный кластер гетерогенный, \pageref{term:cluster_getero}, \pageref{term:cluster_getero2}

вычислительный кластер гомогенный, \pageref{term:cluster_gomo}, \pageref{term:cluster_gomo2}

вычислительный шаблон ячейки, см. вычислительная окрестность ячейки

генотип, \pageref{term:genotype}, \pageref{term:genotype2}, \pageref{term:genotype3}

геометрическое место точек, \pageref{term:gmt}

градиентный спуск, \pageref{term:gradient_spusk}

грань блока, \pageref{term:block_facet}

граф вычислительного кластера, \pageref{term:graph_cluster}

граф дуальный, \pageref{term:dual_graph}, \pageref{term:dual_graph2}, \pageref{term:dual_graph3}, \pageref{term:dual_graph4}

граф кубический, \pageref{term:graph_cubic}

граф потока управления, \pageref{term:graph_cfg}, \pageref{term:graph_cfg2}, \pageref{term:graph_cfg3}

граф расчетной задачи, \pageref{term:graph_task}

двухцветный цикл, \pageref{term:bicolor_cycle}

деформация системы линейных неравенств, \pageref{term:deform_sys_lin_neravenstv}, \pageref{term:deform_sys_lin_neravenstv2}

домен, \pageref{term:domain}, \pageref{term:domain2}, \pageref{term:domain3}, \pageref{term:domain4}, \pageref{term:domain5}, \pageref{term:domain6}

интерфейс касания блоков расчетной сетки, \pageref{term:block_interface}

критический путь исполнения, \pageref{term:critical_path}, \pageref{term:critical_path2}

метод векторизации с помощью комбинирования масок, \pageref{term:meth_vec_comb}

метод векторизации с помощью объединением масок, \pageref{term:meth_vec_union}, \pageref{term:meth_vec_union2}

метод векторизации с помощью слияния путей исполнения, \pageref{term:meth_vec_merge}, \pageref{term:meth_vec_merge2}, \pageref{term:meth_vec_merge3}

метод векторизации с помощью удаления маловероятных регионов, \pageref{term:meth_vec_del_low_prob_regions}, \pageref{term:meth_vec_del_low_prob_regions2}

метод векторизации с проверкой векторных масок, \pageref{term:meth_vec_check}, \pageref{term:meth_vec_check2}, \pageref{term:meth_vec_check3}

метод Годунова, \pageref{term:godunov_method}, \pageref{term:godunov_method2}

метод многослойного перестроения поверхности, \pageref{term:method_remesh_multi}

метод окрестностей перестроения поверхности, \pageref{term:method_remesh_okr}

метод пирамид перестроения поверхности, \pageref{term:method_remesh_pyramid}

метод погруженной границы, \pageref{term:immersed_boundary_method}, \pageref{term:immersed_boundary_method2}, \pageref{term:immersed_boundary_method3}

метод призм перестроения поверхности, \pageref{term:method_remesh_prism}

метод прямоугольников перестроения поверхности, \pageref{term:method_remesh_rect}

метод Тонга перестроения поверхности, \pageref{term:method_remesh_tong}

метод трапеций перестроения поверхности, \pageref{term:method_remesh_trap}

метод устранения самопересечений с восстановлением, \pageref{term:method_selfint_repare}

метод устранения самопересечений с дроблением, \pageref{term:method_selfint_cut}

мутация, \pageref{term:mutation}

область блока, \pageref{term:block_scope}

обратимое преобразование векторной маски, \pageref{term:obratim_preobr}

окрестность, \pageref{term:okrestnost}

опортная вершина в алгоритме декомпозиции, \pageref{term:opor_point}, \pageref{term:opor_point2}

оптимизация <<черная дыра>>, \pageref{term:blackhome_optimization}

пиковая производительность, \pageref{term:peak_performance}

плотность векторной маски, \pageref{term:vector_mask_density}, \pageref{term:vector_mask_density2}, \pageref{term:vector_mask_density3}, \pageref{term:vector_mask_density4}, \pageref{term:vector_mask_density5}, \pageref{term:vector_mask_density6}

показатель максимальной длины границы между доменами, \pageref{term:decomp_maxbord}, \pageref{term:decomp_maxbord2}, \pageref{term:decomp_maxbord3}, \pageref{term:decomp_maxbord4}, \pageref{term:decomp_maxbord5}

показатель неравномерность декомпозиции, \pageref{term:decomp_neravn}, \pageref{term:decomp_neravn2}, \pageref{term:decomp_neravn3}, \pageref{term:decomp_neravn4}, \pageref{term:decomp_neravn5}, \pageref{term:decomp_neravn6}, \pageref{term:decomp_neravn7}

показатель суммарной длины границы между доменами, \pageref{term:decomp_sumbord}, \pageref{term:decomp_sumbord2}, \pageref{term:decomp_sumbord3}, \pageref{term:decomp_sumbord4}, \pageref{term:decomp_sumbord5}

популяция, \pageref{term:population}

предикатное исполнение, \pageref{term:predicate_execution}

предикатное представление, \pageref{term:predicate_view}, \pageref{term:predicate_view2}, \pageref{term:predicate_view3}, \pageref{term:predicate_view4}

произведение Адамара, \pageref{term:hadamar_mul}

профиль исполнения, \pageref{term:execution_profile}, \pageref{term:execution_profile2}, \pageref{term:execution_profile3}

раскраска Тейта, \pageref{term:coloring_tait}

расчетная сетка адаптивная локально-измельчающаяся, \pageref{term:mesh_adaptive}

расчетная сетка блочно-структурированная, \pageref{term:mesh_block_struct2}, \pageref{term:mesh_block_struct3}, \pageref{term:mesh_block_struct4}, \pageref{term:mesh_block_struct5}, \pageref{term:mesh_block_struct6}

расчетная сетка декартова, \pageref{term:mesh_descartes}, \pageref{term:mesh_descartes2}, \pageref{term:mesh_descartes3}

расчетная сетка несогласованная, \pageref{term:mesh_nesoglas}

расчетная сетка неструктурированная поверхностная, \pageref{term:unstruct_surf_calc_mesh}, \pageref{term:unstruct_surf_calc_mesh2}, \pageref{term:unstruct_surf_calc_mesh3}, \pageref{term:unstruct_surf_calc_mesh4}, \pageref{term:unstruct_surf_calc_mesh5}, \pageref{term:unstruct_surf_calc_mesh6}

расчетная сетка простая, \pageref{term:mesh_simple}

расчетная сетка согласованная, \pageref{term:mesh_soglas}

расщепление цикла по индуктивной переменной, \pageref{term:loop_split_by_inductive}

расщепление цикла по условию, \pageref{term:loop_split_by_cond}, \pageref{term:loop_split_by_cond2}, \pageref{term:loop_split_by_cond3}

ребро кроссдоменное, \pageref{term:edge_cross}, \pageref{term:edge_cross2}, \pageref{term:edge_cross3}, \pageref{term:edge_cross4}

ребро междоменное, см. ребро кроссдоменное

решатель римановский, \pageref{term:riemann_solver}, \pageref{term:riemann_solver2}, \pageref{term:riemann_solver3}, \pageref{term:riemann_solver4}, \pageref{term:riemann_solver5}, \pageref{term:riemann_solver6}, \pageref{term:riemann_solver7}

самопересечение расчетной сетки, \pageref{term:mesh_self_intersect}

свертывание системы линейных неравенств, \pageref{term:method_svert_sys_neravenstv}, \pageref{term:method_svert_sys_neravenstv2}

сглаживание в нулевом пространстве, \pageref{term:smooth_null}

сглаживание высот, \pageref{term:smooth_height}

сглаживание нормалей, \pageref{term:smooth_norm}

сильное масштабирование вычислений, \pageref{term:strong_scale}

скрещивание, \pageref{term:crossover}

сортировка Шелла, \pageref{term:shell_sort}, \pageref{term:shell_sort2}

стратегия распараллеливания CHUNKS, \pageref{term:parallel_strategy_chunks}

стратегия распараллеливания INTERLEAVE, \pageref{term:parallel_strategy_interleave}

стратегия распараллеливания RACE, \pageref{term:parallel_strategy_race}

схема Стегера-Уорминга, \pageref{term:steger_warming_scheme}, \pageref{term:steger_warming_scheme2}

счетчик ребра CFG, \pageref{term:counter_edge}

счетчик узла CFG, \pageref{term:counter_node}

теневой слой, \pageref{term:block_shadow_layer}, \pageref{term:block_shadow_layer2}

ускорение при распаралленивании на общей памяти, \pageref{term:shr_speedup}

ускорение при распараллеливании с передачей сообщений, \pageref{term:msg_speedup}, \pageref{term:msg_speedup2}, \pageref{term:msg_speedup3}

ускорение при векторизации, \pageref{term:vec_speedup}

функция-интринсик, \pageref{term:intrinsic}, \pageref{term:intrinsic2}, \pageref{term:intrinsic3}, \pageref{term:intrinsic4}, \pageref{term:intrinsic5}, \pageref{term:intrinsic6}, \pageref{term:intrinsic7}

целочисленный программный контекст, \pageref{term:integer_context}

цикл плоский, \pageref{term:flat_loop}, \pageref{term:flat_loop2}, \pageref{term:flat_loop3}, \pageref{term:flat_loop4}, \pageref{term:flat_loop5}

цикл квазиплоский, \pageref{term:flat_kvazy_flat}

шаблон аппроксимации в методе погруженных границы, \pageref{term:ibm_template}

шаблон сглаживания границы между доменами, \pageref{term:smooth_template}

ширина векториазации, \pageref{term:vec_shir}, \pageref{term:vec_shir2}, \pageref{term:vec_shir3}, \pageref{term:vec_shir4}

штрафная функция, \pageref{term:penalty_function}

эффективность распараллеливания на общей памяти, \pageref{term:shr_eff}

эффективность распараллеливания с передачей сообщений, \pageref{term:msg_eff}, \pageref{term:msg_eff2}, \pageref{term:msg_eff3}

эффективность векторизации, \pageref{term:vec_eff0}, \pageref{term:vec_eff}, \pageref{term:vec_eff2}, \pageref{term:vec_eff3}, \pageref{term:vec_eff4}, \pageref{term:vec_eff5}, \pageref{term:vec_eff6}, \pageref{term:vec_eff7}

эпилог, \pageref{term:epilog}, \pageref{term:epilog2}

ячейка блока или домена внутреняя, \pageref{term:cell_block_inner}, \pageref{term:cell_block_inner2}

ячейка блока или домена граничная, \pageref{term:cell_block_border}, \pageref{term:cell_block_border2}

ячейка блока интерфейсная, \pageref{term:cell_block_interface}

ячейка блока или домена фиктивная, \pageref{term:cell_block_ghost}, \pageref{term:cell_block_ghost2}

ячейка блока или домена MPI обмена, \pageref{term:cell_block_mpi}, \pageref{term:cell_block_mpi2}

ячейка внешняя в методе погруженной границы, \pageref{term:cell_ibm_outer}, \pageref{term:cell_ibm_outer2}

ячейка внутренняя в методе погруженной границы, \pageref{term:cell_ibm_innner}, \pageref{term:cell_ibm_innner2}

ячейка граничная в методе погруженной границы, \pageref{term:cell_ibm_border}, \pageref{term:cell_ibm_border2}

ячейка пересечения при поиске самопересечений, \pageref{term:cell_intersect}

ячейка скрытая при поиске самопересечений, \pageref{term:cell_hidden}

ячейка статическая при поиске самопересечений, \pageref{term:cell_static}

ячейка фиктивная в методе погруженной границы, \pageref{term:cell_ibm_ghost}, \pageref{term:cell_ibm_ghost2}, \pageref{term:cell_ibm_ghost3}, \pageref{term:cell_ibm_ghost4}
