\paragraph{Теоретическая и практическая значимость} \

Разработанный метод перестроения поверхностной расчетной сетки основан на использовании окрестностей геометрических объектов и может быть применен в различных предметных областях.
Методы удаления самопересечений поверхностной расчетной сетки сформулированы в геометрической постановке и могут быть использованы без учета специфики проблемы моделирования ледообразования.
Методы повышения производительности параллельных вычислений в моделях распараллеливания с передачей сообщений, на общей памяти и на уровне отдельных инструкций разработаны преимущественно без привязки к предметной области и могут быть использованы как отдельные самостоятельные результаты.

Предложенные в работе методы и алгоритмы апробированы на суперкомпьютерах Межведомственного суперкомпьютерного центра Российской академии наук и Национального центра <<Курчатовский институт>> и реализованы в рамках инструментов компьютерного моделирования, на которые оформлены 7 свидетельств о государственной регистрации программы для ЭВМ.
В частности, разработанные в рамках диссертации методы перестроения поверхностной неструктурированной расчетной сетки и устранения самопересечений, а также методы повышения производительности вычислений на ней нашли свое отражение в программном модуле компьютерного моделирования процесса обледенения элементов авиационных силовых установок <<Кристалл>>.