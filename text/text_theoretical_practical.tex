\paragraph{Теоретическая и практическая значимость} \

Метод окрестностей перестроения поверхностной расчетной сетки и методы удаления самопересечений вносят теоретический вклад в вычислительную геометрию и могут быть использованы для развития механизмов эволюции сетки в задаче моделирования обледенения.
Алгоритмы распределения блоков расчетной сетки с использованием дробления блоков вносят теоретический вклад в решение задачи выбора наилучшего разбиения.
Алгоритм сглаживания границ между доменами поверхностной расчетной сетки вносит теоретический вклад в решение задачи декомпозиции и может быть использован для обобщения на объемный случай.
Методы векторизации программного кода вносят теоретический вклад в теорию эквивалентных преобразований программ и могут быть использованы в инструментах компиляции программ.

Предложенные в диссертации методы и алгоритмы апробированы на суперкомпьютерах Межведомственного суперкомпьютерного центра Российской академии наук и Национального исследовательского центра <<Курчатовский институт>> и реализованы в рамках инструментов компьютерного моделирования, на которые получены 7 свидетельств о государственной регистрации программ для ЭВМ.
Разработанные методы и алгоритмы реализованы в программном модуле компьютерного моделирования процесса обледенения элементов авиационных силовых установок <<Кристалл>> и внедрены в Национальном исследовательском центре <<Курчатовский институт>>, Центральном институте авиационного моторостроения имени П.И.~Баранова и холдинге <<Вертолеты России>> для использования при выполнении научно-исследовательских и опытно-конструкторских работ.