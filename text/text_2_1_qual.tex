Для решения крупных расчетных задач необходимо использование современных высокопроизводительных вычислительных систем.
Такие системы, как правило, состоят из множества отдельных вычислительных узлов, соединенных между собой каналами связи.
Таким образом, для решения задачи на высокопроизводительной вычислительной системе необходимо сначала разделить ее данные на отдельные подобласти, распределить эти данные по узлам вычислительной системы и организовать обмен сообщениями для учета взаимодействия подобластей \cite{GOST57700HPC}.
Использование вычислительных систем с большим количеством вычислительных узлов продиктовано стремлением сократить время выполнения расчетов.
Исходя из этого основным показателем качества организации высокопроизводительных вычислений является ускорение выполнения задачи при увеличении количества задействованных вычислительных узлов.

\begin{definition}
Ускорением выполнения задачи в модели распараллеливания с передачей сообщений при использовании $k$ узлов высокопроизводительной вычислительной системы будем называть величину $s_{msg}(k) = \frac{T(1)}{T(k)}$, где $T(i)$ -- время выполнения задачи на $i$ вычислительных узлах. 
\end{definition}

В случае идеального сильного масштабирования вычислений $s_{msg}(k) = k$, то есть при увеличении количества узлов в $x$ раз время выполнения задачи сокращается также в $x$ раз (в этом случае будем говорить, что наблюдаестся линейное ускорение вычислений).
Наряду с ускорением выполнения задачи при распараллеливании будем использовать характеристику эффективности распараллеливания.

\begin{definition}
Эффективностью распараллеливания при выполнения задачи в модели с передачей сообщений при использлвании $k$ узлов высокопроизводительной вычислительной системы будем называть величину $e_{msg}(k) = \frac{s_{msg}(k)}{k}$.
\end{definition}

В случае идеального сильного масштабирования вычислений $e_{msg}(k) = 1$.
В дальнейшем будем использовать показатели $s_{msg}$ и $e_{msg}$ для оценки распараллеливания вычислений в модели с обменом сообщениями.

% Показатели эффективности декомпозиции сетки.
\subsection{Показатели качества декомпозиции расчетной сетки}

Высокопроизводительные вычисления, выполняющиеся на расчетных сетках, как правило, состоят из большого количества отдельных итераций по времени, на каждой из которых обрабатываются все ячейки сетки, а между итерациями подобласти расчетной задачи обмениваются данными расчетных ячеек, находящихся вблизи общих границ.
Сначала рассмотрим одну итерацию расчетов без учета информационных обменов.
Рассмотрим расчетную сетку, содержащую $n$ ячеек, и которую требуется декомпозировать на $k$ подобластей для обработки на $k$ вычислительных узлах.
Все ячейки расчетной сетки могут обрабатываться параллельно.
Будем считать, что все ячейки являются одинаковыми с точки зрения времени их обработки.
Если разбить расчетную сетку на $k$ доменов с одинаковым количеством ячеек и обрабатывать каждый домен на отдельном процессоре, то обработка каждого домена на одной итерации будет занимать одно и то же время.
Так как обработка всех доменов будет производиться параллельно, то это время и будет временем, затрачиваемым на одну итерацию обработки всех ячеек сетки.
Если распределение ячеек сетки по доменам будет неравномерным, то время выполнения одной итерации расчетов будет определяться самым крупным доменом (так как время его обработки будет максимальным).
Таким образом, в качестве показателя качества декомпозиции сетки можно принять максимальное абсолютное отклонение размера домена от теоретически возможного среднего значения
\begin{equation}
	D = \max_{1 \le i \le k}{ \left( n_i - \frac{n}{k} \right) },
\end{equation}
 
где $n_i$ – количество ячеек в $i$-ом домене.
В идеальном случае равномерного распределения ячеек по доменам показатель $D$ становится равным нулю.
В худшем случае все ячейки распределяются в один домен и $D = n \left( 1 - \frac{1}{k} \right)$.

Так как абсолютный показатель не слишком удобен для сравнения разных алгоритмов декомпозиции на разных сетках, то дополнительно будем использовать показатель, который характеризует долю накладных расходов неравномерного распределения ячеек по доменам по отношении к идеальному распределению, а именно
\begin{equation}
	D^{*} = \frac{D}{n / k}
\end{equation}

Значение $D^{*}$ изменяется в диапазоне $[0, k - 1]$, где $D^{*} = 0$ означает идеальное распределение ячеек по доменам, а $D^{*} = k - 1$ соответствует наихудшему случаю, когда все ячейки отнесены к одному домену.
Также будем использовать величину $D^{\%} = D^{*} \cdot 100\%$ для возможности указания значения показателя в процентах.

\begin{definition}
Показатели $D$, $D^{*}$, $D^{\%}$ будем называть показателями неравномености декомпозиции.
\end{definition} 

После завершения итерации расчетов требуется произвести информационные обмены данными на границах доменов.
Так как мы считаем, что каждый домен обрабатывается на своем процессоре, то информационный обмен происходит с использованием механизмов межпроцессного взаимодействия (например, с использованием MPI).
Таким образом, для каждой пары доменов, имеющих общую границу необходимо организовывать межпроцессный обмен.
Все такие обмены могут происходить одновременно, и общее затрачиваемое на них время определяется длиной максимальной общей границы между доменами.
Для учета информационных обменов рассмотрим дуальный граф расчетной сетки, вершинами которого являются ячейки сетки.
Две вершины в дуальном графе соединены ребром, если две соответствующие ячейки являются соседними (имеют общую грань).
Множество ребер дуального графа обозначим через $E$, а для каждого отдельного ребра $e$ под $e_a$ и $e_b$ будем понимать инцидентные ему вершины.
Тогда в качестве второй характеристики качества декомпозиции будем использовать величину наиболее протяженной границы между парой доменов, или
\begin{equation}
	\begin{aligned}
		& L_{ij} = \left| \{ e \in E: \{ d(e_a), d(e_b) \} = \{ i, j \} \} \right| \\
		& L = \max_{1 \le i < j \le k}{L_{ij}}
	\end{aligned}
\end{equation}

где $d(v)$ -- домен, к которому относится вершина $v$.
В идеальном случае значение характеристики $L$ может быть сколь угодно малым, даже нулевым (в случае, если сетка представляет собой $k$ одинаковых по количеству ячеек несвязных областей).
В наихудшем случае случае $L = |E|$, если ячейки распределены в два домена в шахматном порядке, и каждое ребро относится к границе между этими двумя доменами.

Наряду с показателем $L$ используется нормированный показатель $L^{*}$, определяемый как
\begin{equation}
	L^{*} = \frac{L}{|E|},
\end{equation}

который изменяется в диапазоне $[0, 1]$ и принимает значение 0 в идельном случае, а 1 -- в наихудшем.
Также будем использовать показатель длины максимальной границы между доменами в процентах от общего количества ребер $L^{\%} = L^{*} \cdot 100\%$.

\begin{definition}
Показатели $L$, $L^{*}$, $L^{\%}$ будем называть показателями максимальной длины границы между доменами.
\end{definition}

В роли дополнительной характеристики качества декомпозиции можно использовать суммарную длину границ между доменами
\begin{equation}
	I = \sum_{1 \le i < j \le k}{L_{ij}},
\end{equation}

что соответствует общему количеству пересылаемых данных в рамках межпроцессного обмена.

\begin{definition}
Междоменным, или кроссдоменным ребром дуального графа будем называть ребро, концы которого относятся к разным доменам.
\end{definition}

Так как домены граничат между собой по ребрам дуального графа, то суммарная длина всех границ между доменами представляет собой просто количество всех ребер, инцидентных двум относящимся к разным доменам ячейкам, то есть количество междоменных ребер.
Наряду с показателем $I$ будем использовать долю междоменных ребер среди общего количества ребер сетки
\begin{equation}
	I^{*} = \frac{I}{|E|}
\end{equation}

Показатель $I^{*}$ также является нормированным, он изменяется в диапазоне $[0, 1]$ и принимает значение 0 в идеальном случае и 1 -- в наихудшем.
Также будем использовать показатель количества междоменных ребер в процентах от общего количества ребер $I^{\%} = I^{*} \cdot 100\%$.

\begin{definition}
Показатели $I$, $I^{*}$, $I^{\%}$ будем называть показателями суммарной длины границ между доменами.
\end{definition}

Для разных задач перечисленные характеристики качества декомпозиции могут иметь различную важность.
Как правило, обработка ячеек занимает основное время высокопроизводительных расчетов, и параметр $D$ более критичен.
Однако, в некоторых случаях вычислительная процедура настолько легковесная, что основные расходы приходятся на межпроцессные обмены.
Например, в работе \cite{Tong2017Remesh} описан алгоритм сглаживания поверхностной расчетной сетки, для которого при распараллеливании по MPI расходы на информационные обмены значительно превышают расходы на вычисления внутри ячеек.
Вместо отдельных показателей качества декомпозиции, можно использовать единый показатель с весовыми коэффициентами и с произвольным набором показателей $D$, $D^{*}$, $D^{\%}$, $L$, $L^{*}$, $L^{\%}$, $I$, $I^{*}$, $I^{\%}$, например
\begin{equation}
	Q(\delta, \lambda, \iota) = \delta D^{*} + \lambda L^{*} + \iota I^{*}
\end{equation}

При выборе алгоритма декомпозиции следует стремиться к уменьшению значения показателя $Q$.
