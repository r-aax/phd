\paragraph{Положения, выносимые на защиту}
\begin{itemize}[noitemsep,topsep=0pt,parsep=0pt,partopsep=0pt]
\item Метод окрестностей перестроения поверхностной неструктурированной расчетной сетки обеспечивает сглаживание ее дефектов, что позволяет повысить стабильность расчетов.
\item Методы удаления самопересечений поверхностной не\-структурированной расчетной сетки позволяют сохранить корректное состояние внешней поверхности при эволюции сетки.
\item Архитектура объемной блочно-структурированной расчетной сетки и алгоритм распределения ее блоков по вычислительным процессам с использованием дробления блоков позволяет повысить равномерность распределения вычислительной нагрузки при распараллеливании вычислений в модели с передачей сообщений.
\item Алгоритм сглаживания границ между доменами поверхностной неструктурированной расчетной сетки обеспечивает нахождение точного решения и позволяет уменьшить объем пересылаемых данных во время межпроцессных обменов при распараллеливании вычислений в модели с передачей сообщений.
\item Методика векторизации программного кода и методы повышения производительности расчетов при помощи векторизации обеспечивают кратное ускорение на широком классе приложений для высокопроизводительных вычислений.
\end{itemize}