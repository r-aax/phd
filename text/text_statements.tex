\item Разработанный метод перестроения поверхностной неструктурированной расчетной сетки обеспечивает сглаживание ее дефектов (острых пиков и впадин), что позволяет повысить стабильность расчетов при моделировании ледообразования.
\item Разработанные методы удаления самопересечений поверхностной не\-структурированной расчетной сетки позволяют избежать аварийного завершения расчетов при моделировании ледообразования.
\item Разработанная архитектура объемной блочно-структурированной расчетной сетки и алгоритм распределения ее блоков по вычислительным процессам с использованием дробления блоков позволяет повысить равномерность распределения вычислительной нагрузки при распараллеливании вычислений в модели с передачей сообщений.
\item Разработанный алгоритм сглаживания границ между доменами поверхностной неструктурированной расчетной обеспечивает нахождение точного решения и уменьшает объем пересылаемых данных во время межпроцессных обменов при распараллеливании вычислений в модели с передачей сообщений.
\item Разработанная методика векторизации программного кода и методы повышения производительности расчетов при помощи векторизации обеспечивают кратное ускорение на широком классе приложений для высокопроизводительных вычислений.