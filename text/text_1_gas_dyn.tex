\subsection{Сопряжение с конечно-объемными газодинамическими решателями}

При расчете обледенения крайне важным элементом данных является поле скоростей в области, окружающей рассматриваемую поверхность.
Поле скоростей вблизи поверхности влияет на движение жидкой пленки по поверхности, что существенным образом определяет картину обледенения.
Также при использовании модели вторичного увлажнения отскочившие от поверхности капли попадают в свободный поток, и поле скоростей необходимо для расчета траектории их движения и зоны вторичного выпадения.

При выборе газадинамического решателя для сопряжения с расчетным кодом обледения на поверхностной сетке стоит обратить внимание на следующие моменты.
Газодинамический решатель является внешним источником данных для расчетов обледенения поверхности, он запускается довольно редко и вычисленное им поле скоростей используется в неизменном виде несколько сотен или тысяч итераций.
Таким образом, при расчете поля скоростей нет необходимости учитывать турбулентность, а реализацию решателя следует выбрать исходя из требования минимизации времени его работы.
Учитывая это, использовались газодинамические решатели, работающие с трехмерными блочно-структурированными расчетными сетками, ввиду их быстродействия.
Однако, при использовании блочно-структурированных расчетных сеток воникла проблема согласования этих сеток с поверхностью, на которой расчитывается обледенение.
Так как перестроение поверхности в процессе обледенения существенно изменяет ее геометрию, то это приводит к проблемам перестроения объемных сеток для газодинамического решателя.

Приведем краткое описание газодинамических решателей, использованных для вычисления поля скоростей вокруг поверхностной сетки при расчете обледенения.



\cite{Blazek2015CFD}
\cite{Duben2014Gas}
