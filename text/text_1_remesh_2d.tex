% Перестроение в двумерном случае.
\subsection{Методы перестроения поверхности в двумерном \\ случае}

Рассмотрим геометрическую задачу о перестроении поверхности в двумерном пространстве в общем виде \cite{Rybakov2019Geo2D}.
Пусть дана поверхностная сетка в виде ломаной без самопересечений, каждая ячейка которой представлена отрезком длины $l_i$ ($0 \le i < n$).
У $i$-ой ячейки инцидентными узлами являются узлы с номерами $i$ и $i + 1$.
Если узлы с номерами $0$ и $n$ совпадают, то сетка является замкнутой.
Известно направление нормали каждой ячейки, а также направление движения каждого узла, совпадающее с направлением суммы единичных нормалей, проведенных к инцидентным ячейкам.
Для двумерного случая это направление лежит на биссектрисе угла, образованного двумя инцидентными ячейками \cite{Fortin2004Remesh2d} (рис.~\ref{fig:text_1_remesh_2d_grid_normals}).
Угол между двумя соседними ячейками с номерами $i - 1$ и $i$ будем обозначать через $2 \phi_i$.

\begin{figure}[ht]
\centering
\includegraphics[width=0.7\textwidth]{pics/text_1_remesh_2d/grid_normals.pdf}
\singlespacing
\captionstyle{center}\caption{Поверхностная сетка с обозначенными направлениями движения узлов.}
\label{fig:text_1_remesh_2d_grid_normals}
\end{figure}

Пусть известно, что за некий малый промежуток времени $i$-ая ячейка сдвигается в направлении своей нормали на некоторую величину $H_i$, заметая таким образом площадь $T_i = l_i H_i$, эту площадь будем называть целевой площадью.
Однако если просто сдвинуть каждую ячейку в направлении своей нормали, то сетка потеряет целостность, поэтому мы можем осуществлять движение только узлов.
Требуется найти такие значения локальных сдвигов узлов сетки $h_i$ ($0 \le i \le n$), чтобы заметаемая площадь $S_i$ между исходной поверхностью и новой поверхностью для каждой ячейки сетки как можно меньше отличалась от значения целевой площади $T_i$.
Для оценки отклонения фактической заметаемой площади от целевой будем использовать обозначения $\Delta_i = S_i - T_i$, $\delta_i = \frac{\Delta_i}{T_i}$.

В применении к задаче ледообразования величина $H_i$ соответствует толщине образозавшегося слоя льда в $i$-ой ячейке.
Новая перестроенная поверхностная сетка соответствует поверхности ледяного покрова, а общая заметаемая площадь при движении поверхности соответствует объему накопленного на поверъности льда.

Для решения поставленной задачи сначала требуется вычислить заметаемую площадь для каждой отдельной ячейки.

\subsubsection{Задача о вычислении заметаемой площади при движении узлов отдельной ячейки}

Рассматривается ячейка, представленная на плоскости отрезком $AB$ длины $l$.
При перемещении точек $A$ и $B$ в новые точки $A_1$ и $B_1$ соответственно образуется четырехугольник $AA_1B_1B$.
Требуется найти его площадь, выраженную явно через параметры $a = AA_1$ и $b = BB_1$ и углы при вершинах $A$ и $B$ (рис.~\ref{fig:text_1_remesh_2d_local}).

\begin{figure}[ht]
\centering
\begin{tabular}{ll}
	\includegraphics[width=0.45\textwidth]{pics/text_1_remesh_2d/local.pdf}
	&
	\includegraphics[width=0.38\textwidth]{pics/text_1_remesh_2d/local2.pdf}
\end{tabular}
\singlespacing
\captionstyle{center}\caption{Вычисление заметаемой площади через смещения узлов ячейки.}
\label{fig:text_1_remesh_2d_local}
\end{figure}

\begin{lemma}\label{lem:text_1_remesh_2d_AA1B1B}
Пусть дан четырехугольник $AA_1B_1B$, в котором $AA_1 = a$, $BB_1 = b$, $\angle A_1AB = \alpha$, $\angle B_1BA = \beta$.
Тогда площадь этого четырехугольника выражается формулой
\begin{equation}\label{eqn:text_1_remesh_2d_one_cell_lemma}
S_{AA_1B_1B} = \frac{1}{2} \left( l (a \sin \alpha + b \sin \beta) - ab \sin (\alpha + \beta) \right)
\end{equation}
\end{lemma}

Опустим перпендикуляры из точек $A_1$ и $B_1$ на прямую $AB$.
Их основаниями будут точки $A_2$ и $B_2$ соответственно.
Искомая площадь может быть представлена в следующем виде:
\begin{equation}
S_{AA_1B_1B} = S_{A_2A_1B_1B_2} - S_{AA_1A_2} - S_{BB_1B_2}
\end{equation}

Ее можно выразить явно через значения углов $\alpha$ и $\beta$:
\begin{equation}
S_{AA_1B_1B} = \frac{1}{2}(l - a \cos \alpha - b \cos \beta)(a \sin \alpha + b \sin \beta) + \frac{1}{2}a^2 \sin \alpha \cos \alpha + \frac{1}{2}b^2 \sin \beta \cos \beta
\end{equation}

После раскрытия скобок и преобразований получим \eqref{eqn:text_1_remesh_2d_one_cell_lemma}.
$\blacksquare$\\

Отметим, что если $\alpha + \beta \ge \pi$ то $S_{AA_1B_1B}$ всегда положительна и возрастает с ростом $a$ или $b$.
Это означает, что лучи $AA_1$ и $BB_1$ не пересекаются (этот случай изображен на рис.~\ref{fig:text_1_remesh_2d_local}).

Рассмотрим случай $\alpha + \beta < \pi$.
Запишем отдельно условия неубывания $S_{AA_1B_1B}$ при росте $a$ и $b$:
\begin{equation}
	\begin{aligned}
		& \frac{\partial S_{AA_1B_1B}}{\partial a} = \frac{1}{2}(l \sin \alpha - b \sin (\alpha + \beta)) \ge 0 \\
		& \frac{\partial S_{AA_1B_1B}}{\partial b} = \frac{1}{2}(l \sin \beta - a \sin (\alpha + \beta)) \ge 0
	\end{aligned}
\end{equation}

или
\begin{equation}\label{eqn:text_1_remesh_2d_cond_abl}
	\begin{aligned}
		& a \le \frac{l \sin \beta}{\sin (\alpha + \beta)} \\
		& b \le \frac{l \sin \alpha}{\sin (\alpha + \beta)}
	\end{aligned}
\end{equation}

Если выполняются оба условия из \eqref{eqn:text_1_remesh_2d_cond_abl}, то четырехугольник $AA_1B_1B$ продолжает оставаться выпуклым.
Если выполнятеся только одно условие из \eqref{eqn:text_1_remesh_2d_cond_abl}, то четырех угольник стал невыпуклым.
Если не выполняется ни одно из условий \eqref{eqn:text_1_remesh_2d_cond_abl}, то произовшло самопересечение сетки, то есть оба отрезка $AA_1$ или $BB_1$ вышли за пределы треугольника $ABO$, где $O$ -- точка пересечения лучей $AA_1$ и $BB_1$.
Случаи самопересечения должны быть обработаны отдельно для возможности продолжения работы с расчетной сеткой.

\subsubsection{Поиск смещений узлов методом градиентного спуска}

Метод градиентного спуска\label{term:gradient_spusk} является одним из наиболее простых методов оптимизации для нахождения локального минимума функции.
При условии, что в любой точке функции можно вычислить ее градиент, то начиная с некоторого начального приближения $x_0$ строится итерационная последовательность~\cite{Kantorovich1984Func}:
\begin{equation}
x^{k+1} = x^k - \gamma_k \nabla f(x_k)
\end{equation}

где $\gamma_k \ge 0$ задает длину шага и, соответственно, скорость градиентного спуска.

Метод градиентного спуска находит свое основное применение в задаче поиска минимума или максимума функции.
Направление антиградиента является направлением наискорейшего убывания функции.
Основная проблема метода заключается в выборе шага $\gamma$.
При слишком больших значениях шага существует вероятность миновать искомый минимум функции.
К тому же, метод не гарантирует нахождение глобального минимума.

Рассмотрим решение задачи определения величин смещения узлов расчетной сетки при перестроении методом градиентного спуска.
Неизвестными параметрами являются величины сдвигов узлов сетки $h_i$ для $0 \le i \le n$.
Опираясь на решение локальной задачи об определении заметаемой площади из леммы~\ref{lem:text_1_remesh_2d_AA1B1B}, можно записать заметаемую площадь при движении отдельной ячейки (см. рис.~\ref{fig:text_1_remesh_2d_local} справа):
\begin{equation}
S_i = \frac{1}{2}\big(l_i(h_i \sin \phi_i + h_{i + 1} \sin \phi_{i+1}) - h_ih_{i + 1} \sin(\phi_i + \phi_{i+1})\big) 
\end{equation}

Для нахождения решения задачи перестроения сетки в двумерном пространстве в соответствие с методом наименьших квадратов будем искать минимум функции
\begin{equation}
D(h_0, \dots, h_{n}) = \sum_{i = 0}^{n - 1}{\Delta_i^2}
\end{equation}

Для нахождения градиента требуется вычислить частные производные $\frac{\partial D}{\partial h_i}$.
Эти частные производные можно записать в явном виде:
\begin{equation}
\frac{\partial D}{\partial h_i} = \frac{\partial (\Delta_{i - 1})^2}{\partial h_i} + \frac{\partial (\Delta_i)^2}{\partial h_i}
\end{equation}

где
\begin{equation}
\begin{cases}
\frac{\partial (\Delta_{i - 1})^2}{\partial h_i} = \Delta_{i - 1} (l_{i - 1} \sin \phi_i - h_{i - 1} \sin(\phi_{i - 1} + \phi_i)) \\
\frac{\partial (\Delta_i)^2}{\partial h_i} = \Delta_i (l_i \sin \phi_i - h_{i + 1} \sin(\phi_i + \phi_{i+1}))
\end{cases}
\end{equation}

\

Также при осуществлении метода градиентного спуска требуется следить за соблюдением дополнительных условий, которые накладываются на неизвестные $h_i$.
Например, очевидным условием является выполнение соотношения $h_i \ge 0$, что запрещает движение сетки в отрицательном направлении.
Также можно использовать более строгие условия $\min(H_{i - 1}, H_i) \le h_i \le \max(H_{i - 1}, H_i)$, которые не позволят величинам смещения узлов сетки выходить за пределы смещений инцидентных им ячеек сетки.

\subsubsection{Приближенные методы перестроения поверхности}

Решение задачи о перестроении сетки методом градиентного спуска оказывается слишком требовательным к вычислительным ресурсам при увеличении размера сетки.
К тому же, качество решения зачастую оказывается неудовлетворительным при попадании в локальные минимумы.
Поэтому для решения поставленной задачи рассмотрим приближенные методы, основанные на представлении целевой площади в виде примитивных геометрических фигур.

\begin{figure}[ht]
\centering
\begin{tabular}{ll}
\includegraphics[width=0.45\textwidth]{pics/text_1_remesh_2d/remesh_rectangles.pdf}
&
\includegraphics[width=0.45\textwidth]{pics/text_1_remesh_2d/remesh_trapeziums.pdf}
\end{tabular}
\singlespacing
\captionstyle{center}\caption{Перестроение поверхности методом прямоугольников (слева) и трапеций (справа).}
\label{fig:text_1_remesh_2d_rectangles_and_trapeziums}
\end{figure}

В качестве первого метода перестроения поверхности рассмотрим приближение, при котором целевая площадь для $i$-ой ячейки представлена прямоугольником со сторонами $l_i$ и $H_i$.
В качестве величины смещения узла берется среднее арифметичское двух высот инцидентных ячеек $h_i = \frac{H_{i - 1} + H_i}{2}$ (см. рис.~\ref{fig:text_1_remesh_2d_rectangles_and_trapeziums} слева).
Этот метод перестроения поверхности будем называть методом прямогольников\label{term:method_remesh_rect}.
Заметаемую $i$-ой ячейкой площадь при использовании метода прямоугольников будем обозначать $S_i^r$, также обозначим $\Delta_i^r = S_i^r - T_i$, $\delta_i^r = \frac{\Delta_i^r}{T_i}$.

В качестве второго метода перестроения поверхности будем рассматривать метод трапеций\label{term:method_remesh_trap}.
В методе трапеций целевая площадь $i$-ой ячейки представляется трапецией с площадью $T_i = l_i H_i$.
Боковые стороны этой трапеции лежат на направлениях движения двух узлов, инцидентных рассматриваемой ячейке.
Высота трапеции $H_i^t$ находится из \eqref{eqn:text_1_geo_prim_aa1b1b} с помощью решения квадратного уравнения.
После построения трапеций для всех ячеек сетки у каждого узла появляется две новые потенциальные позиции для сдвига (образованные ячейкой слева и ячейкой справа).
В качестве финальной новой позиции выбирается их среднее значение (рис.~\ref{fig:text_1_remesh_2d_rectangles_and_trapeziums} справа).
Таким образом, величина смещения узла определяется как $h_i = \frac{H_{i - 1}^t + H_i^t}{2 \sin \phi_i}$.
Заметаемую $i$-ой ячейкой площадь при использовании метода трапеций будем обозначать $S_i^t$, также обозначим $\Delta_i^t = S_i^t - T_i$, $\delta_i^t = \frac{\Delta_i^t}{T_i}$.

Предлагается новый метод перестроения поверхностной сетки, который будем называть методом окрестностей\label{term:method_remesh_okr3}.
В этом методе будем рассматривать фронт движения сетки (см. рис.~\ref{fig:text_1_remesh_2d_okrestnost}).

\begin{figure}[ht]
\centering
\includegraphics[width=0.6\textwidth]{pics/text_1_remesh_2d/remesh_okrestnost.pdf}
\singlespacing
\captionstyle{center}\caption{Перестроение поверхности методом окрестностей.}
\label{fig:text_1_remesh_2d_okrestnost}
\end{figure}

При этом фронтом движения узла с номером $i$ будем считать сферу с центром в этом узле и радиусом $r_i = \frac{H_{i - 1} + H_i}{2}$.
Под фронтом движения ячейки сетки будем понимать выпуклую оболочку фронтов движения двух инцидентных ей узлов.
Тогда фронтом движения сетки является объединение фронтов движения всех ее ячеек.
В качестве нового положения узла будем брать пересечение направления смещения узла и границы фронта движения сетки.
Заметаемую $i$-ой ячейкой площадь при использовании метода окрестностей будем обозначать $S_i^o$, также обозначим $\Delta_i^o = S_i^o - T_i$, $\delta_i^o = \frac{\Delta_i^o}{T_i}$.

\subsubsection{Теоретические оценки точности методов перестроения сетки}

Приведем теоретическую оценку точности приближенных методов перестроения поверхности.
Оценку будем проводить для модельной расчетной сетки, которая удовлетворяет следующим требованиям.
Все ячейки сетки одинаковые и имеют длину $l$, поверхность является выпуклой.
Для любой ячейки $AB$ и ее соседей $A_2A$ и $BB_2$ углы $\angle (\overline{BA}, \overline{AA_2})$ и $\angle (\overline{AB}, \overline{BB_2})$ являются постоянной величиной и равны $\alpha$ (см. рис.~\ref{fig:text_1_remesh_2d_theoretical}).
Пусть величины смещений ячеек $A_2A$, $AB$ и $BB_2$ равны $H_{i - 1}$, $H_i$ и $H_{i + 1}$ соответственно.

\begin{figure}[ht]
\centering
\includegraphics[width=0.8\textwidth]{pics/text_1_remesh_2d/theoretical.pdf}
\singlespacing
\captionstyle{center}\caption{Теоретическая оценка приближенных методов перестроения выпуклой поверхности в двумерном случае.}
\label{fig:text_1_remesh_2d_theoretical}
\end{figure}

Для вычисления величин $\Delta$, $\delta$ необходимо вычислить площадь $S_i = S_{AA_1B_1B}$.
При использовании разных методов перестроения эта площадь будет различаться.

\begin{lemma}\label{lem:text_1_remesh2_vypukl_lemma}
Для ячейки $AB$ выпуклой сетки с равными углами $\angle (\overline{BA}, \overline{AA_2}) = \angle (\overline{AB}, \overline{BB_2}) = \alpha$ (см. рис.~\ref{fig:text_1_remesh_2d_theoretical})
\begin{equation}\label{eqn:text_1_remesh2_saa2b2b_gen}
S_i(h_i, h_{i + 1}) = \frac{1}{2} \sin \alpha \left( \lambda(h_i + h_{i+1}) + h_ih_{i+1} \right)
\end{equation}
где $\lambda = \frac{l}{2 \sin \frac{\alpha}{2}}$, $l = AB$, $h_i = AA_1$, $h_{i + 1} = BB_1$.
\end{lemma}

Так как рассматриваемая сетка выпуклая, то лучи $A_1A$ и $B_1B$ пересекаются в некоторой точке $O$.
При этом треугольник $AOB$ равнобедренный, $\angle BAO = \angle ABO = \angle B_1BX = \angle B_1BB_2 - \alpha = \frac{\pi - \alpha}{2}$, $\phi_i = \phi_{i + 1} = \frac{\pi + \alpha}{2}$, $\angle AOB = \alpha$, откуда $OA = OB = \lambda = \frac{l}{2 \sin \frac{\alpha}{2}}$.
Далее выражаем площадь $S_i = S_{A_1OB_1} - S_{AOB}$, что после преобразования и дает требуемое соотношение \eqref{eqn:text_1_remesh2_saa2b2b_gen}.
$\blacksquare$\\

Далее будем рассматривать линейное изменение величины смещения ячеек сетки, то есть $H_i = H$, $H_{i - 1} = H - \Delta H$, $H_{i + 1} = H + \Delta H$.

Для перестроения методом прямоугольников имеем $h_i = H - \frac{1}{2} \Delta H$, $h_{i + 1} = H + \frac{1}{2} \Delta H$, откуда
\begin{equation}\label{eqn:text_1_remesh2_s_rect}
	S_i^r = \cos \frac{\alpha}{2} \left( lH + \left( H^2 - \frac{1}{4} \Delta H^2 \right) \sin \frac{\alpha}{2} \right)
\end{equation}

Теперь перейдем к вычислению $S_i^t$ для метода трапеций.
В методе трапеций мы должны представить целевые площади в ячейках $A_2A$, $AB$, $BB_2$ с помощью равнобоких трапеций, для которых известно одно из оснований (одинаково для всех ячеек и равно $l$), угол при этом основании (одинаковый для всех ячеек и равен $\phi = \frac{\pi + \alpha}{2}$, а также площади этих трапеций, равные $T_{i - 1} = l(H - \Delta H)$, $T_i = lH$ и $T_{i + 1} = l(H + \Delta H)$ соответственно.
На основании этих данных нам необходимо вычислить высоты трапеций, что можно сделать согласно \eqref{eqn:text_1_geo_prim_trapezium_h_from_s}.
Обозначим высоты этих трапеций:
\begin{equation}
	\begin{aligned}\label{eqn:text_1_remesh_2_Ht_vypukl}
		& H_{i - 1}^t = H^t\left(l(H - \Delta H), l, \frac{\pi + \alpha}{2}\right) \\ 
		& H_i^t = H^t\left(lH, l, \frac{\pi + \alpha}{2}\right) \\
		& H_{i + 1}^t = H^t\left(l(H + \Delta H), l, \frac{\pi + \alpha}{2}\right)
	\end{aligned}
\end{equation}

Тогда $h_i = \frac{H_{i - 1}^t + H_i^t}{2 \sin \phi_i} = \frac{H_{i - 1}^t + H_i^t}{2 \cos \frac{\alpha}{2}}$, $h_{i+1} = \frac{H_i^t + H_{i+1}^t}{2 \sin \phi_{i + 1}} = \frac{H_i^t + H_{i+1}^t}{2 \cos \frac{\alpha}{2}}$ и выражение \eqref{eqn:text_1_remesh2_saa2b2b_gen} для $S_i^t$ принимает следующий вид:
\begin{equation}\label{eqn:text_1_remesh2_trap}
	S_i^t = \frac{1}{4} \left( (H_{i - 1}^t + 2 H_i^t + H_{i + 1}^t)l + (H_{i - 1}^t + H_i^t) (H_i^t + H_{i+1}^t) \tg \frac{\alpha}{2} \right)
\end{equation}

Для получения теоретической оценки точности для метода окрестностей в случае выпуклой сетки рассмотрим следующее вспомогательное утверждение.

\begin{figure}[ht]
\centering
\includegraphics[width=0.7\textwidth]{pics/text_1_remesh_2d/accuracy_okrestnost.pdf}
\singlespacing
\captionstyle{center}\caption{Вычисление смещения узла в методе окрестностей для выпуклой сетки.}
\label{fig:text_1_remesh_2d_accuracy_okrestnost}
\end{figure}

\begin{lemma}\label{lem:text_1_remesh_2d_accuracy_okrestnost}
Пусть $AB = l$, ниже прямой $AB$ расположена точка $X$ такая, что $\angle (\overline{AX}, \overline{BA}) = \alpha$.
Пусть построены две окружности с центрами в точках $A$ и $B$ и с радиусами $r$ и $r + \Delta r$ соответственно.
Для построенных окружностей построена выпуклая оболочка, состоящая из дуг этих окружностей, а также отрезков общих касательных (точки касания общей касательной построенных окружностей, лежащие выше прямой $AB$, обозначим $A_2$ и $B_2$ соответственно).
Пусть на этой выпуклой оболочке -- на окружности с центром в точке $A$, либо на отрезке $A_2B_2$ -- лежит точка $A_1$ такая, что $\angle XAA_1 = \angle BAA_1 = \phi$, $AA_1 = h$, тогда 
\begin{equation}
	\begin{cases}\label{eqn:text_1_remesh_2d_okr_h}
		h = r, \  \sin \frac{\alpha}{2} > \frac{\Delta r}{l}, \\
		h(\alpha, l, r, \Delta r) = \frac{r}{\cos \left( \arcsin \frac{\Delta r}{l} - \frac{\alpha}{2} \right)}, \  \sin \frac{\alpha}{2} \le \frac{\Delta r}{l}
	\end{cases}
\end{equation}
\end{lemma}

Если точка $A_1$ не находится на отрезке $A_2B_2$, то $h = AA_1 = r$.

Рассмотрим случай, когда $A_1$ находится на отрезке $A_2B_2$, в этом случае $\Delta r > 0$.
Через точку $A$ проведем прямую, параллельную $A_2B_2$, до пересечения с $BB_2$ в точке $H$.
Тогда $\angle B_2OB = \angle HAB = \beta = \arcsin \frac{\Delta r}{l}$.
Так как $\angle A_1AB = \phi = \frac{\pi + \alpha}{2}$, то $\angle OAA_2 + \theta + \phi = \pi$, откуда получим $\theta = \beta - \frac{\alpha}{2}$.

Таким образом, условие нахождения точки $A_1$ на отрезке $A_2B_2$ равносильно $\theta \ge 0$.
То есть при $\theta < 0$, получаем $h = r$, а при $\theta \ge 0$ значение $h$ можно выразить из прямоугольного треугольника $AA_1A_2$.
$\blacksquare$\\

Используя лемму~\ref{lem:text_1_remesh_2d_accuracy_okrestnost}, для метода окрестностей можно выразить $S_i^o$ из \eqref{eqn:text_1_remesh2_saa2b2b_gen} подставив туда выражения для $h_i = h(\alpha, l, H - \frac{1}{2}\Delta H, \Delta H)$ и $h_{i + 1} = h(\alpha, l, H + \frac{1}{2}\Delta H, \Delta H)$ из \eqref{eqn:text_1_remesh_2d_okr_h}.

Теперь рассмотрим случай вогнутой сетки, представленный на рис.~\ref{fig:text_1_remesh_2d_theoretical_concave}.

\begin{figure}[ht]
\centering
\includegraphics[width=0.8\textwidth]{pics/text_1_remesh_2d/theoretical_concave.pdf}
\singlespacing
\captionstyle{center}\caption{Теоретическая оценка приближенных методов перестроения вогнутой поверхности в двумерном случае.}
\label{fig:text_1_remesh_2d_theoretical_concave}
\end{figure}

\begin{lemma}
Для ячейки $AB$ вогнутой сетки с равными углами $\angle (\overline{BA}, \overline{AA_2}) = \angle (\overline{AB}, \overline{BB_2}) = \alpha$ (см. рис.~\ref{fig:text_1_remesh_2d_theoretical_concave})
\begin{equation}\label{eqn:text_1_remesh2_saa2b2b_gen2}
S_i = \frac{1}{2} \sin \alpha \left( \lambda(h_i + h_{i+1}) - h_ih_{i+1} \right)
\end{equation}
где $\lambda = \frac{1}{2 \sin \frac{\alpha}{2}}$, $l = AB$, $h_i = AA_1$, $h_{i + 1} = BB_1$, при этом формула \eqref{eqn:text_1_remesh2_saa2b2b_gen2} применима при выполнении хотя бы одного из условий $h_i \le \lambda$, $h_{i+1} \le \lambda$.
\end{lemma}

Вычисления производятся аналогично лемме~\ref{lem:text_1_remesh2_vypukl_lemma}, с той лишь разницей, что $S_i = S_{AOB} - S_{A_1OB_1}$, откуда имеем выражение \eqref{eqn:text_1_remesh2_saa2b2b_gen2}.

Выполнение хотя бы одного из условий $h_i \le \lambda$, $h_{i+1} \le \lambda$ необходимо для предотвращения самопересечения сетки.
$\blacksquare$\\

Для перестроения методом прямоугольников и методом трапеций в случае вогнутой сетки получим формулы, аналогичные \eqref{eqn:text_1_remesh2_s_rect}, \eqref{eqn:text_1_remesh_2_Ht_vypukl} и \eqref{eqn:text_1_remesh2_trap}:
\begin{equation}\label{eqn:text_1_remesh2_s_rect_vogn}
	S_i^r = \cos \frac{\alpha}{2} \left( lH - \left( H^2 - \frac{1}{4} \Delta H^2 \right) \sin \frac{\alpha}{2} \right) \\
\end{equation}
\begin{equation}\label{eqn:text_1_remesh_2_Ht_vogn}
	\begin{aligned}
	& H_{i - 1}^t = H^t\left(l(H - \Delta H), l, \frac{\pi - \alpha}{2}\right) \\ 
	& H_i^t = H^t\left(lH, l, \frac{\pi - \alpha}{2}\right) \\
	& H_{i + 1}^t = H^t\left(l(H + \Delta H), l, \frac{\pi - \alpha}{2}\right)
	\end{aligned}
\end{equation}
\begin{equation}\label{eqn:text_1_remesh2_trap_vogn}
	S_i^t = \frac{1}{4} \left( (H_{i - 1}^t + 2 H_i^t + H_{i + 1}^t)l - (H_{i - 1}^t + H_i^t) (H_i^t + H_{i+1}^t) \tg \frac{\alpha}{2} \right)
\end{equation}

Заметим, что формулы \eqref{eqn:text_1_remesh2_s_rect_vogn} -- \eqref{eqn:text_1_remesh2_trap_vogn} идентичны формулам \eqref{eqn:text_1_remesh2_s_rect} -- \eqref{eqn:text_1_remesh2_trap} с учетом знака $\alpha$, поэтому формулы \eqref{eqn:text_1_remesh2_s_rect} -- \eqref{eqn:text_1_remesh2_trap} применимы как для положительных значений углов $\alpha$ (случай выпуклой сетки), так и для отрицательных значений $\alpha$ (случай вогнутой сетки).

Для получения теоретической оценки точности для метода окрестностей в случае вогнутой сетки рассмотрим следующее вспомогательное утверждение, аналогичное лемме~\ref{lem:text_1_remesh_2d_accuracy_okrestnost}.

\begin{figure}[ht]
\centering
\includegraphics[width=0.7\textwidth]{pics/text_1_remesh_2d/accuracy_okrestnost2.pdf}
\singlespacing
\captionstyle{center}\caption{Вычисление смещения узла в методе окрестностей для вогнутой сетки.}
\label{fig:text_1_remesh_2d_accuracy_okrestnost}
\end{figure}

\begin{lemma}\label{lem:text_1_remesh_2d_accuracy_okrestnost2}
Пусть $AB = l$, выше прямой $AB$ расположена точка $X$ такая, что $\angle (\overline{AX}, \overline{BA}) = \alpha$.
Пусть построены две окружности с центрами в точках $A$ и $B$ и с радиусами $r$ и $r + \Delta r$ соответственно.
Для построенных окружностей построена выпуклая оболочка, состоящая из дуг этих окружностей, а также отрезков общих касательных (точки касания общей касательной построенных окружностей, лежащие выше прямой $AB$, обозначим $A_2$ и $B_2$ соответственно).
Пусть на этой выпуклой оболочке -- на окружности с центром в точке $A$ либо на отрезке $A_2B_2$ -- лежит точка $A_1$ такая, что $\angle XAA_1 = \angle BAA_1 = \phi$, $AA_1 = h$, тогда 
\begin{equation}
	\begin{cases}\label{eqn:text_1_remesh_2d_okr_h2}
		h = r, \ \sin \frac{\alpha}{2} < \frac{-\Delta r}{l}, \\
		h(\alpha, l, r, \Delta r) = \frac{r}{\cos \left( \frac{\alpha}{2} - \arcsin \frac{-\Delta r}{l} \right)}, \ \sin \frac{\alpha}{2} \ge \frac{-\Delta r}{l}
	\end{cases}
\end{equation}
\end{lemma}

Если точка $A_1$ не находится на отрезке $A_2B_2$, то $h = AA_1 = r$.

Рассмотрим случай, когда $A_1$ находится на отрезке $A_2B_2$, в этом случае $\Delta r < 0$.
Через точку $B$ проведем прямую, параллельную $A_2B_2$, до пересечения с $AA_2$ в точке $H$.
Тогда $\angle A_2OA = \angle HBA = \beta = \arcsin \frac{-\Delta r}{l}$.
Так как $\angle A_1AB = \phi = \frac{\pi - \alpha}{2}$, то $\angle A_2AA_1 = \theta = \angle A_2AO - \phi = \frac{\alpha}{2} - \beta$.

Таким образом, условие нахождения точки $A_1$ на отрезке $A_2B_2$ равносильно $\theta \ge 0$.
То есть при $\theta < 0$, получаем $h = r$, а при $\theta \ge 0$ значение $h$ можно выразить из прямоугольного треугольника $AA_1A_2$.
$\blacksquare$\\


Используя \eqref{eqn:text_1_remesh2_s_rect} и \eqref{eqn:text_1_remesh2_trap} можно проанализировать зависимости величин $\delta_i^r$ и $\delta_i^t$ от $\alpha$ и $\frac{\Delta H}{l}$, при этом будем использовать фиксированную величину $H = \frac{l}{4}$ (см. рис.~\ref{fig:text_1_remesh_2d_main_chart}).

\begin{figure}[ht]
\centering
\includegraphics[width=1.0\textwidth]{pics/text_1_remesh_2d/main_chart.png}
\singlespacing
\captionstyle{center}\caption{Графики относительных отклонений заметаемых площадей $\breve{\delta}^r$ и $\breve{\delta}^t$ от целевых значений при перестроении методами прямоугольников и трапеций.}
\label{fig:text_1_remesh_2d_main_chart}
\end{figure}

На рис.~\ref{fig:text_1_remesh_2d_main_chart} слева приведены зависимости $\delta^r$ и $\delta^t$ от угла $\alpha$ при постоянном значении $H$, то есть при $\Delta H = 0$.
Метод трапеций в этом случае по определению абсолютно точен.
Для дальнейшего анализа было выбрано значение угла $\alpha > 0$, при котором метод прямоугольников наименее точен (это значение примерно равно 0,7) и при этом фиксированном значении $\alpha = 0,7$ были построены графики зависимостей $\delta^r$ и $\delta^t$ от $\Delta H$, изменяющегося в диапазоне $[-\frac{l}{2}, \frac{l}{2}]$ (см. рис.~\ref{fig:text_1_remesh_2d_main_chart} справа).
Из рисунка видно, что метод трапеций перестроения поверхности в двумерном случае более точен, чем метод прямоугольников.
