\documentclass[a4paper,14pt]{extarticle}                     % тип документа с размером шрифта 14pt

%---------------------------------------------------------------------------------------------------

%\usepackage{times}                                          % использование Times New Roman
                                                             %     (почему-то сносит все форматирование)
\usepackage[top=2cm,left=3cm,right=1cm,bottom=2cm]{geometry} % размеры полей
\usepackage[]{inputenc}                                      % эта строка нужна, чтобы документ открывался в редакторе MikTex
\usepackage[T2A]{fontenc}                                    % для поддержки русского языка
\usepackage[russian]{babel}                                  % включение русского языка
\usepackage{amsmath,amsthm,amscd,amsfonts,amssymb}           % специальные символы и т.п.
\usepackage{mathrsfs}                                        % специальные символы
\usepackage{indentfirst}                                     % отступ для начала абзаца
\usepackage{textcomp}                                        % текст в формулах
\usepackage{graphicx}                                        % подключение графики
\usepackage{listings}                                        % печать листингов
\usepackage{xcolor}                                          % использование цветов
\usepackage{caption2}                                        % для изменения стиля подписи рисунков
                                                             %     (приводит к warning-у, так что использовать только по необходимости)
\usepackage{verbatim}                                        % использование дополнительных возможностей verbatim           
\usepackage{fancybox}                                        % использование расширенного Verbatim
\usepackage[linesnumbered,boxed]{algorithm2e}                % оформление алгоритмов
\usepackage{booktabs}                                        % поддержка таблиц
\usepackage{makecell}                                        % для перевода строки внутри ячейки таблицы
\usepackage{ulem}                                            % волнистая черта снизу
\usepackage{textcomp}                                        % для коррекции положения тильды
\usepackage{longtable}                                       % многострочные таблицы
\usepackage{morefloats}                                      % подключить большее количество формул
\usepackage[section]{placeins}                               % сброс обработки флотов в конце страницы
\usepackage{float}                                           % расположение флотов прямо тут
\usepackage{setspace}                                        % чтобы менять междустрочный интервал с подписях

%---------------------------------------------------------------------------------------------------

\lstset{
aboveskip=15pt,
belowskip=15pt,
belowcaptionskip=10pt,
language={[ANSI]C++},
basewidth=0.5em,
xleftmargin=20pt,
xrightmargin=20pt,
basicstyle=\linespread{0.8}\small\ttfamily,           % 0.8 - уменьшение расстояния между строк
                                                             % linespread должен идти первым 
keywordstyle=\color[rgb]{0,0,1},
numbers=left,
numberstyle=\tiny,
stepnumber=1,
numbersep=10pt,
showspaces=false,
showstringspaces=false,
showtabs=false,
frame=trBL,
tabsize=2,
captionpos=t,
breaklines=false,
breakatwhitespace=false,
escapeinside={\%*}{*)}
}

%---------------------------------------------------------------------------------------------------

%\renewcommand{\GenericWarning}[2]{\GenericError{#1}{#2}{}{This warning has been turned into a fatal error.}} % Предупреждения -> ошибки.
\newcommand{\textapprox}{\raisebox{0.5ex}{\texttildelow}}    % положение тильды
\renewcommand{\baselinestretch}{1.5}                         % полуторный отступ между строк
\renewcommand{\captionlabeldelim}{.}                         % разделитель между номером рисунка и названием
\numberwithin{equation}{section}                             % нумерация формул по секциям
\numberwithin{figure}{section}                               % нумерация картинок по секциям
\numberwithin{table}{section}                                % нумерация таблиц по секциям
\theoremstyle{plain}                                         % стиль теорем
\newtheorem{theorem}{Теорема}[section]                       % теорема
\newtheorem{lemma}{Лемма}[section]                           % лемма
\newtheorem{definition}{Определение}[section]                % определение
\numberwithin{theorem}{section}                              % нумерация теорем по секциям
\numberwithin{lemma}{section}                                % нумерация лемм по секциям
\numberwithin{definition}{section}                           % нумерация определений по секциям

%---------------------------------------------------------------------------------------------------

\captionstyle{center}
\setlength{\abovecaptionskip}{0pt}
\setlength{\belowcaptionskip}{0pt}

%---------------------------------------------------------------------------------------------------

\begin{document}

\numberwithin{lstlisting}{section}                           % нумерация листингов по секциям
                                                             % определяем тут, так как счетчик листинга до begin{document}
                                                             % еще не существует
                                                             % https://tex.stackexchange.com/questions/441618/how-to-number-the-listings-within-sections

\title{Организация суперкомпьютерных вычислений на поверхностных неструктурированных расчетных сетках}
\author{}
\date{}
\maketitle
\thispagestyle{empty}                                        % не нумеруем первую страницу

\newpage
\renewcommand{\contentsname}{Оглавление}                     % переопределяем команду перед генерацией оглавления
\tableofcontents

%---------------------------------------------------------------------------------------------------

\newpage
\section*{Введение}                      % выключить номер введения
\addcontentsline{toc}{section}{Введение} % но добавить его в оглавление

В настоящее время высокопроизводительные вычисления являются неотъемлемой составляющей научных исследований, промышленных разработок и бизнеса.
Использование высокопроизводительных вычислительных систем \cite{GOST57700HPC} находит широкое применение во всех сферах деятельности человека.

% а) переход к передовым технологиям проектирования и создания высокотехнологичной продукции
Суперкомпьютерные вычисления являются основой для развития передовых технологий проектирования и создания высокотехнологичной продукции.
В частности суперкомпьютеные вычисления применяются в авиационной и космической промышленности \cite{Kornev2021SuperAvio}, автомобилестроении \cite{Wang2020SuperAuto}, проектировании морского и железнодорожного транспорта \cite{Nikitin2018SuperShip,Solovyev2013SuperTrains}, турбин \cite{Duben2022SuperTurbine} и других высокотехнологичных изделий.
Важную роль суперкомпьютеры играют при проектировании образцов вооружения, в частности военной техники и боеприпасов \cite{Ageeva2023SuperMilitary}.
Также суперкомпьютерные вычисления незаменимы при разработке новых материалов, для изучения свойств которых требуется проводить точного атомистическое моделивования структур, состоящих из миллиардов отдельных атомов \cite{Wang2025SuperMolDyn}

% б) переход к экологически чистой и ресурсосберегающей энергетике, повышение эффективности добычи и глубокой переработки углеводородного сырья, формирование новых источников энергии, способов ее передачи и хранения
В энергетической сфере высокопроизводительные вычисления применяются для моделирования объектов генерации электроэнергии, включая атомные станции \cite{Cancemi2025SuperNuc} (в совокупности с процессами внутри ядерных реакторов \cite{Zhang2025SuperNuclear}), ветряные и приливные генераторы \cite{Quint2025SuperWind,Parrado2024SuperTidal}.
Детальное моделирование месторождений углеводородного сырья позволяет повысить эффективность добычи \cite{Usmanov2024SuperPlast}, а создание цифровых моделей месторождений и систем транспортировки \cite{Didenko2023SuperOil} приводит к снижению потерь и обеспечивает прозрачность полного жизненного цикла, начиная с добычи сырья и заканчивая реализацией конечного топлива потребителю.

% в) переход к персонализированной, предиктивной и профилактической медицине, высокотехнологичному здравоохранению и технологиям здоровьесбережения, в том числе за счет рационального применения лекарственных препаратов
Использование больших вычислительных систем позволило извлекать качественно новые данные из точного моделирования крупных органических молекул и их ансамблей \cite{Teplukhin2009SuperBigMolec}.
Во время борьбы с пандемией COVID-19 суперкомпьютерные вычисления играли передовую роль в изучении вируса и разработке вакцины \cite{Colonnelli2021SuperCovid}.
Обработка с помощью искусственного интеллекта больших массивов медицинских данных, включая истории болезней, медицинские анализы и снимки \cite{Ri2024SuperXRay}, в совокупности с геномными исследованиями в настоящее время знаменует переход к персонализированной медицине \cite{Kishore2024SuperPrecMed}.

% г) переход к высокопродуктивному и экологически чистому агро- и аквахозяйству, разработку и внедрение систем рационального применения средств химической и биологической защиты сельскохозяйственных растений и животных, хранение и эффективную переработку сельскохозяйственной продукции
Компьютерное моделирование активно используется в растениеводстве и животноводстве для повышения эффективности выработки продовольствия.
Сюда входит целый спектр применений, начиная от селекции растений и животных \cite{Ahmetshina2020SuperSelection}, мониторинга их здоровья \cite{Mourant2018SuperEpi}, разработки удобрений и кормов \cite{Irfan2016SuperFert} и заканчивая планированием графиков разведения животных и выращивания сельскохозяйственных культур \cite{Zhang2021SuperFertPlan}.

% д) противодействие техногенным, биогенным, социокультурным угрозам, терроризму и экстремистской идеологии, деструктивному иностранному информационно-психологическому воздействию, а также киберугрозам и иным источникам опасности для общества, экономики и государства, укрепление обороноспособности и национальной безопасности страны в условиях роста гибридных угроз
В связи со стремительным распространением цифровизации во всех областях современной жизни и ростом объема цифровых данных особую важность обретает задача обеспечение кибербеопасности и сохранности данных.
Злонамеренные действия в киберсреде потенциально могут привести к серьезным последствиям, включая финансовые потери, техногенные и экологические катастрофы.
Высокопроизводительные вычисления используются для исследования и созданиия новых инструментов информационной безопасности, в том числе с помощью технологий искусственного интеллекта и систем распределенного реестра \cite{Terziyska2024SuperCyber}.

% е) повышение уровня связанности территории Российской Федерации путем создания интеллектуальных транспортных, энергетических и телекоммуникационных систем, а также занятия и удержания лидерских позиций в создании международных транспортно-логистических систем, освоении и использовании космического и воздушного пространства, Мирового океана, Арктики и Антарктики
Ввиду обширности территории России и неравномерности ее заселения и инфраструктурного обеспечения, необходимо создание интеллектуальных транспортных и телекоммуникационых систем для повышения уровня связности территории.
Суперкомпьютерные вычисления применяются для проектирования и развития систем железнодорожного и авиасообщения \cite{Juntana2022SuperFlight}, обеспечения логистики морских перевозок \cite{Yan2024SuperSea}, а также для создания глобальных компьютерных сетей \cite{Abramov2025SuperNets}.

% ж) возможность эффективного ответа российского общества на большие вызовы с учетом возрастающей актуальности синтетических научных дисциплин, созданных на стыке психологии, социологии, политологии, истории и научных исследований, связанных с этическими аспектами научно-технологического развития, изменениями социальных, политических и экономических отношений
%Развитие цифровизации во всех мире затрагивает не только технические стороны жизни человека, но и социально-психологические аспекты.
%В настоящее время можно констатировать, что социально-психологический профиль человека практически полностью определяется по его цифровому следу.
%Это с одной стороны открывает возможности по моделированию социальных и политических процессов \cite{} на основе цифровых данных.
%С другой стороны, доступность цифровых данных создает уязвимости как отдельного человека, так и целых групп населения, поэтому такие угрозы также нужно %анализировать и парировать \cite{}.

% з) объективную оценку выбросов и поглощения климатически активных веществ, снижение их негативного воздействия на окружающую среду и климат, повышение возможности качественной адаптации экосистем, населения и отраслей экономики к климатическим изменениям
Так как техногенное влияние человека на окружающую среду постоянно возрастает, то в настоящее время высокопроизводительные вычисления применяются для решения задач экологии.
В частности с помощью суперкомпьютерного моделирования и глобальных моделей проводятся климатические исследования \cite{Kulkarni2024SuperClimate}, исследования мирового океана \cite{Wei2024SuperOcean}, экосистем \cite{Rahman2024SuperSpecies} и оцениваются выбросы в окружающую среду и их последствия \cite{Ostromsky2024SuperAir}.

% и) переход к развитию природоподобных технологий, воспроизводящих системы и процессы живой природы в виде технических систем и технологических процессов, интегрированных в природную среду и естественный природный ресурсооборот
В последнее время особый акцент в технологическом развитии делается на природоподобные технологии, основанные на воспроизведении систем и процессов живой природы.
Как правило эти системы и процессы состоят из большого количества взаимодействующих элементов, которые требуют точного моделирования, что возможно только с использованием суперкомпьютеров.
К природоподобным направлениям, связанным с высокопроизводительными вычислениями, можно отнести разработку нейроморфных процессоров \cite{Rhodes2019SuperNuero}, технологии синтеза и воспроизведения тканей и органов человека \cite{Wang2012SuperTissues}, топологическую оптимизацию в проектировании изделий и строительстве \cite{Fedchikov2024SuperBim} и другие направления.

\paragraph{Актуальность темы.}

Приведенный выше, но не полный перечень сфер применения суперкомпьютерных вычислений отражает основные приоритеты научно-технологического развития Российской Федерации.
Развитие суперкомпьютерных технологий необходимо для обеспечения места России среди мировых технологических лидеров, поэтому вопросы создания и эффективного использования высокопроизводительных вычислительных систем являются крайне актуальными, особенно в условиях настоящего дефицита высокопроизводительных ресурсов в Российской Федерации \cite{Voevodin2021SuperRussia}.

% Суррогатные вычисления.
Широкое применение суперкомпьютерного моделирования в проектировании сложных технических систем связано с задачей выбора оптимальной конфигурации при большом количестве входных параметров.
По мере усложнения проектируемых систем и роста количества параметров возникла проблема дефицита вычислительных ресурсов, что привело к появлению концепции суррогатного моделирования \cite{Jiang2020Surrogate,Barcenas2023Surrogate,Catalani2024Surrogate}, то есть использования упрощенных суррогатных моделей, обученных на результатах суперкомпьютерного моделирования на ограниченных наборах данных.
Но даже использование суррогатного моделирования может лишь частично снизить потребность в вычислительных ресурсах, и не умаляет актуальность проблемы эффективного их использования.

% Задачи на поверхностях.
К числу задач суперкомпьютерного моделирования на поверхностных расчетных сетках относятся задачи внешнего обтекания \cite{Mitin2020Flow}, течения пленки жидкости по поверхности \cite{Li2014Film}, обледенения поверхности \cite{Koshelev2020Ice,Sorokin2020Ice}.
При этом задача обледенения поверхности является комплексной, так как для получения адекватной картины ледообразования необходимо учитывать множество сопряженных процессов, включая обтекание тела, выпадение на поверхность влаги и ледяных кристаллов из окружающего потока \cite{Cui2023Impingement}, взаимодействие выпадающего вещества с поверхностью \cite{Cui2021Impingement}, течение жидкости по поверхности в виде тонкой пленки или отдельных нитей \cite{Alexeenko2013Ice}, теплопроводность на поверхности \cite{Domingos2015IceHeat}, а также через слой жидкости и льда \cite{Xin2013Ice} и многие другие процессы.
Также в процессе образования слоя льда меняется сама геометрия рассматриваемой поверхности, так как форма образовавшихся ледяных наростов существенным образом влияет на все связанные процессы, что приводит к необходимости перестроения расчетных сеток.

\paragraph{Цели диссертационной работы.}

\paragraph{Методы исследования.}

\paragraph{Научная новизна и практическая значимость.}

Описанные в работе методы и алгоритмы апробированы на суперкомпьютерах Межведомственного суперкомпьютерного центра Российской академии наук и Национального центра <<Курчатовский институт>>, реализованы в рамках инструментов суперкомпьютерного моделирования, в результате чего зарегистрировано семь программ для ЭВМ \cite{CertRybakov2019AVX,CertRybakov2020PrepStruct,CertGoryachev2020Crys,CertRybakov2021PrepUnstruct,CertRybakov2023Mon,CertGoryachev2023Crys,CertRybakov2024Surf}.

\paragraph{Основные научные и практические результаты, выносимые на защиту.}

\begin{enumerate}
\item Метод перестроения неструктурированной поверхностной расчетной сетки с использованием окрестностей ячеек для задачи ледообразования.
\item Метод устранения самопересечений неструктурированной поверхностной расчетной сетки.
\item Алгоритм распределения вычислительной нагрузки при расчетах на блочно-структурированной сетке с дроблением блоков.
\item Алгоритм сглаживания границ доменов при декомпозиции неструктурированной поверхностной расчетной сетки.
\end{enumerate}

\paragraph{Апробация работы.}

Материалы диссертации докладывались на следующих конференциях и семинарах:

\begin{itemize}
\item V Национальный Суперкомпьютерный Форум (НСКФ-2016), Россия, Переславль-Залесский, Институт программных систем им. А.~К.~Айламазяна, 29 ноября -- 2 декабря 2016 г.
\item VII Национальный Суперкомпьютерный Форум (НСКФ-2018), Россия, Переславль-Залесский, Институт программных систем им. А.~К.~Айламазяна, 27-30 ноября 2018 г.
\item X Национальный Суперкомпьютерный Форум (НСКФ-2021), Россия, Переславль-Залесский, Институт программных систем им. А.~К.~Айламазяна, 30 ноября -- 3 декабря 2021 г.
\item I Международная научная конференция <<Конвергентные когнитивно-информационные технологии>>, Россия, Москва, Московский государственный университет им. М.~В.~Ломоносова, 25-26 ноября 2016 г.
\item III Международная научная конференция <<Конвергентные когнитивно-информационные технологии>>, Россия, Москва, Московский государственный университет им. М.~В.~Ломоносова, 22-25 ноября 2018 г.
\item XVI Международная конференция <<Супервычисления и математическое моделирование>>, Россия, Саров, Российский федеральный ядерный центр -- Всероссийский научно-исследовательский институт экспериментальной физики, 3-7 октября 2016 г.
\item Шестая всероссийская конференция <<Вычислительный эксперимент в аэроакустике>>, Россия, Светлогорск, организатор -- Институт прикладной математики им. М.~В.~Келдыша РАН, 19-24 сентября 2016 г.
\item Слет разработчиков отечественных CFD кодов (CFD Weekend 2023), Россия, Москва, Институт прикладной математики им. М.~В.~Келдыша, 9-10 декабря 2023 г.
\item Слет разработчиков отечественных CFD кодов (CFD Weekend 2022), Россия, Москва, Институт прикладной математики им. М.~В.~Келдыша, 3-4 декабря 2022 г.
\item The Intel Xeon Phi users group Russia annual meeting (IXPUG Russia-2017), // Joint Supercomputer Center of the Russian Academy of Sciences (JSCC RAS), Russia, Moscow, 1-2 June 2017.
\end{itemize}

\paragraph{Содержание.}

%---------------------------------------------------------------------------------------------------

\newpage
\section*{Глава 1. Геометрические методы организации \\ работы с неструктурированной поверхностной \\ расчетной сеткой} % выключить номер первой главы
\addcontentsline{toc}{section}{Глава 1. Геометрические методы организации работы с неструктурированной поверхностной расчетной сеткой}                                                                                                           % но добавить ее в оглавление
\addtocounter{section}{1}                                                                                         % а теперь и счетчик продвинуть
\setcounter{subsection}{0}
\setcounter{figure}{0}
\setcounter{equation}{0}
\setcounter{table}{0}
\setcounter{theorem}{0}
\setcounter{lemma}{0}
\setcounter{definition}{0}

\input text_1_geo_prim.tex
\input text_1_remesh_2d.tex
\input text_1_remesh_3d.tex
\input text_1_remesh_common_envelope.tex
\input text_1_int.tex
\input text_1_gas_dyn.tex
\input text_1_immersed_boundary_method.tex

\subsection{Выводы из главы}

TODO

%---------------------------------------------------------------------------------------------------

\newpage
\section*{Глава 2. Методы распаралеливания вычислений с передачей сообщений}                      % выключить номер первой главы
\addcontentsline{toc}{section}{Глава 2. Методы распаралеливания вычислений с передачей сообщений} % но добавить ее в оглавление
\addtocounter{section}{1}                                                                                                % а теперь и счетчик продвинуть
\setcounter{subsection}{0}
\setcounter{figure}{0}
\setcounter{equation}{0}
\setcounter{table}{0}
\setcounter{theorem}{0}
\setcounter{lemma}{0}
\setcounter{definition}{0}
\setcounter{lstlisting}{0}

\input text_2_1_qual.tex
\input text_2_2_block.tex
\input text_2_3_getero.tex
\input text_2_4_withcut.tex
\input text_2_5_decompsurf.tex
\input text_2_6_smooth.tex
\input text_2_7_genetic.tex
\input text_2_8_scaling.tex

\subsection{Выводы из главы}

TODO

%---------------------------------------------------------------------------------------------------

\newpage
\section*{Глава 3. Методы распараллеливания \\ вычислений на общей памяти}                      % выключить номер первой главы
\addcontentsline{toc}{section}{Глава 3. Методы распараллеливания вычислений на общей памяти} % но добавить ее в оглавление
\addtocounter{section}{1}                                                                    % а теперь и счетчик продвинуть
\setcounter{subsection}{0}
\setcounter{figure}{0}
\setcounter{equation}{0}
\setcounter{table}{0}
\setcounter{theorem}{0}
\setcounter{lemma}{0}
\setcounter{definition}{0}
\setcounter{lstlisting}{0}

\input text_3_graph_prim.tex
\input text_3_edge_coloring.tex
\input text_3_omp1.tex
%\input text_3_omp2.tex

\subsection{Выводы из главы}

TODO

%---------------------------------------------------------------------------------------------------

\newpage
\section*{Глава 4. Методы векторизации программного \\ кода для повышения его производительности}                      % выключить номер первой главы
\addcontentsline{toc}{section}{Глава 4. Методы векторизации программного кода для повышения его производительности} % но добавить ее в оглавление
\addtocounter{section}{1}                                                                                           % а теперь и счетчик продвинуть
\setcounter{subsection}{0}
\setcounter{figure}{0}
\setcounter{equation}{0}
\setcounter{table}{0}
\setcounter{theorem}{0}
\setcounter{lemma}{0}
\setcounter{definition}{0}
\setcounter{lstlisting}{0}

\input text_4_01_vec_description.tex
\input text_4_02_small_matr.tex
\input text_4_03_spec_matr.tex
\input text_4_04_flat.tex
\input text_4_05_ibm.tex
\input text_4_06_vec_loc_branch.tex
\input text_4_07_vec_mrg_under_cond.tex
\input text_4_08_vec_check_mask.tex
\input text_4_09_vec_comb_mask.tex

% Опционально можно добавить, тут страницы на 2-3.
%\input text_4_tozh.tex

\input text_4_10_mesh_intersect.tex
\input text_4_11_vec_riemann.tex
\input text_4_12_vec_irreg.tex
\input text_4_13_vec_integer.tex

\subsection{Выводы из главы}

TODO

%---------------------------------------------------------------------------------------------------

\newpage
\section*{Заключение}                                        % выключить номер заключения
\addcontentsline{toc}{section}{Заключение}                   % но добавить его в оглавление

%---------------------------------------------------------------------------------------------------

\input text_abbr.tex         % список сокращений
\input text_term.tex
\input text_bibliography.tex % список используемой литературы

%---------------------------------------------------------------------------------------------------

\end{document}
