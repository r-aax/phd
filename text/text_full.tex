\documentclass[a4paper,14pt]{extarticle}                     % тип документа с размером шрифта 14pt

%---------------------------------------------------------------------------------------------------

%\usepackage{times}                                          % использование Times New Roman
                                                             %     (почему-то сносит все форматирование)
\usepackage[top=2cm,left=3cm,right=1cm,bottom=2cm]{geometry} % размеры полей
\usepackage[]{inputenc}                                      % эта строка нужна, чтобы документ открывался в редакторе MikTex
\usepackage[T2A]{fontenc}                                    % для поддержки русского языка
\usepackage[russian]{babel}                                  % включение русского языка
\usepackage{amsmath,amsthm,amscd,amsfonts,amssymb}           % специальные символы и т.п.
\usepackage{indentfirst}                                     % отступ для начала абзаца
\usepackage{textcomp}                                        % текст в формулах
\usepackage{graphicx}                                        % подключение графики
\usepackage{caption2}                                        % для изменения стиля подписи рисунков
                                                             %     (приводит к warning-у, так что использовать только по необходимости)
\usepackage{verbatim}                                        % использование дополнительных возможностей verbatim           
\usepackage{fancybox}                                        % использование расширенного Verbatim

%---------------------------------------------------------------------------------------------------

\renewcommand{\baselinestretch}{1.5}                         % полуторный отступ между строк
\renewcommand{\captionlabeldelim}{.}                         % разделитель между номером рисунка и названием
\numberwithin{equation}{section}                             % нумерация формул по секциям
\numberwithin{figure}{section}                               % нумерация картинок по секциям
\numberwithin{table}{section}                                % нумерация таблиц по секциям
\theoremstyle{plain}                                         % стиль теорем
\newtheorem{theorem}{Теорема}[section]                       % теорема
\newtheorem{lemma}{Лемма}[section]                           % лемма
\newtheorem{definition}{Определение}[section]                % определение
\numberwithin{theorem}{section}                              % нумерация теорем по секциям
\numberwithin{lemma}{section}                                % нумерация лемм по секциям
\numberwithin{definition}{section}                           % нумерация определений по секциям

%---------------------------------------------------------------------------------------------------

\begin{document}

\title{Организация суперкомпьютерных вычислений на поверхностных неструктурированных расчетных сетках}
\author{}
\date{}
\maketitle
\thispagestyle{empty}                                        % не нумеруем первую страницу

\newpage
\renewcommand{\contentsname}{Оглавление}                     % переопределяем команду перед генерацией оглавления
\tableofcontents

%---------------------------------------------------------------------------------------------------

\newpage
\section*{Введение}                      % выключить номер введения
\addcontentsline{toc}{section}{Введение} % но добавить его в оглавление


\paragraph{Актуальность темы.}

\paragraph{Цели диссертационной работы.}

\paragraph{Методы исследования.}

\paragraph{Научная новизна и практическая значимость.}

\paragraph{Основные научные и практические результаты, выносимые на защиту.}

\paragraph{Апробация работы.}

\paragraph{Содержание.}

%---------------------------------------------------------------------------------------------------

\newpage
\section*{Глава 1. TODO}                      % выключить номер первой главы
\addcontentsline{toc}{section}{Глава 1. TODO} % но добавить ее в оглавление
\addtocounter{section}{1}                     % а теперь и счетчик продвинуть
\setcounter{subsection}{0}
\setcounter{figure}{0}
\setcounter{equation}{0}
\setcounter{theorem}{0}
\setcounter{lemma}{0}
\setcounter{definition}{0}

\input text_1_remesh_common_envelope.tex
\input text_1_immersed_boundary_method.tex

%---------------------------------------------------------------------------------------------------

\newpage
\section*{Глава 2. TODO}                      % выключить номер первой главы
\addcontentsline{toc}{section}{Глава 2. TODO} % но добавить ее в оглавление
\addtocounter{section}{2}                     % а теперь и счетчик продвинуть
\setcounter{subsection}{0}
\setcounter{figure}{0}
\setcounter{equation}{0}
\setcounter{theorem}{0}
\setcounter{lemma}{0}
\setcounter{definition}{0}

%---------------------------------------------------------------------------------------------------

\newpage
\section*{Глава 3. TODO}                      % выключить номер первой главы
\addcontentsline{toc}{section}{Глава 3. TODO} % но добавить ее в оглавление
\addtocounter{section}{3}                     % а теперь и счетчик продвинуть
\setcounter{subsection}{0}
\setcounter{figure}{0}
\setcounter{equation}{0}
\setcounter{theorem}{0}
\setcounter{lemma}{0}
\setcounter{definition}{0}

%---------------------------------------------------------------------------------------------------

\newpage
\section*{Глава 4. TODO}                      % выключить номер первой главы
\addcontentsline{toc}{section}{Глава 4. TODO} % но добавить ее в оглавление
\addtocounter{section}{4}                     % а теперь и счетчик продвинуть
\setcounter{subsection}{0}
\setcounter{figure}{0}
\setcounter{equation}{0}
\setcounter{theorem}{0}
\setcounter{lemma}{0}
\setcounter{definition}{0}

%---------------------------------------------------------------------------------------------------

\newpage
\section*{Заключение}                                        % выключить номер заключения
\addcontentsline{toc}{section}{Заключение}                   % но добавить его в оглавление

%---------------------------------------------------------------------------------------------------

\input text_abbr.tex         % список сокращений
\input text_term.tex         % предметный указатель
\input text_bibliography.tex % список используемой литературы

%---------------------------------------------------------------------------------------------------

\end{document}
