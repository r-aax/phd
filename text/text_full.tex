\documentclass[a4paper,14pt]{extarticle}                     % тип документа с размером шрифта 14pt

%---------------------------------------------------------------------------------------------------

%\usepackage{times}                                          % использование Times New Roman
                                                             %     (почему-то сносит все форматирование)
\usepackage[top=2cm,left=3cm,right=1cm,bottom=2cm]{geometry} % размеры полей
\usepackage[]{inputenc}                                      % эта строка нужна, чтобы документ открывался в редакторе MikTex
\usepackage[T2A]{fontenc}                                    % для поддержки русского языка
\usepackage[russian]{babel}                                  % включение русского языка
\usepackage{amsmath,amsthm,amscd,amsfonts,amssymb}           % специальные символы и т.п.
\usepackage{mathrsfs}                                        % специальные символы
\usepackage{indentfirst}                                     % отступ для начала абзаца
\usepackage{textcomp}                                        % текст в формулах
\usepackage{graphicx}                                        % подключение графики
\usepackage{listings}                                        % печать листингов
\usepackage{xcolor}                                          % использование цветов
\usepackage{caption2}                                        % для изменения стиля подписи рисунков
                                                             %     (приводит к warning-у, так что использовать только по необходимости)
\usepackage{verbatim}                                        % использование дополнительных возможностей verbatim           
\usepackage{fancybox}                                        % использование расширенного Verbatim
\usepackage[linesnumbered,boxed]{algorithm2e}                % оформление алгоритмов
\usepackage{booktabs}                                        % поддержка таблиц
\usepackage{makecell}                                        % для перевода строки внутри ячейки таблицы
\usepackage{ulem}                                            % волнистая черта снизу
\usepackage{textcomp}                                        % для коррекции положения тильды
\usepackage{longtable}                                       % многострочные таблицы
\usepackage{morefloats}                                      % подключить большее количество формул
\usepackage[section]{placeins}                               % сброс обработки флотов в конце страницы
\usepackage{float}                                           % расположение флотов прямо тут
\usepackage{setspace}                                        % чтобы менять междустрочный интервал с подписях
\usepackage{cite}                                            % для использования диапазона цитирования

%---------------------------------------------------------------------------------------------------

\lstset{
aboveskip=15pt,
belowskip=15pt,
belowcaptionskip=10pt,
language={[ANSI]C++},
basewidth=0.5em,
xleftmargin=20pt,
xrightmargin=20pt,
basicstyle=\linespread{0.8}\small\ttfamily,                  % 0.8 - уменьшение расстояния между строк
                                                             % linespread должен идти первым 
keywordstyle=\color[rgb]{0,0,1},
numbers=left,
numberstyle=\tiny,
stepnumber=1,
numbersep=10pt,
showspaces=false,
showstringspaces=false,
showtabs=false,
frame=trBL,
tabsize=2,
captionpos=t,
breaklines=false,
breakatwhitespace=false,
escapeinside={\%*}{*)}
}

%---------------------------------------------------------------------------------------------------

%\renewcommand{\GenericWarning}[2]{\GenericError{#1}{#2}{}{This warning has been turned into a fatal error.}} % Предупреждения -> ошибки.
\newcommand{\textapprox}{\raisebox{0.5ex}{\texttildelow}}    % положение тильды
\renewcommand{\baselinestretch}{1.5}                         % полуторный отступ между строк
\renewcommand{\captionlabeldelim}{.}                         % разделитель между номером рисунка и названием
\numberwithin{equation}{section}                             % нумерация формул по секциям
\numberwithin{figure}{section}                               % нумерация картинок по секциям
\numberwithin{table}{section}                                % нумерация таблиц по секциям
\theoremstyle{plain}                                         % стиль теорем
\newtheorem{theorem}{Теорема}[section]                       % теорема
\newtheorem{lemma}{Лемма}[section]                           % лемма
\newtheorem{definition}{Определение}[section]                % определение
\numberwithin{theorem}{section}                              % нумерация теорем по секциям
\numberwithin{lemma}{section}                                % нумерация лемм по секциям
\numberwithin{definition}{section}                           % нумерация определений по секциям

%---------------------------------------------------------------------------------------------------

\captionstyle{center}
\setlength{\abovecaptionskip}{0pt}
\setlength{\belowcaptionskip}{0pt}

%---------------------------------------------------------------------------------------------------

\begin{document}

\numberwithin{lstlisting}{section}                           % нумерация листингов по секциям
                                                             % определяем тут, так как счетчик листинга до begin{document}
                                                             % еще не существует
                                                             % https://tex.stackexchange.com/questions/441618/how-to-number-the-listings-within-sections

\title{Повышение эффективности высокопроизводительных вычислений на поверхностных расчетных сетках с изменяемой геометрией}
\author{Рыбаков~А.~А.}
\date{27.08.2025}
\maketitle
\thispagestyle{empty}                                        % не нумеруем первую страницу

\newpage
\renewcommand{\contentsname}{Оглавление}                     % переопределяем команду перед генерацией оглавления
\tableofcontents

%---------------------------------------------------------------------------------------------------

%\input text_intro.tex                                        % введение
%\input text_2dr.tex                                          % перестроение в двумерном случае
%\input text_3dr.tex                                          % перестроение в трехмерной случае
%\input text_int.tex                                          % пересечение с сетками
%\input text_par.tex                                          % распараллеливание
\input text_vec.tex	                                         % векторизация
%\input text_conclusion.tex                                   % заключение

%---------------------------------------------------------------------------------------------------

%\newpage
%\section*{Глава 3. Методы распараллеливания \\ вычислений на общей памяти}                      % выключить номер первой главы
%\addcontentsline{toc}{section}{Глава 3. Методы распараллеливания вычислений на общей памяти} % но добавить ее в оглавление
%\addtocounter{section}{1}                                                                    % а теперь и счетчик продвинуть
%\setcounter{subsection}{0}
%\setcounter{figure}{0}
%\setcounter{equation}{0}
%\setcounter{table}{0}
%\setcounter{theorem}{0}
%\setcounter{lemma}{0}
%\setcounter{definition}{0}
%\setcounter{lstlisting}{0}

%\input text_3_graph_prim.tex
%\input text_3_edge_coloring.tex
%\input text_3_omp1.tex
%\input text_3_omp2.tex

%\subsection{Выводы из главы}

%TODO

%---------------------------------------------------------------------------------------------------





%---------------------------------------------------------------------------------------------------

%\newpage
%\section*{Заключение}                                        % выключить номер заключения
%\addcontentsline{toc}{section}{Заключение}                   % но добавить его в оглавление

%---------------------------------------------------------------------------------------------------

\input text_abbr.tex         % список сокращений
\input text_term.tex
\input text_bibliography.tex % список используемой литературы

%---------------------------------------------------------------------------------------------------

\end{document}
