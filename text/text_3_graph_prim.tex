В предыдущей главе были рассмотрены вопросы, касающиеся распараллеливания вычислений между отдельными узлами суперкомпьютерного кластера.
Современные микропроцессоры, входящие в состав вычислительных узлов, содержат десятки вычислительных ядер \cite{Section3IntroIntel,Section3IntroAMD,Kuzminsky2022ARM}, что с учетом многопоточности приводит необходимости организации вычислений в рамках одного узла с использованием сотен потоков, работающих на общей памяти.

Так же как и в случае распределения вычисления между узлами суперкомпьютера распараллеливание вычислений на общей памяти преследует цель сокращения времени выполнения приложений.
Исходя из этого основным показателем качества организации высокопроизводительных вычислений на общей памяти является ускорение выполнения задачи при увеличении количества задействованных потоков.

\begin{definition}
Ускорением выполнения задачи в модели распараллеливания на общей памяти\label{term:shr_speedup} при использовании $k$ потоков будем называть величину $s_{shr}(k) = \frac{T(1)}{T(k)}$, где $T(i)$ -- время выполнения задачи с использованием $i$ потоков.
\end{definition}

\begin{definition}
Эффективностью распараллеливания при выполнении задачи на общей памяти\label{term:shr_eff} при использовании $k$ потоков будем называть величину $e_{shr}(k) = \frac{s_{shr}(k)}{k}$.
\end{definition}

Параметры $s_{shr}$ и $e_{shr}$ аналогичны параметрам ускорения\label{term:msg_speedup3} и эффективности\label{term:msh_eff3} распараллеливания вычислений в модели с передачей сообщений.
В случае идеального распараллеливания вычислений $s_{shr}(k) = k$, $e_{shr}(k) = 1$.

Основной проблемой снижения эффективности распараллеливания вычислений на общей памяти являются конфликты при доступе разных потоков к одной и той же области данных.
В этой главе рассмотрены вопросы организации вычислений на поверхностной неструктурированной расчетной сетке\label{term:unstruct_surf_calc_mesh6} на общей памяти с целью устранения конфликтов по данным между разными потоками.

Во время выполнения расчетов на поверхностной неструктурированной сетке при обработке ячейки необходимо обращаться за данными к ее вычислительной окрестности\label{term:cell_calc_template3}, которая в простейшем случае представляет собой все соседние по ребрам ячейки.
Аналогично, все ячейки из окрестности обращаются к рассматриваемой ячейке, и при одновременном обращении к каким-либо данным на запись возникает конфликт между потоками, обрабатывающими разные ячейки окрестности.
Добиться устранения таких конфликтов можно путем запрета параллельной обработке ячеек, принадлежащих к вычислительно окрестности одной и той же ячейки.
Эта задача напрямую связана с задачей реберной раскраски дуального графа расчетной сетки, в результате которой множество ребер разбиваются на подмножества, не содержащие конфликтующих ребер.

% Положения теории графов, используемые в разделе.
\subsection{Положения теории графов, используемые в главе}\label{sec:text_3_graph_prim}

Рассмотрим кубический плоский граф\label{term:graph_cubic} и вопросы его реберных раскрасок.
При этом, наряду с доказательствами существования самих раскрасок будем рассматривать алгоритмы построения этих раскрасок.
Сначала рассмотрим наиболее тривиальное утверждение.

%------------------------------------------------------------------------------------------------------

\begin{lemma}\label{lem:text_3_graph_prim_coloring5}
Кубический граф допускает реберную раскраску в $5$ цветов, и эта раскраска может быть построена с линейной сложностью по количеству ребер.
\end{lemma}
Для построения требуемой раскраски достаточно последовательно перебрать все ребра.
Так как для каждого ребра существует ровно $4$ смежные ребра, то из пяти цветов всегда можно выбрать цвет, в который не покрашено ни одно из этих смежных ребер.
Таким образом, раскраска строится за один проход по всем ребрам, то есть с линейной сложностью.
$\blacksquare$\\

%------------------------------------------------------------------------------------------------------

В лемме~\ref{lem:text_3_graph_prim_coloring5} не использовалась планарность графа, а количество цветов явно избыточно.
Согласно теореме Визинга \cite{Vizing1964}, \cite{Vizing1965} кубический граф допускает реберную раскраску в $4$ цвета.
Однако, теорема Визинга также не использует планарность графа, и алгоритм, построенный исходя из доказательства, будет иметь сложность, выше линейной по количеству ребер \cite{Soifer2009}.
Приведем альтернативное доказательство с использованием планарности графа и более низкой сложностью алгоритма построения реберной раскраски в $4$ цвета.

\begin{lemma}\label{lem:text_3_graph_prim_coloring4}
Кубический граф допускает реберную раскраску в $4$ цвета, и эта раскраска может быть построена с линейной сложностью по количеству ребер.
\end{lemma}

Доказательство существования раскраски будем проводить по индукции по количеству вершин графа.
База индукции: кубический граф с минимальным количеством вершин это $K_4$.
Для этого графа утвержление очевидно.

Предположим, что утверждение теоремы верно для всех кубических графов с порядок которых изменяется от $4$ до $n - 2$ (отметим, что порядок кубического графа не может быть нечетным).

Рассмотрим кубический планарный граф, с множеством вершин $V$ ($n = |V| = \nu$), множеством ребер $E$ ($|E| = \varepsilon$) и множеством граней $F$ ($|F| = \zeta$).
При этом внешнюю область вокруг графа также считаем гранью (рассматриваем граф, расположенный на сфере).
Также через $\zeta_i$ обозначим количество граней, содержашей ровно $i \ge 3$ вершин (в этом случае будем говорить, что грань имеет размер $i$).
\begin{equation}
	\zeta = \sum_{i = 3}^{\infty}{\zeta_i}.
\end{equation}
Рассмотрим соотношения, выполняемые для этого графа.

Так как степень каждой вершины равна $3$, то лемма о рукопожатиях для этого графа приобретает вид
\begin{equation}
	3 \nu = 2 \varepsilon.
\end{equation}

Так как каждое ребро входит ровно в две грани, то верно соотношение
\begin{equation}
	2 \varepsilon = \sum_{i = 3}^{\infty}{i \zeta_i}.
\end{equation}

Наконец, для рассматриваемого графа выполняется соотношение Эйлера $\nu + \zeta - \varepsilon = 2$.
Подставив в соотношение Эйлера выражения для $\nu$, $\zeta$ и $\varepsilon$ получим
\begin{equation}
	2 \cdot 2\varepsilon + 6 \sum_{i = 3}^{\infty}{\zeta_i} - 3 \cdot 2\varepsilon = 12
\end{equation}
\begin{equation}
	6 \sum_{i = 3}^{\infty}{\zeta_i} - \sum_{i = 3}^{\infty}{i \zeta_i} = 12
\end{equation}
\begin{equation}\label{eqn:text_3_graph_prim_euler}
	\sum_{i = 3}^{\infty}{(6 - i) \zeta_i} = 12
\end{equation}

Из \eqref{eqn:text_3_graph_prim_euler} видно, что в рассматриваемом графе найдется грань размера $3$, $4$ или $5$.
Рассмотрим каждый этих случаев отдельно.

Случай 1. Сначала рассмотрим случай наличия в графе гамака, состоящего из двух треугольных граней, как показано на рис.~\ref{fig:text_3_graph_prim_coloring4_gamak} слева.

\begin{figure}[ht]
\centering
\includegraphics[width=1.0\textwidth]{./pics/text_3_graph_prim/coloring4_gamak.pdf}
\singlespacing
\captionstyle{center}\caption{Случай наличия в графе гамака из двух треугольных граней.}
\label{fig:text_3_graph_prim_coloring4_gamak}
\end{figure}

Стянув гамак в одно ребро (рис.~\ref{fig:text_3_graph_prim_coloring4_gamak} в центре), получим граф с $n - 4$ вершинами, ребра которого могут быть раскрашены в 4 цвета по предположению индукции.
Тогда по этой раскраске не представляется сложным раскрасить исходный граф, как показано на рис.~\ref{fig:text_3_graph_prim_coloring4_gamak} справа.

Случай 2. В графе нет гамаков, состоящих из двух треугольных граней, но найдется треугольная грань (см. рис.~\ref{fig:text_3_graph_prim_coloring4_face3} слева).

\begin{figure}[ht]
\centering
\includegraphics[width=1.0\textwidth]{./pics/text_3_graph_prim/coloring4_face3.pdf}
\singlespacing
\captionstyle{center}\caption{Случай наличия в графе треугольной грани.}
\label{fig:text_3_graph_prim_coloring4_face3}
\end{figure}

Стянем треугольную грань $ABC$ в точку $O$, как показано на рис.~\ref{fig:text_3_graph_prim_coloring4_face3} в центре.
Так как в графе не было гамаков, состоящих из двух треугольных граней, то у грани $ABC$ не было соседних треугольных граней по ребру, а значит при стягивании $ABC \rightarrow O$ ни одна грань кроме $ABC$ не выродилась.
Это значит, что мы получили плоский кубический граф с $n - 2$ вершинами, ребра которого могут быть раскрашены в 4 цвета по предположению индукции.
Тогда по этой раскраске можно построить раскраску исходного графа, как показано на рис.~\ref{fig:text_3_graph_prim_coloring4_face3} справа.

Случай 3. В графе нет треугольных граней, но найдется четырехугольная грань (см. рис.~\ref{fig:text_3_graph_prim_coloring4_face4} слева).

\begin{figure}[ht]
\centering
\includegraphics[width=1.0\textwidth]{./pics/text_3_graph_prim/coloring4_face4.pdf}
\singlespacing
\captionstyle{center}\caption{Случай наличия в графе четырехугольной грани.}
\label{fig:text_3_graph_prim_coloring4_face4}
\end{figure}

Рассмотрим четырехугольную грань $ABCD$.
Выполним стягивание ребер $AD \rightarrow P$, $BC \rightarrow Q$.
Так как в графе не было треугольных граней, то при стягивании выродилась только грань $ABCD$.
Мы получили плоский кубический граф с $n - 2$ вершинами, ребра которого могут быть раскрашены в 4 цвета по предположению индукции.
По этой раскраске построим раскраску исходного графа.
Для этого перенесем цвета всех ребер (кроме $PQ$) на исходный граф, непокрашенными останутся только ребра цикла $A-B-C-D$.
Пусть $\gamma(PQ) = a$.
Если для покраски остальных 4 ребер, инцидентных вершинам $P$ и $Q$, были использованы только 2 цвета ($b$ и $c$), то исходный граф можно покрасить в 4 цвета, покрасив ребра цикла $A-B-C-D$ в цвета $a$ и $d$ (см. рис.~\ref{fig:text_3_graph_prim_coloring4_face4} сверху).
Если для покраски остальных 4 ребер, инцидентных вершинам $P$ и $Q$, были использованы все оставшиеся три цвета ($b$, $c$ и $d$), то одним из этих цветов было покрашено 2 ребра (пусть это будет цвет $b$), а двумя другими цветами -- по одному ребру (см. рис.~\ref{fig:text_3_graph_prim_coloring4_face4} снизу).
В этом случае раскрасим ребра цикла $A-B-C-D$ в цвета $a$, $c$, $d$ следующим образом.
К циклу $A-B-C-D$ примыкает одно ребро цвета $c$, оно смежно двум ребрам этого цикла (которые являются смежными между собой), которые не могут быть покрашены с цвет $c$.
То есть в циикле $A-B-C-D$ найдется пара смежных ребер, одно из которых может быть покрашено в цвет $c$ (обозначим эту пару $p_c$).
Аналогично, в цикле $A-B-C-D$ найдется пара смежных ребер, одно из которых может быть покрашено в цвет $d$ (обозначим эту пару $p_d$).
Вне зависимости от того пересекаются ли $p_c$ и $p_d$, в цикле $A-B-C-D$ можно найти такие несмежные ребра $e_c$ и $e_d$, что $e_c \in p_c$, $e_d \in p_d$.
Тогда покрасим ребро $e_c$ в цвет $c$, ребро $e_d$ -- в цвет $d$, а остальные два несмежных ребра -- в цвет $a$ (см. рис.~\ref{fig:text_3_graph_prim_coloring4_face4} снизу справа).

Случай 4. В графе нет треугольных и четырехугольных граней, но найдется пятиугольная грань.

\begin{figure}[ht]
\centering
\includegraphics[width=1.0\textwidth]{./pics/text_3_graph_prim/coloring4_face5.pdf}
\singlespacing
\captionstyle{center}\caption{Случай наличия в графе пятиугольной грани.}
\label{fig:text_3_graph_prim_coloring4_face5}
\end{figure}

Рассмотрим пятиугольную грань $ABCDE$.
Удалим ребра $ED$ и $BC$.
Стянем вершины $E$, $A$, $B$ в одну вершину $O$.
Стянем вершины $D$, $C$ в одну вершину степени 2, которую сразу удалим, соединим инцидентные ей ребра в одно $D'C'$ (см. рис.~\ref{fig:text_3_graph_prim_coloring4_face5}, а).
Так как в графе отсутствовали треугольные грани, то ни одна грань кроме $ABCDE$ не выродилась.
Таким образом, мы получили граф с $n - 4$ вершинами, ребра которого могут быть раскрашены в 4 цвета.
По этой раскраске построим раскраску исходного графа.
Для этого перенесем цвета всех ребер на исходный граф, непокрашенными останутся только ребра цикла $A-B-C-D-E$.
Если для покраски ребер $OA'$, $OE'$, $OB'$, $D'C'$ были использованы только три цвета ($a$, $b$, $c$), то без ограничения общности можно рассмотреть только два варианта раскраски: $\gamma(D'C') = \gamma(OA')$ (см. рис.~\ref{fig:text_3_graph_prim_coloring4_face5}, а) и $\gamma(D'C') \ne \gamma(OA')$ (см. рис.~\ref{fig:text_3_graph_prim_coloring4_face5}, б). 
Случай же, когда для покраски ребер $OA'$, $OE'$, $OB'$, $D'C'$ были использованы все 4 цвета, представлен на рис.~\ref{fig:text_3_graph_prim_coloring4_face5}, в.
Во всех трех случаях ребра цикла $A-B-C-D-E$ могут быть раскрашены с сохранением правильной реберной раскраски в 4 цвета.

Рассмотрев все возможные случаи нам удалось раскрасить текущий граф с $n$ вершинами в 4 цвета, опираясь на реберные раскраски графов меньших порядков.
Таким образом, предположение индукции верно, и плоский кубический граф может быть раскрашен в 4 цвета.

Доказательство существования раскраски конструктивно, алгоритм раскраски может быть построен путем стягивания гамаков и граней графа вплоть до графов минимального размера, с последующим восстановлением и посторением раскраски.
Каждое действие по стягиванию и построению раскраски текущего графа на основе раскраски графа меньшего порядка, осуществляется за количество действий $O(1)$.
Каждое стягивание уменьшает количество граней графа хотя бы на 1, таким образом общая сложность алгоритма $O(\zeta)$, что для кубического графа равносильно $O(\nu)$ или $O(\varepsilon)$.
$\blacksquare$\\

%------------------------------------------------------------------------------------------------------

Утверждение о возможности раскраски ребер плоского кубического графа в три цвета является горазд более сильным, оно верно для плоских кубических графов без мостов и равносильно задаче о четырех красках \cite{Soifer2009,Tait1880}.

%------------------------------------------------------------------------------------------------------

%\begin{lemma}\label{lem:text_3_graph_prim_cycle2_inner_space}
%Количество вершин графа, расположенных внутри двухцветного цикла, а также количество ребер, направленных внутрь двухцветного цикла, четные.
%\end{lemma}

%\begin{figure}[ht]
%\centering
%\includegraphics[width=0.5\textwidth]{./pics/text_3_graph_prim/cycle2_inner_space.pdf}
%\singlespacing
%\captionstyle{center}\caption{Структура внутренней области двухцветного цикла.}
%\label{fig:text_3_graph_prim_cycle2_inner_space}
%\end{figure}

%На рис.~\ref{fig:text_3_graph_prim_cycle2_inner_space} изображен двухцветный цикл (без ограничения общности это цикл $a/b$).
%Внутрь этого цикла направлены ребра цвета $c$.
%Пусть количество этих ребер равно $\tilde{\varepsilon}_c$ (граничные ребра).
%Пусть также кроме этих ребер во внутренней области находится $\nu$ вершин, $\varepsilon_a$ ребер цвета $a$, $\varepsilon_b$ %ребер цвета $b$ и $\varepsilon_c$ ребер цвета $c$ (внутренние ребра).

%Сумма степеней всех вершин $\nu$ равна $3\nu$, эта сумма складывается из двух концов всех внутренних ребер и одного конца граничных ребер, то есть $2\varepsilon_a + 2\varepsilon_b + 2\varepsilon_c + \tilde{\varepsilon}_c = 3\nu$.
%Так как в каждой вершине входятся ребра разных цветов, то количество концов каждого цвета во внутренней области двухцветного цикла одинаково, то есть $2\varepsilon_a = 2\varepsilon_b = 2\varepsilon_c + \tilde{\varepsilon}_c$.
%Таким образом, количество ребер разных цветов выражается следующим образом:
%\begin{equation}\label{eqn:text_3_graph_prim_cycle2_inner_space}
%	\left\{
%		\begin{aligned}
%			& \varepsilon_a = \frac{\nu}{2} \\
%			& \varepsilon_b = \frac{\nu}{2} \\
%			& \varepsilon_c = \frac{\nu - \tilde{\varepsilon}_c}{2}
%		\end{aligned}
%	\right.
%\end{equation}

%Из соотношений \eqref{eqn:text_3_graph_prim_cycle2_inner_space} и целости чисел $\varepsilon_a$, $\varepsilon_b$, $\varepsilon_c$ следует утверждение леммы. $\blacksquare$\\

%------------------------------------------------------------------------------------------------------
