Первые три главы посвящены задаче разработки метода перестроения поверхностной расчетной сетки с возможностью устранения дефектов и методов удаления самопересечений поверхностной расчетной сетки для повышения стабильности вычислений при моделировании ледообразования.
В первой главе исследуется перестроение поверхностной расчетной сетки в двумерном случае, рассматриваются известные методы перестроения, предлагается новый метод и выводятся аналитические оценки точности и сглаживания дефектов.
Во второй главе исследуется перестроение поверхностной расчетной сетки в трехмерном случае, рассматриваются известные методы и предлагается обобщение нового предложенного метода перестроения на трехмерный случай.
В третьей главе исследуется корректировка поверхностной расчетной сетки после перестроения, исследуется проблема удаления самопересечений, анализируются существующие методы, предлагаются новые методы удаления самопересечений, исследуется проблема сопряжения с объемной расчетной сеткой, возникающая вследствие изменения геометрии поверхностной сетки.
Четвертая глава посвящена задаче разработки методов повышения производительности параллельных вычислений на поверхностных и объемных расчетных сетках в моделях распараллеливания с передачей сообщений и на общей памяти.
Пятая глава посвящена задаче разработки методов векторизации программного кода и методики повышения производительности суперкомпьютерных приложений с помощью векторизации вычислений.