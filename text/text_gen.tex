% Общая проблематика работы с телами со сложной геометрией. Обзорная глава.
\newpage
\section*{Глава 1. Высокопроизводительные вычисления для областей со сложной геометрией} % выключить номер первой главы
\addcontentsline{toc}{section}{Глава 1. Высокопроизводительные вычисления для областей со сложной геометрией} % но добавить ее в оглавление
\setcounter{section}{1}                                                  % а теперь и счетчик продвинуть
\setcounter{subsection}{0}
\setcounter{figure}{0}
\setcounter{equation}{0}
\setcounter{table}{0}
\setcounter{theorem}{0}
\setcounter{lemma}{0}
\setcounter{definition}{0}

% Задачи со сложной геометрией.
Области со сложной геометрией встречаются во многих научно-практических задачах.
В качестве примеров таких задач можно привести численное моделирование
обтекания тел со сложной поверхностью \cite{Bacosi2023CFDComplex},
многофазных течений с несмешивающимися фазами \cite{Menshov2022MultiPhase},
течений в пористых средах \cite{Balashov2016Porous},
потоков с наличием взвеси твердых или жидких частиц \cite{Sposobin2022FlowsWithParticles},
движения воды над неровным дном \cite{Elizarova2014SlozhnoyeDno},
электростатического поля \cite{Sorokin2019Electrostatic} и переноса излучения \cite{Zhukovskii2015NeutronsTransfer} в сложных объектах и системах,
процессов в толще грунта \cite{Vabishchev2013Railroad} или льда \cite{Chen2019Ice} со сложными внешними границами.
Эти задачи характеризуются тем, что в расчетной области присутствуют различные подобласти, разделенные границами (поверхностями), имеющими сложную форму.
Эти границы могут быть поверхностью твердого тела, как в задаче обтекания, или разделять разные фазы многофазного течения, они могут перемещаться в расчетной области, как в задаче моделирования потока с наличием взвеси твердых частиц, или видоизменяться и перестраиваться (эволюционировать), как в задаче обледенения.

% Разные сетки.
Для численного решения задач, в которых приходится иметь дело с областями со сложной геометрией, применяются различные подходы.

% Неструктурированные сетки.
Использование неструктурированных расчетных сеток позволяет описать расчетную область произвольной сложности, они просты в построении, однако их использование связано с определенными проблемами.
По сравнению со структурованными сетками на неструктурированных сетках сложнее выполнять аппроксимацию уравнений в частных производных, а шаблоны разностной схемы для разных точек или ячеек сетки могут различаться \cite{Samarskii2011NeregSetki}.
Использование неструктурированных сеток существенно осложняет построение схем повышенной точности \cite{Abalakin2013Reconstr}. 
При решении задач газовой динамики дискретизация уравнений на неструктурированных сетках приводит к разреженной матрице коэффициентов, которая является несимметричной и не имеет диагонального преобладания, что приводит к замедлению скорости сходимости \cite{Volkov2014NestructurProblemy}.
Неструктурированные сетки более требовательны к вычислительным ресурсам, так как для их обработки требуется хранить информацию об элементах (узлы, ребра, грани, ячейки) и связях между ними.
Навигация по неструктурированной сетке связана с косвенностью обращения в память, так как количество смежных и инцидентных объектов элементов неструктурированной сетки непостоянно, и связи, как правило, хранятся в списках.
Также при использовании неструктурированных сеток возникают дополнительные сложности при попытке оптимизировать вычисления с помощью использования специализированных векторных или матричных операций микропроцессора или сопроцессора.
Существует достаточно много современных подходов, позволяющих оптимизировать применение неструктурированных сеток в плане аппроксимации и решения уравнений в частных производных: многосеточные методы (MultiGrid method, MG-method\label{abbr:mg-1}) для решения задач математической физики на неструктурированных сетках \cite{Martynenko2013MnogosetTechno}, метод Галёркина с разрывными базисными функциями (Разрывный Метод Галёркина, РМГ\label{abbr:rmg-1}, Discontinuous Galerkin Method, DGM\label{abbr:dgm-1}) для решения гиперболических систем уравнений на неструктурированных сетках \cite{Bahvalov2017Galerkin}, конечно-объемные методы с использованием полиномиальной реконструкции переменных \cite{Zhang2009PolyReconstr}, -- однако, в любом случае сложность организации вычислений на неструкурированных сетках много выше, чем на структурированных.

% Бессеточные методы.
Для решения широкого класса научно-практических задач применяются бессеточные методы, в которых вместо расчетных сеток используются наборы произвольно распределенных узлов или частиц.
Так, метод сглаженных частиц (Smoothed Particle Hydrodynamics, SPH\label{abbr:sph-1}) \cite{Liu2010SPH}, в котором в расчете участвует набор взаимодействующих между собой и обладающих материальными свойствами частиц, широко распространен при решении задач гидродинамики \cite{Potapov2024SPH}, включая задачи со свободной поверхностью и деформируемыми границами \cite{Davydov2013SPH}.
SPH прост в реализации, автоматически обеспечивает сохранение массы и позволяет производить моделирование больших деформаций, однако для получения адекватных результатов необходимо использование большого количества частиц, что приводит к существенным вычислительным затратам.
В методе дискретных вихрей (МДВ\label{abbr:mdv-1}) движение жидкости рассматривается как движение множества взаимодействующих между собой и сохраняющих свою интенсивность вихрей \cite{Sumbatyan2022Vihri}.

% Сетки с криволинейными границами.
При решении задач со сложной геометрией могут использоваться структурированные криволинейные сетки, согласующиеся с границей расчетной области \cite{Zaitsev2012BuildCurveMesh}.
Использование таких сеток требует больших вычислительных затрат для построения и поддержания ее в согласованном со сложной границей состоянии.
Особенно это критично для задач, в которых границы между подобластями изменяются с течением времени \cite{Lykosov2012SuperClimate}.
Задача построения структурированной криволинейной сетки решается с помощью поиска взаимно-однозначного отображения вычислительного пространства на физичиское пространство (см рис.~\ref{fig:gen_struct_curve_gen}).
Существуют методы решения задачи отображения с помощью трансфинитной интерполяции \cite{Titov2020WaveFields}, методом Томпсона с помощью решения уравнения Пуассона \cite{Yatsuhno2023MeshGenElliptic}, а в \cite{Khairulin2025NeuroGenStructMesh} предложено решение по генерации структурированных криволинейных сеток с использованием искусственной нейронной сети.

\begin{figure}[ht]
\centering
\includegraphics[width=0.6\textwidth]{fig/gen_struct_curve_gen.png}
\singlespacing
\captionstyle{center}\caption{Создание структурированной криволинейной сетки.}
\label{fig:gen_struct_curve_gen}
\end{figure}

% Декартовы сетки и сложные границы.
С точки зрения эффективности вычислений наиболее предпочтительно использование простых декартовых сеток, направленных вдоль осей координат.
В этом случае необходимо дополнительно выполнять аппроксимацию краевых условий на криволинейных границах.
Такие методы обладают универсальностью и могут быть использованы для решения широкого класса задач \cite{Lykosov2012SuperClimate}.
Среди методов, позволяющих выполнить аппроксимацию граничных условий и обеспечить применение декартовых сеток в сочетании с криволинейными границами, можно назвать метод ступенчатого представления границы \cite{Petrosyan2010Stupen}, метод скошенных ячеек \cite{Vinnikov2005Skosh}, метод свободной границы \cite{Lutskiy2014SvobodGran}, метод объема жидкости (Volume Of Fluid, VOF\label{abbr:vof-1}) для аппроксимации свободной поверхности \cite{Mohan2024VOF}, метод погруженной границы \cite{Mortikov2010Immersed} и другие.
В некоторых работах метод скошенных ячеек рассматривается как разновидность метода погруженной границы \cite{Abalakin2018Immersed}, а метод свободной границы имеет сходство с методом погруженной границы с использованием штрафных функций \cite{Afendikov2017SvobNotPogruzh}.

\begin{figure}[ht]
\centering
\includegraphics[width=0.8\textwidth]{fig/gen_immersed.pdf}
\singlespacing
\captionstyle{center}\caption{Точки, используемые для аппроксимации физических величин в фиктивных ячейках в методе погруженной границы.}
\label{fig:gen_immersed}
\end{figure}

% Метод погруженной границы.
В методе погруженной границы с использованием штрафных функций граница раздела двух сред моделируется путем добавления в исходное уравнение источниковых членов, учитывающих влияние границы \cite{Abalakin2018Immersed}.
В методе погруженной границы с использованием фиктивных ячеек, одной из областей применения которого является моделирование течений вблизи поверхности сложной формы, расчетная область разделена на внешнюю область, в которой моделируется течение, и внутреннюю область, представляющую собой обтекаемое тело (см. рис.~\ref{fig:gen_immersed}).
Наряду с обычными ячейками, находящимися во внешней области (внешние ячейки) для расчетов используются также ячейки из внутренней области, входящие в вычислительный шаблон хотя бы одной внешней ячейки, они называются фиктивными ячейками.
Во время расчетов параметры течения в фиктивных ячейках аппроксимируются через параметры в других точках расчетной области с учетом условий на границе.
На рис.~\ref{fig:gen_immersed} в двумерной иллюстрации показаны возможные точки, используемые при аппроксимации для точки фиктивной ячейки $G$: точки внешних ячеек $F_i$, ближайшая точка границы $H$, середина отрезка $XY$ -- точка $M$ (где $X$, $Y$ -- точки пересечения параллельных осям координат прямых $GX$, $GY$ с границей \cite{Mittal2005Immersed}), образ $R$ точки $G$ относительно границы \cite{Yousefzadeh2019Immersed}, -- могут быть использованы и другие точки.
Известны различные подходы к аппроксимации параметров в фиктивных точках, включая линейную аппроксимацию \cite{Vinnikov2007Immersed}, билинейную/трилинейную аппроксимацию \cite{Luo2016Immersed}, квадратичную аппроксимацию \cite{Tseng2003Immersed}, разложение параметров в фиктивной точке в ряд Тейлора \cite{Peter2016Immersed}.
