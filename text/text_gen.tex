% Общая проблематика работы с телами со сложной геометрией. Обзорная глава.
\newpage
\section*{Глава 1. Высокопроизводительные вычисления для областей со сложной геометрией} % выключить номер первой главы
\addcontentsline{toc}{section}{Глава 1. Высокопроизводительные вычисления для областей со сложной геометрией} % но добавить ее в оглавление
\setcounter{section}{1}                                                  % а теперь и счетчик продвинуть
\setcounter{subsection}{0}
\setcounter{figure}{0}
\setcounter{equation}{0}
\setcounter{table}{0}
\setcounter{theorem}{0}
\setcounter{lemma}{0}
\setcounter{definition}{0}

% Задачи со сложной геометрией.
Области со сложной геометрией встречаются во многих научно-практических задачах.
В качестве примеров таких задач можно привести численное моделирование
обтекания тел со сложной поверхностью \cite{Bacosi2023CFDComplex},
многофазных течений с несмешивающимися фазами \cite{Menshov2022MultiPhase},
течений в пористых средах \cite{Balashov2016Porous},
потоков с наличием взвеси твердых или жидких частиц \cite{Sposobin2022FlowsWithParticles},
движения воды над неровным дном \cite{Elizarova2014SlozhnoyeDno},
электростатического поля \cite{Sorokin2019Electrostatic} и переноса излучения \cite{Zhukovskii2015NeutronsTransfer} в сложных объектах и системах,
процессов в толще грунта \cite{Vabishchev2013Railroad} или льда \cite{Chen2019Ice} со сложными внешними границами.
Эти задачи характеризуются тем, что в расчетной области присутствуют различные подобласти, разделенные границами (поверхностями), имеющими сложную форму.
Эти границы могут быть поверхностью твердого тела, как в задаче обтекания, или разделять разные фазы многофазного течения, они могут перемещаться в расчетной области, как в задаче моделирования потока с наличием взвеси твердых частиц, или видоизменяться и перестраиваться (эволюционировать), как в задаче обледенения.

% Разные сетки.
Для численного решения задач, в которых приходится иметь дело с областями со сложной геометрией, применяются различные подходы.

% Неструктурированные сетки.
Использование неструктурированных расчетных сеток позволяет описать расчетную область произвольной сложности, они просты в построении, однако их использование связано с определенными проблемами.
По сравнению со структурованными сетками на неструктурированных сетках сложнее выполнять аппроксимацию уравнений в частных производных, а шаблоны разностной схемы для разных точек или ячеек сетки могут различаться \cite{Samarskii2011NeregSetki}.
Использование неструктурированных сеток существенно осложняет построение схем повышенной точности \cite{Abalakin2013Reconstr}. 
При решении задач газовой динамики дискретизация уравнений на неструктурированных сетках приводит к разреженной матрице коэффициентов, которая является несимметричной и не имеет диагонального преобладания, что приводит к замедлению скорости сходимости \cite{Volkov2014NestructurProblemy}.
Неструктурированные сетки более требовательны к вычислительным ресурсам, так как для их обработки требуется хранить информацию об элементах (узлы, ребра, грани, ячейки) и связях между ними.
Навигация по неструктурированной сетке связана с косвенностью обращения в память, так как количество смежных и инцидентных объектов элементов неструктурированной сетки непостоянно, и связи, как правило, хранятся в списках.
Также при использовании неструктурированных сеток возникают дополнительные сложности при попытке оптимизировать вычисления с помощью использования специализированных векторных или матричных операций микропроцессора или сопроцессора.
Существует достаточно много современных подходов, позволяющих оптимизировать применение неструктурированных сеток в плане аппроксимации и решения уравнений в частных производных: многосеточные методы (MultiGrid method, MG-method\label{abbr:mg-1}) для решения задач математической физики на неструктурированных сетках \cite{Martynenko2013MnogosetTechno}, метод Галёркина с разрывными базисными функциями (Разрывный Метод Галёркина, РМГ\label{abbr:rmg-1}, Discontinuous Galerkin Method, DGM\label{abbr:dgm-1}) для решения гиперболических систем уравнений на неструктурированных сетках \cite{Bahvalov2017Galerkin}, конечно-объемные методы с использованием полиномиальной реконструкции переменных \cite{Zhang2009PolyReconstr}, -- однако, в любом случае сложность организации вычислений на неструкурированных сетках много выше, чем на структурированных.

% Бессеточные методы.
Для решения широкого класса научно-практических задач применяются бессеточные методы, в которых вместо расчетных сеток используются наборы произвольно распределенных узлов или частиц.
Так, метод сглаженных частиц (Smoothed Particle Hydrodynamics, SPH\label{abbr:sph-1}) \cite{Liu2010SPH}, в котором в расчете участвует набор взаимодействующих между собой и обладающих материальными свойствами частиц, широко распространен при решении задач гидродинамики \cite{Potapov2024SPH}, включая задачи со свободной поверхностью и деформируемыми границами \cite{Davydov2013SPH}.
SPH прост в реализации, автоматически обеспечивает сохранение массы и позволяет производить моделирование больших деформаций, однако для получения адекватных результатов необходимо использование большого количества частиц, что приводит к существенным вычислительным затратам.
В методе дискретных вихрей (МДВ\label{abbr:mdv-1}) движение жидкости рассматривается как движение множества взаимодействующих между собой и сохраняющих свою интенсивность вихрей \cite{Sumbatyan2022Vihri}.

% Сетки с криволинейными границами.
При решении задач со сложной геометрией могут использоваться структурированные криволинейные сетки, согласующиеся с границей расчетной области \cite{Zaitsev2012BuildCurveMesh}.
Использование таких сеток требует больших вычислительных затрат для построения и поддержания ее в согласованном со сложной границей состоянии.
Особенно это критично для задач, в которых границы между подобластями изменяются с течением времени \cite{Lykosov2012SuperClimate}.
Задача построения структурированной криволинейной сетки решается с помощью поиска взаимно-однозначного отображения вычислительного пространства на физичиское пространство (см рис.~\ref{fig:gen_struct_curve_gen}).
Существуют методы решения задачи отображения с помощью трансфинитной интерполяции \cite{Titov2020WaveFields}, методом Томпсона с помощью решения уравнения Пуассона \cite{Yatsuhno2023MeshGenElliptic}, а в \cite{Khairulin2025NeuroGenStructMesh} предложено решение по генерации структурированных криволинейных сеток с использованием искусственной нейронной сети.

\begin{figure}[ht]
\centering
\includegraphics[width=0.6\textwidth]{fig/gen_struct_curve_gen.png}
\singlespacing
\captionstyle{center}\caption{Создание структурированной криволинейной сетки.}
\label{fig:gen_struct_curve_gen}
\end{figure}

% Декартовы сетки и сложные границы.
С точки зрения эффективности вычислений наиболее предпочтительно использование простых декартовых сеток, направленных вдоль осей координат.
В этом случае необходимо дополнительно выполнять аппроксимацию краевых условий на криволинейных границах.
Такие методы обладают универсальностью и могут быть использованы для решения широкого класса задач \cite{Lykosov2012SuperClimate}.
Среди методов, позволяющих выполнить аппроксимацию граничных условий и обеспечить применение декартовых сеток в сочетании с криволинейными границами, можно назвать метод ступенчатого представления границы \cite{Petrosyan2010Stupen}, метод скошенных ячеек \cite{Vinnikov2005Skosh}, метод свободной границы \cite{Lutskiy2014SvobodGran}, метод объема жидкости (Volume Of Fluid, VOF\label{abbr:vof-1}) для аппроксимации свободной поверхности \cite{Mohan2024VOF}, метод погруженной границы \cite{Mortikov2010Immersed} и другие.
В некоторых работах метод скошенных ячеек рассматривается как разновидность метода погруженной границы \cite{Abalakin2018Immersed}, а метод свободной границы имеет сходство с методом погруженной границы с использованием штрафных функций \cite{Afendikov2017SvobNotPogruzh}.

\begin{figure}[ht]
\centering
\includegraphics[width=0.8\textwidth]{fig/gen_immersed.pdf}
\singlespacing
\captionstyle{center}\caption{Точки, используемые для вычисления физических величин в фиктивных ячейках в методе погруженной границы.}
\label{fig:gen_immersed}
\end{figure}

% Метод погруженной границы.
В методе погруженной границы с использованием штрафных функций граница раздела двух сред моделируется путем добавления в исходное уравнение источниковых членов, учитывающих влияние границы \cite{Abalakin2018Immersed}.
В методе погруженной границы с использованием фиктивных ячеек, одной из областей применения которого является моделирование течений вблизи поверхности сложной формы, расчетная область разделена на внешнюю область, в которой моделируется течение, и внутреннюю область, представляющую собой обтекаемое тело (см. рис.~\ref{fig:gen_immersed}).
Наряду с обычными ячейками, находящимися во внешней области (внешние ячейки) для расчетов используются также ячейки из внутренней области, входящие в вычислительный шаблон хотя бы одной внешней ячейки, они называются фиктивными ячейками.
Во время расчетов параметры течения в фиктивных ячейках определяются через параметры в других точках расчетной области с помощью интерполяции с учетом условий на границе.
На рис.~\ref{fig:gen_immersed} в двумерной иллюстрации показаны возможные точки, используемые при определении параметров для точки фиктивной ячейки $G$: точки внешних ячеек $F_i$, ближайшая точка границы $H$, середина отрезка $XY$ -- точка $M$ (где $X$, $Y$ -- точки пересечения параллельных осям координат прямых $GX$, $GY$ с границей \cite{Mittal2005Immersed}), образ $R$ точки $G$ относительно границы \cite{Yousefzadeh2019Immersed}, -- могут быть использованы и другие точки.
Известны различные подходы к интерполяции параметров в фиктивных точках, включая линейную интерполяцию \cite{Vinnikov2007Immersed}, билинейную/трилинейную интерполяцию \cite{Luo2016Immersed}, квадратичную интерполяцию \cite{Tseng2003Immersed}, разложение параметров в фиктивной точке в ряд Тейлора \cite{Peter2016Immersed}.

% Можно сконцентрироваться на вычислениях на декартовых сетках.
С помощью перечисленных выше методов можно организовать вычисления внутри областей со сложной геометрией, используя при этом блочно-структурированные сетки, что открывает широкие возможности для повышения производительности вычислений.
Организация данных внутри блочно-структурированных сеток в виде наборов многомерных массивов и инвариантность вычислительных шаблонов для ячеек таких сеток позволяют организовывать вычисления с высокой степенью параллелизма на всех уровнях: на уровне отдельных вычислительных узлов -- на распределенной памяти, на уровне параллельных потоков -- на общей памяти и на уровне отдельных инструкций.
Использование упорядоченных структурированных данных позволяет существено упростить организацию распараллеливания вычислений между отдельными вычислительными процессами и потоками, что применется во многих расчетных приложениях \cite{VolkovBogorodskii2010UseMPI,Milyukova2023UseMPI,Saczek2018UseMPI}.
Наличие упорядоченных структур позволяет быстро и эффективно распределить вычисления по процессам/потокам с минимальным количеством накладных расходов, так как структурированная расчетная область допускает достаточно простую декомпозицию (см. рис.~\ref{fig:gen_struct_block}).
Параллелизм на уровне отдельных инструкций выражен во VLIW (Very Long Instruction Word\label{abbr:vliw-1}) архитектурах \cite{Tyulyaeva2016UseVLIW}, а также в вычислениях, при организации которых одна инструкция применяется сразу к некоторому набору данных, что имеет место при использовании GPU\label{abbr:gpu-2} \cite{Volkov2019UseGPU} и векторных инструкций CPU\label{abbr:cpu-2} \cite{Brykov2022UseVect}.

\begin{figure}[ht]
\centering
\includegraphics[width=0.6\textwidth]{fig/gen_struct_block.png}
\singlespacing
\captionstyle{center}\caption{Различные примеры декомпозиции структурированной расчетной области.}
\label{fig:gen_struct_block}
\end{figure}

% Адаптивные сетки.
В качестве дополнительного аспекта организации вычислений на блочно-структурированных сетках можно отметить возможность использования адаптивности (Adaptive Mesh Refinement, AMR\label{abbr:amr-1}).
Так как блочно-структурированные сетки отличаются простотой реализации и скоростью обработки, то функционал для работы с ними может быть расширен с помощью дробления отдельных ячеек на более мелкие, адаптируюясь таким образом к сложной геометрии \cite{Plenkin2015Adaptive} или особенностям решения \cite{Bragin2019Adaptive}.
Локальная адаптация может применяться также на уровне блоков сетки, а не отдельных ячеек \cite{Borisov2015Adaptive}.

% Остаются вычисления, связанные с самой поверхностью.
При выполнении вычислений в расчетной области со сложной границей сама эта сложная граница должна быть описана достаточно точно.
В некоторых программных кодах описание сложной границы выполняется через адаптацию расчетной сетки, без явного представления поверхности, как это имеет место в программном комплексе FlowVision \cite{Zhluktov2010SurfaceFlowVision}.
Однако, в большинстве случаев в программных пакетах для суперкомпьютерного моделирование отдается предпочтение представлению сложной границы с помощью поверхностной неструктурированной сетки \cite{Liu2025SurfaceMesh}.
Использование поверхностной неструктурированной сетки для представления сложной границы встречается в таких широко известных программным пакетах как ЛОГОС \cite{Borisenko2020Surface}, ANSYS \cite{ANSYS2021Mesh}, Star-CCM+ \cite{StarCCM2023Mesh}, Comsol Multiphysics \cite{Comsol2020Mesh} и многих других.
В общем случае для описания произвольной геометрии сложной поверхности необходимо использование расчетных сеток с треугольными ячейками, так как при наличии более трех узлов в ячейке невозможно гарантировать, что она останется плоской.
Обычным методом построения таких сеток является метод подвижного фронта \cite{Ermakov2020GenMesh}, в котором сначала строится неструктурированная сетка на плоскости, начиная с границ области и продвигаясь внутрь путем добавления новых треугольников, а затем построенная сетка переносится на трехмерную поверхность с помощью отображения. 
Различные усовершенствованные методы построения позволяют повысить детализацию сетки в зависимости от свойст расчетной области \cite{Fan2024GenMesh}, а также ускорить построение для моделей высокой детализации и выполнить коррекцию результирующей сетки \cite{Evstifeeva2022GenMesg}.

% Как параллелить вычисления на поверхности.
Для получения качественных результатов моделирования на сложных границах расчетных областей необходимо выполнять вычисления на поверхностной сетке.
К таким задачам, в частности, относятся расчет высоты ледяного покрова обтекаемой воздушным потоком поверхности \cite{Strijhak2021Ice}, расчет течения тонкого слоя жидкости по поверхности, \cite{Gubaidullin2017Plenka}, расчет МГД-устойчивости двумерных плазменных конфигураций \cite{Medvedev2022Plazm} и другие задачи.
При выполнении вычислений на поверхностной неструктурированной сетке, так же как и в случае блочно-структурированных сеток, для повышения производительности расчетов необходимо использовать все доступные уровни распараллеливания.

% Декомпозиция расчетной сетки.
При выполнении распараллеливания на распределенной памяти возникает проблема декомпозиции неструктурированной сетки и организации межпроцессных обменов.
В результате декомпозиции граф разбивается на подграфы, которые соотвествуют доменам сетки, обрабатываемым в отдельных вычислительных процессах.
Декомпозиция неструктурированной сетки, или ее дуального графа, существенным образом отличается от аналогичной задачи, решаемой для структурированной сетки.
Решению задачи о декомпозиции неструктурированной сетки на близкие по размеру домены с минимизацией количества ребер, находащихся на границах между разными доменами, посвящено большое количество исследований \cite{Zheleznyakova2017Decomp}.
К наиболее известным методам декомпозиции можно отнести рекурсивную геометрическую бисекцию, заполнение пространства с помощью кривой Гильберта, различные алгоритмы наращивания доменов, локально-корректирующие алгоритмы Кернигана-Лина и Фидуччи-Маттейсеса и их модификации, спектральную бисекцию, иерархические алгоритмы с использованием огрубления и уточнения сетки (см. рис.~\ref{fig:gen_deomp_hierarchy}), и многие другие.

\begin{figure}[ht]
\centering
\includegraphics[width=0.6\textwidth]{fig/gen_decomp_hierarchy.png}
\singlespacing
\captionstyle{center}\caption{Схема иерархической декомпозиции расчетной сетки.}
\label{fig:gen_deomp_hierarchy}
\end{figure}

Геометрическая бисекция основана на разделении произвольного домена на две равные части, опираясь на геометрическое положение вершин графа -- множество вершин может быт разбито как с помощью одной из плоскостей, параллельных координатным осям $OX$, $OY$, $OZ$, так с помощью прямой общего положения, и вообще с помощью произвольного отношения порядка, введенного для вершин дуального графа сетки \cite{Yakobovskii2013Decomp}.
Декомпозиция на основе кривой Гильберта опирается на свойство локальности этой кривой -- звенья кривой с близкими номерами имеют близкие координаты в пространстве, что позволяет строить достаточно компактные домены \cite{Arzumanyan2015Decomp}.
Декомпозиция с помощью наращивания доменов представляет собой группу алгоритмов, в которых домены, инициируемые некоторыми начальными вершинами, последовательно расширяются путем добавления ближайших к ним соседей, и это расширение может выполняться последовательно, как в первых реализациях (на примере алгорима Фархата \cite{Farhat1988Decomp}), так и одновременно для всех доменов \cite{Golovchenko2014Decomp}.
Инкрементный метод декомпозиции основан на понятии ядра домена заданного уровня, определяемого через кратчайшее расстояние от вершины до множества граничных вершин домена, и ориентирован на формирование доменов с соблюдением связности ядер заданного уровня \cite{Yakobovskii2005Decomp}.
Алгоритм Кернигана-Лина является алгоритмом уточнения разбиения на два домена $A$ и $B$, он основан на итерационном обмене этих доменов вершинами между собой, то есть на каждой итерации алгоритма оценивается полезность обмена вершинами $a \in A$ и $b \in B$ на основании количества внутренних и внешних связей этих вершин в своих доменах \cite{Kernighan1970Decomp}.
Алгоритм Феддучи-Маттейсеса предлагает эвристику для нахождения перемещаемой между доменами точки с линейной сложностью, что позволяет достаточно быстро производить локальную коррекцию разбиения \cite{Fiduccia1982Decomp}.
Метод спектральной бисекции основан на использовании спектральной матрицы графа (матрицы Лапласа), и направлен на минимизацию количества ребер, соединяющих вершины из разных доменов, что эквивалентно задаче минимизации $\sum_{i, j: e_{ij} \in E}{(x_i - x_j)^2}$ при $\sum_{i = 1}^{|V|}{x_i} = 0$, где $V$ -- множество вершин графа, $E$ -- множество ребер графа, $e_{ij}$ -- ребро между ввершинами с индексами $i$ и $j$, $x_i = \pm 1$ -- признак принадлежности вершины одному из двух доменов \cite{Berzun2013Decomp}.
Иерархические алгоритмы основаны на трех основных действиях: огрубление графа -- в процессе которого создается последовательность постепенно огрубленных графов меньшего размера, начальная декомпозиция -- разбиение наиболее огрубленного графа на домены, восстановление графа -- обратное движение по последовательности огрубленных графов с выполнением локального уточнения декомпозиации на каждом шаге \cite{Yakobovskii2009Decomp}.
В одном из первых пакетов декомпозиции графов Chaco поддержаны различные методы декомпозиции, включая спектральную, геометрическую и иерархическую \cite{Hendrickson2011Chaco}.
Среди других классических пакетов декомпозиции можно отметить ParMETIS с реализаций иерархической декомпозиции \cite{Karypis2013ParMETIS}, основанный на алгоритме Фархата и последующих локальных уточнениях декомпозиции пакет JOSTLE \cite{McManus1995JOSTLE}, пакет с рекурсивной бисекционной декомпозицией SCOTCH \cite{Pellegrini1996SCOTCH} и многие другие \cite{Voropinov2009Decomp}.
В современном отечественном пакете GridSpiderPar реализован параллельный алгоритм инкрементной декомпозиции, превосходящий зарубежные аналоги по дисбалансу вершин в доменах, количеству несвязных доменов и количеству ребер, соединяющих вершины из разных доменов \cite{Golovchenko2015GridSpiderPar}.

% Общая память.
При выполнении вычислений с использованием многоядерных микропроцессоров возникают вопросы распараллеливания вычислений на общей памяти, когда несколько параллельных потоков одновременно имеют доступ в одной и той же области памяти.
При организации вычислений неизбежно возникают конфликты по обращению к данным -- когда два потока потенциально могут обраться на запись к одному и тому же элементу данных в памяти (классический пример операции \texttt{x += v}).
При использовании стандарта распараллеливания программ OpenMP \cite{Akimova2023OpenMP} такие конфликты могут быть разрешены с помощью использования атомарных операций или критических секций (директивы \texttt{atomic}, \texttt{critical}).
Как правило, использование средств OpenMP приводит к снижению эффективности вычислений, поэтому возникает необходимость разработки алгоритмических решений к разрешению конфликтов.
Так для конечно-элементных вычислений описаны подходы к разрешению конфликтов, основанные на изменении порядка суммирования данных при вычислениях \cite{Kopysov2019SharedMem}, а также на разделении конечно-элементных сеток на отдельные слои \cite{Novikov2016SharedMem}.
При реализации конечно-объемных численных методов на неструктурированных сетках полностью устранить конфликты по данным при использовании общей памяти позволяет использование разбиение множества обрабатываемых элементов сетки на неконфликтующие подмножества, что выполняется с помощью раскрасок графа \cite{Gulicheva2022SharedMem}.

% Как перестраивать поверхность.
