% Общая проблематика работы с телами со сложной геометрией. Обзорная глава.
\newpage
\section*{Глава 1. Высокопроизводительные вычисления для областей со сложной геометрией} % выключить номер первой главы
\addcontentsline{toc}{section}{Глава 1. Высокопроизводительные вычисления для областей со сложной геометрией} % но добавить ее в оглавление
\setcounter{section}{1}                                                  % а теперь и счетчик продвинуть
\setcounter{subsection}{0}
\setcounter{figure}{0}
\setcounter{equation}{0}
\setcounter{table}{0}
\setcounter{theorem}{0}
\setcounter{lemma}{0}
\setcounter{definition}{0}

% Задачи со сложной геометрией.
Области со сложной геометрией встречаются во многих научно-практических задачах.
В качестве примеров таких задач можно привести численное моделирование
обтекания тел со сложной поверхностью \cite{Bacosi2023CFDComplex},
многофазных течений с несмешивающимися фазами \cite{Menshov2022MultiPhase},
течений в пористых средах \cite{Balashov2016Porous},
потоков с наличием взвеси твердых или жидких частиц \cite{Sposobin2022FlowsWithParticles},
движения воды над неровным дном \cite{Elizarova2014SlozhnoyeDno},
электростатического поля \cite{Sorokin2019Electrostatic} и переноса излучения \cite{Zhukovskii2015NeutronsTransfer} в сложных объектах и системах,
процессов в толще грунта \cite{Vabishchev2013Railroad} или льда \cite{Chen2019Ice} со сложными внешними границами.
Эти задачи характеризуются тем, что в расчетной области присутствуют различные подобласти, разделенные границами (поверхностями), имеющими сложную форму.
Эти границы могут быть поверхностью твердого тела, как в задаче обтекания, или разделять разные фазы многофазного течения, они могут перемещаться в расчетной области, как в задаче моделирования потока с наличием взвеси твердых частиц, или видоизменяться и перестраиваться (эволюционировать), как в задаче обледенения.

% Разные сетки.
Для численного решения задач, в которых приходится иметь дело с областями со сложной геометрией, применяются различные подходы.

% Неструктурированные сетки.
Использование неструктурированных расчетных сеток позволяет описать расчетную область произвольной сложности, они просты в построении, однако их использование связано с определенными проблемами.
По сравнению со структурованными сетками на неструктурированных сетках сложнее выполнять аппроксимацию уравнений в частных производных, а шаблоны разностной схемы для разных точек или ячеек сетки могут различаться \cite{Samarskii2011NeregSetki}.
Использование неструктурированных сеток существенно осложняет построение схем повышенной точности \cite{Abalakin2013Reconstr}. 
При решении задач газовой динамики дискретизация уравнений на неструктурированных сетках приводит к разреженной матрице коэффициентов, которая является несимметричной и не имеет диагонального преобладания, что приводит к замедлению скорости сходимости \cite{Volkov2014NestructurProblemy}.
Неструктурированные сетки более требовательны к вычислительным ресурсам, так как для их обработки требуется хранить информацию об элементах (узлы, ребра, грани, ячейки) и связях между ними.
Навигация по неструктурированной сетке связана с косвенностью обращения в память, так как количество смежных и инцидентных объектов элементов неструктурированной сетки непостоянно, и связи, как правило, хранятся в списках.
Также при использовании неструктурированных сеток возникают дополнительные сложности при попытке оптимизировать вычисления с помощью использования специализированных векторных или матричных операций микропроцессора или сопроцессора.
Существует достаточно много современных подходов, позволяющих оптимизировать применение неструктурированных сеток в плане аппроксимации и решения уравнений в частных производных: многосеточные методы (MultiGrid method, MG-method\label{abbr:mg-1}) для решения задач математической физики на неструктурированных сетках \cite{Martynenko2013MnogosetTechno}, метод Галёркина с разрывными базисными функциями (Разрывный Метод Галёркина, РМГ\label{abbr:rmg-1}, Discontinuous Galerkin Method, DGM\label{abbr:dgm-1}) для решения гиперболических систем уравнений на неструктурированных сетках \cite{Bahvalov2017Galerkin}, конечно-объемные методы с использованием полиномиальной реконструкции переменных \cite{Zhang2009PolyReconstr}, -- однако, в любом случае сложность организации вычислений на неструкурированных сетках много выше, чем на структурированных.

% Бессеточные методы.
Для решения широкого класса научно-практических задач применяются бессеточные методы, в которых вместо расчетных сеток используются наборы произвольно распределенных узлов или частиц.
Так, метод сглаженных частиц (Smoothed Particle Hydrodynamics, SPH\label{abbr:sph-1}) \cite{Liu2010SPH}, в котором в расчете участвует набор взаимодействующих между собой и обладающих материальными свойствами частиц, широко распространен при решении задач гидродинамики \cite{Potapov2024SPH}, включая задачи со свободной поверхностью и деформируемыми границами \cite{Davydov2013SPH}.
SPH прост в реализации, автоматически обеспечивает сохранение массы и позволяет производить моделирование больших деформаций, однако для получения адекватных результатов необходимо использование большого количества частиц, что приводит к существенным вычислительным затратам.
В методе дискретных вихрей (МДВ\label{abbr:mdv-1}) движение жидкости рассматривается как движение множества взаимодействующих между собой и сохраняющих свою интенсивность вихрей \cite{Sumbatyan2022Vihri}.

Использование блочно-структурированных криволинейных сеток требует больших вычислительных затрат для построения сетки и поддержания ее в согласованном со сложной границей состоянии.
Особенно это критично для задач, в которых границы между подобластями изменяются с течением времени.
Компромисом является использование более простых декартовых сеток с аппроксимацией краевых условий на криволинейных границах.
Такие методы обладают универсальностью и могут быть использованы для решения широкого класса задач \cite{Lykosov2012SuperClimate}.

метод свободной границы ?

метод погруженной границы \cite{Mortikov2010Immersed}
