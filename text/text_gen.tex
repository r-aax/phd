% Общая проблематика работы с телами со сложной геометрией. Обзорная глава.
\newpage
\section*{Глава 1. Высокопроизводительные вычисления для областей со сложной геометрией} % выключить номер первой главы
\addcontentsline{toc}{section}{Глава 1. Высокопроизводительные вычисления для областей со сложной геометрией} % но добавить ее в оглавление
\setcounter{section}{1}                                                  % а теперь и счетчик продвинуть
\setcounter{subsection}{0}
\setcounter{figure}{0}
\setcounter{equation}{0}
\setcounter{table}{0}
\setcounter{theorem}{0}
\setcounter{lemma}{0}
\setcounter{definition}{0}

% Задачи со сложной геометрией.
Области со сложной геометрией встречаются во многих научно-практических задачах.
В качестве примеров таких задач можно привести численное моделирование
обтекания тел со сложной поверхностью \cite{Bacosi2023CFDComplex},
многофазных течений с несмешивающимися фазами \cite{Menshov2022MultiPhase},
течений в пористых средах \cite{Balashov2016Porous},
потоков с наличием взвеси твердых или жидких частиц \cite{Sposobin2022FlowsWithParticles},
электростатического поля \cite{Sorokin2019Electrostatic} и переноса излучения \cite{Zhukovskii2015NeutronsTransfer} в сложных объектах и системах,
процессов в толще грунта \cite{Vabishchev2013Railroad} или льда \cite{Chen2019Ice} со сложными внешними границами.
Эти задачи характеризуются тем, что в расчетной области присутствуют различные подобласти, разделенные границами (поверхностями), имеющими сложную форму.
Эти границы могут быть поверхностью твердого тела, как в задаче обтекания, или разделять разные фазы многофазного течения, они могут перемещаться в расчетной области, как в задаче моделирования потока с наличием взвеси твердых частиц, или видоизменяться и перестраиваться (эволюционировать), как в задаче обледенения.

% Разные сетки.
Для численного решения задач, в которых приходится иметь дело с областями со сложной геометрией, могут быть быть применены различные подходы.
Использование неструктурированный расчетных сеток позволяет...
Использование бессеточных методов позволяет...
Использование блочно-структурированных криволинейных сеток требует больших вычислительных затрат для построения сетки и поддержания ее в согласованном со сложной границей состоянии.
Особенно это критично для задач, в которых границы между подобластями изменяются с течением времени.
Компромисом является использование более простых декартовых сеток с аппроксимацией краевых условий на криволинейных границах.
Такие методы обладают универсальностью и могут быть использованы для решения широкого класса задач \cite{Lykosov2012SuperClimate}.

метод свободной границы ?

метод погруженной границы \cite{Mortikov2010Immersed}
