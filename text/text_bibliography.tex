% Список литературы.
\newpage
\renewcommand{\baselinestretch}{1.0}
\begin{thebibliography}{99}
\setlength{\itemsep}{0pt}
\addcontentsline{toc}{section}{Список литературы} % добавить в оглавление список литературы

%---------------------------------------------------------------------------------------------------
% Введение.

% ГОСТ по высокопроизводительным вычислительным системам.
\bibitem{GOST57700HPC}ГОСТ Р 57700.27-2020. Высокопроизводительные вычислительные системы. Термины и определения.

% Приоритет - а) переход к передовым технологиям проектирования и создания высокотехнологичной продукции.
% Авиация.
\bibitem{Kornev2021SuperAvio}\textbf{Корнев~А.~В., Козелков~А.~С.} Применение отечественных суперкомпьютерных технологий для создания перспективных образцов авиационной техники. // Современные информационные технологии и ИТ-образование, 2021, Т.~17, №~2, С.~250-264. DOI:~10.25559/SITITO.17.202102.250-264
% Автомобили.
\bibitem{Wang2020SuperAuto}\textbf{Wang~Y., Han~P., Hou~J.} Design and implement of HPC simulation cloud platform for automobile industry. // In book: International Conference on Applications and Techniques in Cyber Intelligence ATCI 2019, 2020. DOI:~10.1007/978-3-030-25128-4\_38
% Судостроение.
\bibitem{Nikitin2018SuperShip}\textbf{Никитин~В.~С.} Численное моделирование -- эффективный подход к решению прикладных задач судостроения. // Труды Крыловского государственного научного центра, 2018, 2(384). DOI:~10.24937/2542-2324-2018-2-384-5-8
% Железная дорога.
\bibitem{Solovyev2013SuperTrains}\textbf{Соловьев~В.~П.} О возможностях суперкомпьютерных и грид-технологий и перспективах их внедрения на железнодорожном транспорте. // Бюллетень объединенного ученого совета ОАО РЖД, 2013, №~4, С.~1-15.
% Турбины.
\bibitem{Duben2022SuperTurbine}\textbf{Duben~A., Gorobets~A., Soukov~S., Marakueva~O., Shuvaev~N., Zagitov~R.} Supercomputer simulations of turbomachinery problems with higher accuracy on unstructured meshes. // In: Voevodin~V., Sobolev~S., Yakobovskiy~M., Shagaliev~R. (eds) Supercomputing, RuSCDays 2022, Lecture Notes in Computer Science, Vol.~13708, Springer, Cham. DOI:~10.1007/978-3-031-22941-1\_26 
% Военная техника, нац. безопасность.
\bibitem{Ageeva2023SuperMilitary}\textbf{Агеева~А.~Ф.} Роль суперкомпьютеров в вопросах национальной безопасности. // Развитие экономики в условиях цифровой трансформации: Проблемы, достижения и инновации, 2023, №~1. DOI:~ 10.51409/v.a.2023.03.01.005
% Молекулярная динамика.
\bibitem{Wang2025SuperMolDyn}\textbf{Wang~X., Meng~X., Guo~Z. et al.} 29-billion atoms molecular dynamics simulation with Ab Initio accuracy on 35 million cores of new Sunway supercomputer. // IEEE Transactions on Computers, 2025, 99, P.~1-14. DOI:~10.1109/TC.2025.3540646

% Приоритет - б) переход к экологически чистой и ресурсосберегающей энергетике, повышение эффективности добычи и глубокой переработки углеводородного сырья, формирование новых источников энергии, способов ее передачи и хранения.
% Атомная станция.
\bibitem{Cancemi2025SuperNuc}\textbf{Cancemi~S., Frano~R., Angelucci~M.} HPC modelling for nuclear safety assessment. // Journal of Physics: Conference Series, 2025, 2940:012025. DOI:~10.1088/1742-6596/2940/1/012025
% Моделирование процессов в ядерных реакторах - движение нейтронов и взаимодействие с веществом.
\bibitem{Zhang2025SuperNuclear}\textbf{Zhang~Z., Liu~T., Wang~C., Guo~Y., Pan~J., Zhao~D., Wu~X., Yang~M.} Parallel optimization of Monte Carlo neutron transport method based on Sunway Bluelight II Supercomputer. // The Journal of Supercomputing, 2025, 81:706. DOI:~10.1007/s11227-025-07190-1
% Ветряки.
\bibitem{Quint2025SuperWind}\textbf{Quint~D., Lundquist~J., Rosencrans~D.} Simulations suggest offshore wind farms modify low-level jets. // Wind Energy Systems, 2025, 10(1), P.~117-142. DOI:~10.5194/wes-10-117-2025
% Приливные.
\bibitem{Parrado2024SuperTidal}\textbf{Parrado~D., Rueda-Bayona~J.} Three-dimensional hydrodynamic and CFD modeling of a tidal barrage power plant without sluicing in Buenaventura, Colombia. // Infraindustrias, 2024, 9, 127. DOI:~10.3390/infrastructures9080127
% Модель пласта.
\bibitem{Usmanov2024SuperPlast}\textbf{Усманов~Д.~И., Поташев~К.~А., Салимьянова~Д.~Р.} Идентификация граничных условий фильтрационной модели нефтяного пласта по замерам давления в скважинах. Часть 1: Однородный пласт. // Ученые записки Казанского университета. Серия Физико-математическе науки, 2024, Т.~166, Кн.~4, С.~603-623. DOI:~10.26907/2541-7746.2024.4.603-623
% Нефть и газ.
\bibitem{Didenko2023SuperOil}\textbf{Didenko~N., Skripnuk~D., Merkulov~V., Kikkas~K., Skripniuk~K.} Methodology for the formation of a digital model of the life cycle of an offshore oil and gas platform. // Resources, 2023, 12, 86. DOI:~10.3390/resources12080086

% Приоритет - в) переход к персонализированной, предиктивной и профилактической медицине, высокотехнологичному здравоохранению и технологиям здоровьесбережения, в том числе за счет рационального применения лекарственных препаратов.
% Большие молекулы.
\bibitem{Teplukhin2009SuperBigMolec}\textbf{Теплухин~А.} Моделирование больших молекулярных агрегатов с использованием параллельных вычислений методом Монте-Карло. // В книге: Методы компьютерного моделирования для исследовния полимеров и биополимеров, под. ред. Иванов~В., Рабинович~А., Хохлов~А., М:. Либроком, 2009, Гл.~17, С.~553-570. DOI:~10.13140/RG.2.1.2050.6725
% COVID-19
\bibitem{Colonnelli2021SuperCovid}\textbf{Colonnelli~I., Cantalupo~B., Spampinato~C. et al.} Bringing AI pipelines onto cloud-HPC: setting a baseline for accuracy of COVID-19 AI diagnosis. // arXiv, 2021. DOI:~10.48550/arXiv.2108.01033
% Обработка снимков.
\bibitem{Ri2024SuperXRay}\textbf{Ri~S., Robert~D., Soren~P. et al.} Comparing the output of an artificial intelligence algorithm in detecting radiological signs of pulmonary tuberculosis in digital chest X-Rays and their smartphone-captured photos of X-Ray films: Retrospective study. // JMIR Formative Research, 2024, 8:e55641. DOI:~10.2196/55641
% Персонализированная медицина.
\bibitem{Kishore2024SuperPrecMed}\textbf{Kishore~B., Pinjala~S.} Transformation of evidence-based medicine to precision medicine by AI: A roadmap. // Journal of the American Association of Physicians of Indian Origin, 2024, 4(1,2), P.~14-36.

% Приоритет - г) переход к высокопродуктивному и экологически чистому агро- и аквахозяйству, разработку и внедрение систем рационального применения средств химической и биологической защиты сельскохозяйственных растений и животных, хранение и эффективную переработку сельскохозяйственной продукции.
% Растениеводство.
\bibitem{Ahmetshina2020SuperSelection}\textbf{Ахметшина~А., Стрыгина~К., Хлесткина~Е., Пороховинова~Е., Брач~Н.} Высокопроизводительное секвенирование в генетике и селекции льна. // Экологическая генетика, 2020, Т.~18, №~1, С.~103-124. DOI:~10.17816/ecogen16126
% Животноводство.
\bibitem{Mourant2018SuperEpi}\textbf{Mourant~J., Fenimore~P., Manore~C., McMahon~B.} Decision support for mitigation of livestock disease: Rinderpest as a case study. // Veterinary Epidemiology and Economics, 2018, Vol.~5. DOI:~10.3389/fvets.2018.00182
% Удобрения.
\bibitem{Irfan2016SuperFert}\textbf{Irfan~S., Razali~R., Ku~K., Mansor~N.} Modelling and simulations of controlled release fertilizer. // 4th International Conference on Fundamental and Applied Sciences Conference Proceedings, 2016, 1787, 080025. DOI:~10.1063/1.4968164
% Планирование сельхоз производства.
\bibitem{Zhang2021SuperFertPlan}\textbf{Zhang~Z., Yu~Z., Zhang~Y., Shi~Y.} Optimized nitrogen fertilizer application strategies under supplementary irrigation improved winter wheat (Triticum aestivum L.) yield and grain protein yield. // PeerJ, 2021, 9:e11467. DOI:~10.7717/peerj.11467

% Приоритет - д) противодействие техногенным, биогенным, социокультурным угрозам, терроризму и экстремистской идеологии, деструктивному иностранному информационно-психологическому воздействию, а также киберугрозам и иным источникам опасности для общества, экономики и государства, укрепление обороноспособности и национальной безопасности страны в условиях роста гибридных угроз
% Кибербезопасность.
\bibitem{Terziyska2024SuperCyber}\textbf{Terziyska~M., Terziyski~Z., Nikolova-Alexieva~V., Valeva~K.} Opportunities to achieve cyber security through blockchain and smart contracts. // Computer Science and Interdisciplinary Research Journal, 2024, Vol.~1, Issue~2. DOI:~10.70862/CSIR.2024.0102-09

% Приоритет - е) повышение уровня связанности территории Российской Федерации путем создания интеллектуальных транспортных, энергетических и телекоммуникационных систем, а также занятия и удержания лидерских позиций в создании международных транспортно-логистических систем, освоении и использовании космического и воздушного пространства, Мирового океана, Арктики и Антарктики.
% Авиасообщение.
\bibitem{Juntana2022SuperFlight}\textbf{Juntana~P., Delahaye~D., Chaimatanan~S., Alam~S.} Hyperheuristic approach based on reinforcement learning for air traffic complexity mitigation. // Journal of Aerospace Computing, Information and Communication, 2022, 19(1), P.~1-16. DOI:~10.2514/1.i011048
% Морские перевозки.
\bibitem{Yan2024SuperSea}\textbf{Yan~Q., Song~R., Kim~K.-H., Wang~Y., Feng~X.} Optimizing container relocation operations by using deep reinforcement learning. // Maritime Policy \& Management, 2024, P.~1-23. DOI:~10.1080/03088839.2024.2424865
% Глобальные сети.
\bibitem{Abramov2025SuperNets}\textbf{Abramov~A., Gonchar~A., Evseev~A., Shabanov~B.} Results of the development of National Research Computer Network of Russia within the framework of the National Project "Science and Universities" in 2021—2024. // Informational Technologies, 2025, Vol.~31, №~1. DOI:~10.17587/it.31.35-41

% Приоритет - ж) возможность эффективного ответа российского общества на большие вызовы с учетом возрастающей актуальности синтетических научных дисциплин, созданных на стыке психологии, социологии, политологии, истории и научных исследований, связанных с этическими аспектами научно-технологического развития, изменениями социальных, политических и экономических отношений.

% Приоритет - з) объективную оценку выбросов и поглощения климатически активных веществ, снижение их негативного воздействия на окружающую среду и климат, повышение возможности качественной адаптации экосистем, населения и отраслей экономики к климатическим изменениям.
% Климатические модели.
\bibitem{Kulkarni2024SuperClimate}\textbf{Kulkarni~P., Manoharan~S., Gaddi~A.} Advancing climate modeling through high-performance computing: Towards more accurate and efficient simulations. // EAI Endorsed Transactions on Energy Web, 2024. DOI:~10.4108/ew.7049
% Глобальная модель океана.
\bibitem{Wei2024SuperOcean}\textbf{Wei~J., Han~X., Yu~J., Jiang~J. et al.} A performance-portable kilometer-scale global ocean model on ORISE and new Sunway heterogeneous supercomputers. // International Conference for High Performance Computing, Networking, Storage and Analysis, Atlanta, US, 2024. DOI:~10.1109/SC41406.2024.00009
% Экосистемы.
\bibitem{Rahman2024SuperSpecies}\textbf{Rahman~A., Tikhonov~G., Oksanen~J., Rossi~T., Ovaskainen~O.} Accelerating joint species distribution modeling with Hmsc-HPC: A 1000x faster GPU deployment. // bioRxiv, 2024. DOI:~10.1101/2024.02.13.580046
% Модель загрязнения воздуха.
\bibitem{Ostromsky2024SuperAir}\textbf{Ostromsky~T.} Optimization, performance and scalability experiments of a large air pollution model by using the EuroHPC petascale supercomputer Discoverer. // Studies in Computational Intelligence, 2024. DOI:~10.1007/978-3-031-57320-0\_8

% Приоритет - и) переход к развитию природоподобных технологий, воспроизводящих системы и процессы живой природы в виде технических систем и технологических процессов, интегрированных в природную среду и естественный природный ресурсооборот.
% Нейроморфные процессеры.
\bibitem{Rhodes2019SuperNuero}\textbf{Rhodes~O., Peres~L., Rowley~A., Gait~A., Plana~L., Brenninkmeijer~C., Furber~S.} Real-time cortical simulation on neuromorphic hardware. // Philosophical Transactions A, 2019, 378:20190160. DOI:~10.1098/rsta.2019.0160
% Органы и ткани человека.
\bibitem{Wang2012SuperTissues}\textbf{Wang~Y., Eddy~J., Price~N.} Reconstruction of genome-scale metabolic models for 126 human tissues using mCADRE. // BMC Systems Biology, 2012, №~6, 153.
% Топологическая оптимизация.
\bibitem{Fedchikov2024SuperBim}\textbf{Fedchikov~V.} Topological optimization methods in the design of metal structures of buildings. // E3S Web of Conferences, 2024, 533, 02023. DOI:~10.1051/e3sconf/202453302023

% Суррогатное моделирование.
\bibitem{Jiang2020Surrogate}\textbf{Jiang~P., Zhou~Q., Shao~X.} Surrogate model-based engineering design and optimization. // 2020, Springer, 246~P.
\bibitem{Barcenas2023Surrogate}\textbf{Barcenas~O., Pioquinto~J., Kurkina~E., Lukyanov~O.} Surrogate aerodynamic wing modeling based on a multilayer perceptron. // Aerospace, 2023, Vol.~10, 149, 19~P. DOI:~10.3390/aerospace10020149
\bibitem{Catalani2024Surrogate}\textbf{Catalani~G., Agarwal~S., Bertrand~X., Tost~F., Bauerheim~M., Morlier~J.} Aero-Nef: neural fields for rapid aircraft aerodynamics simulations. // arXiv, 2024, DOI:~10.48550/arXiv.2407.19916

% Суперкомпьютеры в России.
\bibitem{Voevodin2021SuperRussia}\textbf{Voevodin~V., Chulkevich~R., Kostenetskiy~P. et al} Administration, monitoring and analysis of supercomputers in Russia: a survey of 10 HPC centers. // Supercomputing Frontiers and Innovations, 2021, Vol.~8, No.~3, P.~82-103. DOI:~10.14529/jsfi210305
% Выбор суперкомпьютеров.
%\bibitem{Maleki2023SuperChoice}\textbf{Maleki~D., Mansouri~A., Arianyan~E.} An integrated taxonomy of standard indicators for ranking and selecting supercomputers. // IET Computers \& Digital Techniques, 2023, 17(3-4), P.~162-179. DOI:~10.1049/cdt2.12061

% Основные задачи в пространстве и на поверхности.
\bibitem{Lobanova2023GeneralGas}\textbf{Лобанова~Е.~Е., Тарасов~Н.~И.} Применение цифровой платформы для моделирования задач газовой динамики. // Препринты ИПМ им. М.~В.~Келдыша, 2023, №~77, 20~С. DOI:~10.20948/prepr-2023-77
\bibitem{Taboada2013GeneralElectro}\textbf{Taboada~J., Ara{\'u}jo~M., Basteiro~F., Rodr{\'i}guez~J., Landesa~L.} MLFMA-FFT parallel algorithm for the solution of extremely large problems in electromagnetics. // Proceedings of the IEEE, 2013, Vol.~101, No.~2, P.~350-363. DOI:~10.1109/JPROC.2012.2194269
\bibitem{Liu2020GeneralThermo}\textbf{Liu~Z.-K.} Computational thermodynamics and its applications. // Acta Materialia, 2020, Vol.~200, P.~745-792. DOI:~10.1016/j.actamat.2020.08.008
\bibitem{Kudryavzeva2014GeneralVolumeMesh}\textbf{Кудрявцева~Л.~Н.} Методы самоорганизации и оптимизации для построения трехмерных расчетных сеток. // Диссертация на соискание ученой степени кандидата физико-математических наук, 2014, Москва.
\bibitem{Zheleznyakova2016GeneralSurfMesh}\textbf{Железнякова~А.~Л.} Унифицированный подход к созданию сложных виртуальных поверхностей и расчетных сеток для комплексного имитационного 3D моделирования современных изделий аэрокосмической техники. // Физико-химическая кинетика в газовой динамике, 2016, Т.~17(2). URL:~http://chemphys.edu.ru/issues/2016-17-2/articles/634 (дата обращения \StrDate)

% Задачи на поверхностях.
\bibitem{Mitin2020Flow}\textbf{Митин~А., Калашников~C., Янковский~Е., Аксенов~А., Жлуктов~С., Чернышев~С.} Методические аспекты численного решения задач внешнего обтекания на локально-адаптивных сетках с использованием пристеночных функций. // Компьютерные исследования и моделирование, 2020, Т.~12, №~6, С.~1269-1290. DOI:~10.20537/2076-7633-2020-12-6-1269-1290
\bibitem{Li2014Film}\textbf{Li~Y., Jeong~D., Kim~J.} Adaptive mesh refinement for simulation of thin film flows. // Meccanica, 2014, Vol.~49, P.~239-252. DOI:~ 10.1007/s11012-013-9788-6
\bibitem{Koshelev2020Ice}\textbf{Кошелев~К., Мельникова~В., Стрижак~С.} Разработка решателя iceFoam для моделирования процесса обледенения. // Труды ИСП РАН, 2020, Т.~32, Вып.~4, С.~217-234. DOI:~10.15514/ISPRAS-2020-32(4)-16
\bibitem{Sorokin2020Ice}\textbf{Сорокин~К., Бывальцев~П., Аксенов~А., Жлуктов~С., Савицкий~Д., Бабулин~А., Шевяков~В.} Численное моделирование обледенения в программном комплексе FlowVision. // Компьютерные исследования и моделирование, 2020, Т.~2020, №~1, С.~83-96. DOI:~10.20537/2076-7633-2020-12-1-83-96
\bibitem{Cui2023Impingement}\textbf{Cui~X., Habashi~W.} A dendritic freezing model for in-flight supercooled large droplets impingement and solidification. // Computers \& Fluids, 2023, Vol.~254(2), 105778. DOI:~10.1016/j.compfluid.2023.105778
\bibitem{Cui2021Impingement}\textbf{Cui~X., Habashi~W.} SPH simulation of supercooled large droplets impacting hydrophobic and superhydrophobic surfaces. // Computers \& Fluids, 2021. DOI:~10.1016/j.compfluid.2021.105055
\bibitem{Alexeenko2013Ice}\textbf{Алексеенко~С., Приходько~А.} Численное моделирование обледенения цилиндра и профиля. Обзор моделей и результатов расчетов. // Ученые записки ЦАГИ, 2013, Т.~XLIV, №~6, С.25-57.
\bibitem{Domingos2015IceHeat}\textbf{Domingos~R., Silva~D.} 3D computational methodology for bleed air ice protection system parametric analysis. // SAE Technical Paper 2015-01-2109, 2015. DOI:~10.4271/2015-01-2109
\bibitem{Xin2013Ice}\textbf{Li~X., Bai~J., Hua~J., Wang~K., Zhang~Y.} A spongy icing model for aircraft icing. // Chineese Journal of Aeronautics, 2014, Vol.~27(1), P.~40-51. DOI:~10.1016/j.cja.2013.12.004

% Безопасность полетов.
\bibitem{Raj2020IntroIce}\textbf{Raj~L., Yee~K., Myong~R.} Sensivity of ice and aerodynamic performance degradation to critical physical and modeling parameters affecting airfoil icing. // Aerospace Science and Technology, 2020, 98, 105659, P.~1–46. DOI:~10.1016/j.ast.2019.105659

% Программные пакеты для льда.
\bibitem{Martini2022IntroIce}\textbf{Martini~F., Ibrahim~H., Montoya~L., Rizk~P., Ilinca~A.} Turbulence modeling of iced wind turbine airfoils. // Energies, 2022, 15, 8325, P.~1–21. DOI:~10.3390/en15228325
\bibitem{Shannon2019IntroIce}\textbf{Shannon~T., McClain~S.} As assessment of LEWICE roughness and convection enhancement models. // SAE International Journal of Advances and Current Practices in Mobility, 2(1). DOI:~10.4271/2019-01-1977
\bibitem{Villedieu2014IntroIce}\textbf{Villedieu~P., Trontin~P., Chauvin~R.} Glaciated and mixed phase ice accretion modeling using ONERA 2D icing suite. // Transactions of Japanese Society for Medical and Biological Engineering, 51. DOI:~10.2514/6.2014-2199
\bibitem{Son2010IntroIce}\textbf{Son~C.-K., Oh~S.-J., Yee~K.} Prediction of glaze ice accretion on 2D airfoil. // Journal of the Korean Society for Aeronautical \& Space Sciences, 2010, 38(8), P.~747-757. DOI:~10.5139/JKSAS.2010.38.8.747
\bibitem{Galanov2021IntroIce}\textbf{Galanov~N.~G., Sarazov~A.~V., Zhuchkov~R.~N., Kozelkov~A.~S.} Application of various ice accretion simulation approaches in the LOGOS software package. // Journal of Physics: Conference Series, 2021, 2099, 012029, P.~1–8. DOI:~10.1088/1742-6596/2099/1/012029
\bibitem{Galanov2021IntoIceLOGOS}\textbf{Галанов~Н.Г., Козелков~А.~С., Жучков~Р.~Н., Саразов~А.~В.} Тестирование методики моделирования процесса обледенения в пакете программ ЛОГОС. // Информационные системы и технологии ИСТ-2021, сборник материалов XXVII Международной научно-технической конференции, 2021, C.~848-854.

% Модели льда.
\bibitem{Bartkus2018IntroIce}\textbf{Bartkus~T., Struk~P., Tsao~J.-C.} Evaluation of a thermodynamic ice-crystal icing model using experimental ice accretion data. // in Proceedings of the Atmospheric and Space Environments Conference, 2018, P.~1-18. DOI:~10.2514/6.2018-4129
\bibitem{Zhang2017IntroIce}\textbf{Zhang~X., Wu~X., Min~J.} Aircraft icing model considering both rime ice property variability and runback water effect. // International Journal of Heat and Mass Transfer, 2017, 104, P.~510–516. DOI:~10.1016/j.ijheatmasstransfer.2016.08.086
\bibitem{Pena2016IntroIce}\textbf{Pena~D., Hoarau~Y., Laurendeau~E.} A single step ice accretion model using level-set method. // Journal of Fluids and Structures, 2016, 65, P.~278–294. DOI:~10.1016/j.jfluidstructs.2016.06.001
\bibitem{Wang2023IntroIce}\textbf{Wang~Z., Zhong~W., Liu~C., Zhao~H., Liu~S.} Numerical simulation of ice crystal supercooled droplet mixed phase icing based on the improved Messinger model. // International Journal of Aerospace Engineering, 2023, 1776, P.~1–12. DOI:~10.1155/2023/6696084
\bibitem{Liu2022IntroIce}\textbf{Liu~Y., Wang~T., Song~Z., Chen~M.} Spreading and freezing of supercooled water droplets impacting an ice surface. // Applied Surface Science, 2022, 583(1): 152374. DOI:~10.1016/j.apsusc.2021.152374
\bibitem{Ruan2023IntroIce}\textbf{Ruan~Q.} Analysis of the types of aircraft icing and the solutions. // Applied and Computational Engineering, 2023, 9(1), P.~177–181. DOI:~10.54254/2755-2721/9/20230082

% Свидетельства программ для ЭВМ.
\bibitem{CertGoryachev2023Crys}Свидетельство о государственной регистрации программы для ЭВМ №~2023666962 <<Программный модуль компьютерного моделирования процесса обледенения элементов авиационных силовых установок (<<Кристалл 2023>>)>>. Авторы: \textbf{Горячев~А., Горячев~П., Рыбаков~А.} Дата регистрации: 08.08.2023
\bibitem{CertGoryachev2020Crys}Свидетельство о государственной регистрации программы для ЭВМ №~2020615575 <<Программный модуль расчета процесса обледенения элементов авиационных силовых установок (КРИСТАЛЛ)>>. Авторы: \textbf{Горячев~А., Горячев~П., Рыбаков~А.} Дата регистрации: 26.05.2020
\bibitem{CertRybakov2021PrepUnstruct}Свидетельство о государственной регистрации программы для ЭВМ №~2021660227 <<Программа подготовки неструктурированной поверхностной расчетной сетки для проведения расчетов на суперкомпьютере (crys-gsu)>>. Авторы: \textbf{Рыбаков~А., Фрейлехман~С., Шумилин~С.} Дата регистрации: 23.06.2021
\bibitem{CertRybakov2020PrepStruct}Свидетельство о государственной регистрации программы для ЭВМ №~2020618596 <<Программа подготовки блочно-структурированной расчетной сетки для проведения расчетов на суперкомпьютере (GridMaster)>>. Авторы: \textbf{Рыбаков~А., Чопорняк~А.} Дата регистрации: 30.07.2020
\bibitem{CertRybakov2024Surf}Свидетельство о государственной регистрации программы для ЭВМ №~2024689396 <<Программа организации вычислений на поверхностной сетке>>. Авторы: \textbf{Рыбаков~А.} Дата регистрации: 05.12.2024
\bibitem{CertRybakov2023Mon}Свидетельство о государственной регистрации программы для ЭВМ №~2023684174 <<Программа эмуляции и мониторинга эффективности векторного кода плоских циклов>>. Авторы: \textbf{Рыбаков~А.} Дата регистрации: 14.11.2023
\bibitem{CertRybakov2019AVX}Свидетельство о государственной регистрации программы для ЭВМ №~2019665254 <<Векторизованная с помощью набора инструкций AVX-512 программа реализации решения задачи Римана о распаде произвольного разрыва>>. Авторы: \textbf{Рыбаков~А., Шумилин~С.} Дата регистрации: 21.11.2019

%---------------------------------------------------------------------------------------------------
% Глава 1. - методы перестроения поверхностной расчетной сетки в двумерном случае.

% Перестроение по фиксированным направлениям.
\bibitem{Fortin2004Remesh2d}\textbf{Fortin~G., Ilinca~A., Laforte~J.-L., Brandi~V.} New roughness computation method and geometric accretion model for airfoil icing. // Journal of Aircraft, 20024, Vol.~41, P.~119–127. DOI:~10.2514/1.173
% Решение в общем виде для двумерного случая.
\bibitem{Rybakov2019Geo2D}\textbf{Rybakov~A., Shumilin~S.} Approximate methods of the surface mesh deformation in two-dimensional cases. // Lobachevskii Journal of Mathematics, 2019, Vol.~40, No.~11, P.~1848-1852. DOI:~10.1134/S1995080219110258
% Метод градиентного спуска.
\bibitem{Kantorovich1984Func}\textbf{Канторович~Л., Акилов~Г.} Функциональный анализ. // Москва <<Наука>>, 1984, 752~С.

%---------------------------------------------------------------------------------------------------
% Глава 2. - методы перестроения поверхностной расчетной сетки в трехмерном случае.

% Перестроение в трехмерном случае.
\bibitem{Meshcheryakov2023GeoEvo}\textbf{Meshcheryakov~A., Rybakov~A.} Evolution of the surface computational mesh in the ice accretion process. // Lobachevskii Journal of Mathematics, 2023, Vol.~44, No.~11, P.~361-378. DOI:~10.1134/S1995080223110367
% Конечно-объемный метод расчета льда.
\bibitem{Beaugendre2003Ice}\textbf{Beaugendre~H.}A PDE-based approach to in-flight ice accretion. // PhD Thesis (Dep. of Mech. Eng., McGill Univ., Montreal, Qu{\'e}bec, 2003).
% Многослойное перестроение.
\bibitem{BourgaultCote2017}\textbf{Bourgault-Côté~S., Hasanzadeh~K., Lavoie~P., Laurendeau~E.} Multi-layer methodologies for conservative ice growth. // 7th European Conference for Aeronautics and Aerospace Sciences (EUCASS), 2017. DOI: 10.13009/EUCASS2017-258
% Метод Тонга.
\bibitem{Thompson2013Remesh}\textbf{Thompson~D., Tong~X., Arnoldus~Q., Collins~E., McLaurin~D., Luke~E.} Discrete surface evolution and mesh deformation for aircraft icing applications. // in Proceedings of the 5th AIAA Atmospheric and Space Environments Conference, 2013, P.~1–20. DOI:~10.2514/6.2013-2544
\bibitem{Tong2017Remesh}\textbf{Tong~X., Thompson~D., Arnoldus~Q., Collins~E., Luke~E.} Three-dimensional surface evolution and mesh deformation for aircraft icing applications. // Journal of Aircraft, 2017, Vol.~54, P.~1047–1063. DOI:~10.2514/1.C033949
\bibitem{Jiao2007Offsetting}\textbf{Jiao~X.} Face offsetting: A unified approach for explicit moving interfaces. // Journal of Computational Physics, 2007, Vol.~220. P.~612–625. DOI:~10.1016/j.jcp.2006.05.021
\bibitem{Jiao2006Smooth}\textbf{Jiao~X.} Volume and feature preservation in surface mesh optimization. // in Proceedings of the 15th International Meshing Roundtable, 2006, P.~359–373. DOI:~10.1007/978-3-540-34958-7\_21
\bibitem{Shumilin2021Smooth}\textbf{Шумилин~С.~С.} Методы закрепления граничных узлов при сглаживании треугольной поверхностной сетки. // Программные системы: Теория и приложения, 2021, Т.~12, №~2(49), С.~193-206. DOI:~10.25209/2079-3316-2021-12-2-193-206

% Перестроение с помощью общей огибающей
\bibitem{Rybakov2023GeoRemesh}\textbf{Рыбаков~А.} Геометрическое перестроение расчетной сетки с помощью общей огибающей семейства сфер в задаче ледообразования. // Современные информационные технологии и ИТ-образование, 2023, Т.~19, №~2, С.~282-291. DOI:~10.25559/SITITO.019.202302.282-291
\bibitem{Rybakov2017Flight}\textbf{Рыбаков~А.} Оптимизация задачи об определении конфликтов с опасными зонами движения летательных аппаратов для выполнения на Intel Xeon Phi. // Программные продукты и системы, 2017, Т.~30, №~3, С.~524-528. DOI:~10.15827/0236-235X.119.3.524-528

% Репозиторий 3D моделей.
\bibitem{StanfordModels} The Stanford 3D Scanning Repository, \\ URL:~https://graphics.stanford.edu/data/3Dscanrep/ (дата обращения \StrDate)

%---------------------------------------------------------------------------------------------------
% Глава 3. - пересечение расчетных сеток.

% Самопересечения.
\bibitem{Freylekhman2022GeoIntersect}\textbf{Freylekhman~S., Rybakov~A.} Self-intersection elimination for unstructured surface computational meshes. // Lobachevskii Journal of Mathematics, 2022, Vol.~43, No.~10, P.~2846-2852. DOI:~10.1134/S1995080222130133
\bibitem{Rivara2019Delaunay}\textbf{Rivara~M.-C., Rodrigez-Moreno~P.} Tuned terminal triangles centroid Delaunay algorithm for quality triangulation. // in Proceedings of the 27th International Meshing Roundtable, 2019, P.~211–228. DOI:~10.1007/978-3-030-13992-6\_12
\bibitem{Rakotoarivelo2019Remesh}\textbf{Rakotoarivelo~H., Ledoux~F.} Accurate manycore-accelerated manifold surface remesh kernels. // in Proceedings of the 27th International Meshing Roundtable, 2019, P.~405–423. DOI:~10.1007/978-3-030-13992-6\_22
\bibitem{Borouchaki2000Remesh}\textbf{Borouchaki~H., Laug~P., George~P.-L.} Parametric surface meshing using a combined advancing-front generalized Delaunay approach. // International Journal of Numeric Methods in Enineering., 2000, Vol.~49, P.~233–259. DOI:~10.1002/1097-0207(20000910/20)49:1/23.0.CO;2-G
\bibitem{Panchal2022Tri}\textbf{Panchal~D., Jayaswal~D.} Feature sensitive geometrically faithful highly regular direct triangular isotropic surface remeshing. // Sadhana, 2022, Vol.~47(94), P.~1–19. DOI:~10.1007/s12046-022-01866-7
\bibitem{Jung2004Int}\textbf{Jung~W., Shin~H., Choi~B.} Self-intersection removal in triangular mesh offsettings. // CAD Journal, 2004, No.~1, P.~477–484. DOI:~10.1080/16864360.2004.10738290
\bibitem{Charton2021Repair}\textbf{Charton~J., Baek~S., Kim~Y.} Mesh repairing using topology graphs. // Journal of Computational Design and Engineering, 2021, Vol.~8, P.~251–267. DOI:~10.1093/jcde/qwaa076
\bibitem{Skorkovska2018Int}\textbf{Skvorkovska~V., Kolingerov~I., Benes~B.} A Simple and robust approach to computation of meshes intersection. // in Proceedings of the 13th International Joint Conference on Computer Vision, Imaging and Computer Graphics Theory and Applications, 2018, Vol.~1, P.~175–182. DOI:~10.5220/0006538401750182

% Сопряжение.
\bibitem{Blazek2015CFD}\textbf{Blazek~J.} Computational fluid dynamics. Principles and applications. // Elsevier, 2015, 451~P.
\bibitem{Kulikovsky2001Gas}\textbf{Куликовский~А., Погорелов~Н., Семенов~А.} Математические вопросы численного решения гиперболических систем уравнений. // М.: Физматлит, 2001, 608~С.
\bibitem{Smirnova2018Euler}\textbf{Смирнова~Н.} Сравнение схем с расщеплением потока для численного решения уравнения Эйлера сжимаемого газа. // Труды МФТИ, 2018, Т.~10, №~1, С.~122-141.
\bibitem{Borisov2018Riemann}\textbf{Борисов~В., Рыков~Ю.} Точный римановский солвер в алгоритмах решения задач многокомпонентной газовой динамики. // Препринты ИПМ им. М.~В.~Келдыша, 2018, №~96, 28~С. DOI:~10.20948/prepr-2018-96
\bibitem{Toro1999Riemann}\textbf{Toro~E.} Riemann solvers and numerical methods for fluid dynamics: A practical introduction. // Springer, Berlin–Heidelberg, 1999, 645~P.
\bibitem{riemannvecGithub}Vectorized Riemann Solver. // репозиторий github.com. https://github.com/r-aax/riemann\_vec (дата обращения \StrDate)
\bibitem{Bendersky2014RANSILES}\textbf{Бендерский~Л., Любимов~Д.} Применение RANS/ILES метода высокого разрешения для исследования сложных турбулентных струй. // Ученые записки ЦАГИ, 2024, Т.~XLV, №~2, С.~22-36. 
\bibitem{Chernikov1963}\textbf{Черников~С.} Свертывание конечных систем линейных неравенств. // Доклады АН СССР, 1963, Т. 152, № 5, С. 1075-1078.

% Метод погруженной границы.
\bibitem{Mahesh2003Turbulent}\textbf{Mahesh~K., Constantinescu~G., Moin~P.} Simulating turbulent flows in complex geometries. // Proceedings of FEDSM2003 2003 4th ASME JSME Joint Fluids Engineering Conference, 2003. DOI:~10.1115/FEDSM2003-45337
\bibitem{Ye2020Grid}\textbf{Ye~H., Liu~Y., Chen~B., Liu~Z., Zheng~J., Pang~Y., Chen~J.} Hybrid grid generation for viscous flow simulation in complex geometries. // 2020. DOI:~10.21203/rs.3.rs-31698/v1
\bibitem{Wright2015LEWICE}\textbf{Wright~W., Struk~P., Bartkus~T., Addy~G.} Recent advances in the LEWICE icing model. // SAE Technical Paper, 2015. DOI:~10.4271/2015-01-2094
\bibitem{Abalakin2018Immersed}\textbf{Абалакин~И., Жданова~Н., Козубская~Т.} Метод погруженных границ для численного моделирования невязких сжимаемых течений. // Журнал вычислительной математики и математической физики, 2018, Т.~58, №~9, С.~1462-1471. DOI:~10.31857/S004446690002525-8
\bibitem{Mori2008Immersed}\textbf{Mori~Y., Peskin~C.} Implicit second-order immersed boundary methods with boundary mass. // Comput. Methods Appl. Mech. Engrg, 2008, Vol.~197, P.~2049-2067. DOI:~10.1016/J.CMA.2007.05.028
\bibitem{Kim2004Immersed}\textbf{Kim~J., Choi~H.} An immersed-boundary finite-volume method for simulation of heat transfer in complex geometries. // KSME International Journal, 2004, Vol.~18, №~6, P.~1026-1035. DOI:~10.1007/BF02990875
\bibitem{Clarke1996Cartesian}\textbf{Clarke~D., Salas~M., Hassan~H.} Euler calculations for multielement airfoils using cartesian grids. // AIAA Journal, 1986, Vol.~24, №~3, P.~353-358. DOI:~10.2514/3.9273
\bibitem{Farrashkhalvat2003Grid}\textbf{Farrashkhalvat~M., Miles~J.} Basic structured grid generation with an introduction to unstructured grid generation. // 2003, Butterworth Heinemann, 231~P.
\bibitem{Rybakov2017Mesh}\textbf{Рыбаков~А.} Внутреннее представление и механизм межпроцессного обмена для блочно-структурированной сетки при выполнении расчетов на суперкомпьютере. // Программные системы: теория и приложения, 2017, Т.~8, Вып.~1, С.~121-134. DOI:~10.25209/2079-3316-2017-8-1-121-134
\bibitem{Savin2019RANSILES}\textbf{Savin.~G., Benderskiy~L., Lyubimov~D., Rybakov~A.} RANS/ILES method optimization for effective calculations on supercomputer. // Lobachevskii Journal of Mathematics, 2019, Vol.~40, No.~5, P.~566-573. DOI:~10.1134/S1995080219050172
\bibitem{Giordano2019Load}\textbf{Giordano~A., De Rango~A., Rongo~R., D'Ambrosio~D., Spataro~W.} A dynamic load balancing technique for parallel execution of structured grid models. // 2020, In: Sergeyev Y., Kvasov D. (eds) Numerical Computations: Theory and Algorithms. NUMTA 2019. Lecture Notes in Computer Science, Vol.~11973, Springer. DOI:~10.1007/978-3-030-39081-5\_25
\bibitem{Fadlun2000Immersed}\textbf{Fadlun~E., Verzicco~R., Orlandi~P., Mohd-Yusof~J.} Combined immersed-boundary finite-difference methods for three-dimensional complex flow simulations. // Journal of Computational Physics, 2000, Vol.~161, P.~35-60. DOI:~10.1006/jcph.2000.6484
\bibitem{Mittal2005Immersed}\textbf{Mittal~R., Iaccarino~G.} Immersed boundary methods. // Annual. Rev. Fluid Mech., 2005, Vol.~37, P.~239-261. DOI:~10.1146/annurev.fluid.37.061903.175743
\bibitem{Tseng2003Immersed}\textbf{Tseng~Y.-H., Ferziger~J.} A ghost-cell immersed boundary method for flow in complex geometry. // Journal of Computational Physics, 2003, Vol.~192, P.~593-623. DOI:~10.1016/j.jcp.2003.07.024
\bibitem{Peter2016Immersed}\textbf{Peter~S., De~A.} A parallel implementation of the ghost-cell immersed boundary method with application to stationary and moving boundary problems. // Sadhana, 2016, Vol.~41, №~4, P.~441-450. DOI:~10.1007/s12046-016-0484-9
\bibitem{Rybakov2020GeoIBM}\textbf{Рыбаков~А.} Метод погруженной границы с использованием фиктивных ячеек в трехмерной постановке. // Современные информационные технологии и ИТ-образование, 2020, Т.~16, №~2, С.~321-330. DOI:~10.25559/SITITO.16.202002.321-330
\bibitem{Vinnikov2007Immersed}\textbf{Винников~В., Ревизников~Д.} Метод погруженной границы для расчета сверхзвукового обтекания затупленных тел на прямоугольных сетках. // Электронный журнал «Труды МАИ», 2007, №~27, 13~С.
\bibitem{Wackers2011AMR}\textbf{Wackers~J., Deng~G., Leroyer~A., Queutey~P., Visonneau~M.} Adaptive grid refinement for hydrodynamic flows. // Computers \& Fluids, 2011, Vol.~55, P.~85-100. DOI:~10.1016/j.compfluid.2011.11.004
\bibitem{Zhou2014Refinement}\textbf{Zhou~L., Yunjun~Y., Anlong~G., Weijiang~Z.} Unstructured adaptive grid refinement for flow feature capture. // Procedia Engineering, 2014, Vol.~99, P.~477-483. DOI:~10.1016/j.proeng.2014.12.561
\bibitem{Plas2015Refinement}\textbf{van der Plas~P., Veldman~A., van der Heiden~H., Luppes~R.} Adaptive grid refinement for free-surface flow simulations in offshore applications. // Proceedings of the ASME 2015 34th International Conference on Ocean, Offshore and Arctic Engineering OMA2015, 2015. DOI:~10.1115/OMAE2015-42029

%===================================================================================================

% Глава 4 - распараллеливание

% Распределение блоков для гетерогенного кластера.
\bibitem{Rybakov2018Distr}\textbf{Рыбаков~А.} Распределение вычислительной нагрузки между узлами гетерогенного вычислительного кластера. // Электронный научный журнал: Программные продукты, системы и алгоритмы, 2018, №~1, С.~26-32. DOI:~10.15827/2311-6749.26.300
\bibitem{Rybakov2017Part}\textbf{Рыбаков~А.} Партицирование графа смежности блочно-структурированной сетки. // Современные информационные технологии и ИТ-образование, 2017, Т.~13, №~1, С.~43-48. DOI:~10.25559/SITITO.2017.1.422

% Распределение блоков с дроблением.
\bibitem{Rybakov2016WithCut}\textbf{Рыбаков~А.} Распределение вычислительной нагрузки между узлами суперкомпьютерного кластера при расчетах задач газовой динамики с дроблением расчетной сетки. // Современные информационные технологии и ИТ-образование, 2016, Т.~12, №~2, С.~101-107.
\bibitem{Romanovskii1977Extreme}\textbf{Романовский~И.~В.} Алгоритмы решения экстремальных задач. // Издательство <<Наука>>, Москва, 1977.
\bibitem{Savin2019RANS}\textbf{Savin~G., Benderskiy~L., Lyubimov~D., Rybakov~A.} RANS/ILES method optimization for effective calculations on supercomputer. // Lobachevskii Journal of Mathematics, 2019, Vol.~40, No.~5, P.~566-573. DOI:~10.1134/S1995080219050172
\bibitem{Bendersky2017Eff}\textbf{Бендерский~Л., Любимов~Д., Рыбаков~А.} Анализ эффективности масштабирования при расчетах высокоскоростных турбулентных течений на суперкомпьютере RANS/ILES методом высокого разрешения. // Труды НИИСИ РАН, 2017, Т.~7, №~4, С.~32-40. DOI:~10.25682/NIISI.2018.4.9975
\bibitem{Bendersky2018Block}\textbf{Бендерский~Л., Любимов~Д., Рыбаков~А.} Инструментарий подготовки блочно-структурированной сетки для проведения расчетов методом RANS/ILES. // Труды НИИСИ РАН, 2018, Т.~8, №~4, С.~102-106. DOI:~10.25682/NIISI.2018.4.0012

% Декомпозиция поверхностной сетки (общие).
\bibitem{Rybakov2020Decomp}\textbf{Рыбаков~А., Чопорняк~А.} Декомпозиция поверхностной неструктурированной расчетной сетки для масштабирования вычислений на суперкомпьютере. // Современные информационные технологии и ИТ-образование, 2020, Т.~16, №~4, С.~851-861. DOI:~10.25559/SITITO.16.202004.851-86
\bibitem{Naca0012}NACA-0012. URL:~https://orangeflowcfd.com/naca-0012 (дата обращения \StrDate)
\bibitem{Farhat1988Decomp}\textbf{Farhat~C.} A simple and efficient automatic fem domain decomposer. // Computers \& Structures, 1988, Vol.~28(5), P.~579-602. DOI:~10.1016/0045-7949(88)90004-1
\bibitem{Preis1997Decomp}\textbf{Preis~R., Diekmann~R.} PARTY -- a software library for graph partitioning. // In: Topping B.H.V. (ed.) Advances in Computational Mechanics for Parallel and Distributed Processing. Civil-Comp Press, Edinburgh, UK, 1997, P.~63-71. DOI:~10.4203/ccp.45.3.1 
\bibitem{Yakobovsky2005Decomp}\textbf{Якобовский~М.} Инкрементальный алгоритм декомпозиции графов. // Вестник Нижегородского университета им. Н.~И.~Лобачевского. Серия «Математическое моделирование и оптимальное управление», 2005, №~1(28), С.~243-250.
\bibitem{Urschel2014Decomp}\textbf{Urschel~J., Zikatanov~ L.} Spectral bisection of graphs and connectedness. // Linear Algebra and its Applications, 2014, Vol.~449, P.~1–16. DOI:~10.1016/j.laa.2014.02.007
\bibitem{Zhao2019Decomp}\textbf{Zhao~L., Liu~Y., Zhang~C. et al.} Automatic optimal block decomposition for structured mesh generation using genetic algorithm. // Journal of the Brazil Society of Mechanical Sciences and Engineering, 2019, Vol.~41. DOI:~10.1007/s40430-018-1510-0
\bibitem{Kapyris1998Decomp}\textbf{Kapyris~G., Kumar~V.} A fast and high quality multilevel scheme for partitioning irregular graphs. // SIAM Journal on Scientific Computing, 1998, Vol.~20, Issue~1, P.~359-392. DOI:~10.1137/S1064827595287997
\bibitem{Golovchenko2020Decomp}\textbf{Головченко~Е.} Обзор алгоритмов декомпозиции графов. // Препринты ИПМ им. М.~В.~Келдыша, 2020, №~2, С.~1-38. DOI:~10.20948/prepr-2020-2
\bibitem{Zheleznyakova2017Decomp}\textbf{Железнякова~А.} Эффективные методы декомпозиции неструктурированных адаптивных сеток для высокопроизводительных расчетов при решении задач вычислительной аэродинамики. // Физико-химическая кинетика в газовой динамике, 2017, Т.~18, №~1, С.~1-20.

% Генетический алгоритм декомпозиции.
\bibitem{Chahar2021Gen}\textbf{Chahar~V., Katoch~S., Chauhan~S.} A review on genetic algorithm: past, present and future. // Multimedia Tools and Applications, 2021, Vol.~80, №~4. DOI:~10.1007/s11042-020-10139-6
\bibitem{Wirayanti2025Gen}\textbf{Wirayanti~N., Sriwindono~H.} Implementation of hybrid genetic algorithm for solving the teacher placement problem. // Social Science and Humanities Journal, 2025, Vol.~9, Issue~1, P.~6341-6347. DOI:~10.18535/sshj.v9i01.1460
\bibitem{Naaman2025Gen}\textbf{Naaman~D., Ahmed~B., Ibrahim~I.} Optimization by nature: a review of genetic algorithm techniques. // The Indonesian Journal of Computer Science, 2025, Vol.~14, Issue~1, P.~268-284. DOI:~10.33022/ijcs.v14i1.4596
\bibitem{Charilogis2024Gen}\textbf{Charilogis~V., Tsoulos~I.} Introducing a parallel genetic algorithm for global optimization problems. // AppliedMath, 2024, Vol.~4, P.~709-730. DOI:~10.3390/appliedmath4020038
\bibitem{Dawei2025Gen}\textbf{Dawei~Q.} Using genetic algorithm to optimize the training plan and game strategy of basketball players. // Scalable Computing: Practice and Experience, 2025, Vol.~26, Issue~1, P.~441-449. DOI:~10.12694/scpe.v26i1.3845
\bibitem{Fang2025Gen}\textbf{Fang~Q.} Optimization of multi-objective crop planning strategies based on genetic algorithms. // Highlights in Science, Engineering and Technology, 2025, Vol.~133, P.~68-75. DOI:~10.54097/7qv57m30.
\bibitem{Mahmood2024Gen}\textbf{Mahmood~H., Al-Fatlawi~O., Al-Janabi~M., Sadeq~D., Al-Jumaah~Y., Khan~M.} Optimizing well placement with genetic algorithms: a case study. // Journal of Petroleum Research and Studies, 2024, Vol.~14, №~4, P.~37-51. DOI:~10.52716/jprs.v14i4.895
\bibitem{Kangra2024Gen}\textbf{Kangra~K., Singh~J.} A genetic algorithm-based feature selection approach for diabetes prediction. // IAES International Journal of Artificial Intelligence (IJ-AI), 2024, Vol.~13, №~2, P.~1489-1498. DOI: 10.11591/ijai.v13.i2.pp1489-1498
\bibitem{Sangeetha2025Gen}\textbf{Sangeetha~V., Vaneeta~M., Mamatha~A., Shoba~M., Deepa~S., Sujatha~V., Sujatha~S.} Breast cancer prediction using genetic algorithm and sand cat swarm optimization algorithm. // Indinesian Journal of Electrical Engineering and Computer Science, 2025, Vol.~37, №~2, P.~849-858. DOI:~10.11591/ijeecs.v37.i2.pp849-858
\bibitem{Kralev2024Gen}\textbf{Kralev~V., Kraleva~R.} Combining genetic algorithm with local search method in solving optimization problems. // Electronics, 2024, Vol.~13, Issue~4126. DOI:~10.3390/electronics13204126
\bibitem{Bogdanov2023Gen}\textbf{Богданов~М, Рудинский~И.} Использование генетического алгоритма для решения задачи поиска оптимального пути. // Вестник науки и образования Северо-Запада России, 2023, Т.~9, №~2, С.~76-88.
\bibitem{Malhotra2024Graph}\textbf{Malhotra~K., Vasa~K., Chaudhary~N., Vishnoi~A., Sapra~V.} A solution to graph coloring problem using genetic algorithm. // EAI Endorsed Transactions on Scalable Information Systems, 2024. DOI:~10.4108/eetsis.5437
\bibitem{Chaouche2023Graph}\textbf{Chaouche~A., Boulif~M.} Edge-set reduction to efficiently solve the graph partitioning problem with the genetic algorithm. // arXiv, 2023. DOI: 10.48550/arXiv.2307.10410
\bibitem{Li2020Graph}\textbf{Li~M., Cui~H., Zhou~C., Xu~S.} GAP: genetic algorithm based large-scale graph partition in heterogeneous cluster. // IEEE Access, 2020. DOI:~10.1109/ACCESS.2020.3014351
\bibitem{Baranov2025Gen}\textbf{Baranov~A., Aladyshev~O., Bragin~K.} Job mapping cyclic composite algorithm for supercomputer resource manager. // Lecture Notes in Computer Science, 2025, Vol.~15406, P.~377-389. DOI:~10.1007/978-3-031-78459-0\_27

% Сглаживание границ доменов.
\bibitem{Bagrov2021Smooth}\textbf{Багров~А., Рыбаков~А.} Сглаживание границ между доменами поверхностной расчетной сетки. // Современные информационные технологии и ИТ-образование, 2021, Т.~17, №~2, С.~265-274. DOI:~10.25559/SITITO.17.202102.265-274

% Масштабирование вычислений на суперкомпьютере.
\bibitem{Shabanov2021Scaling}\textbf{Shabanov~B., Rybakov~A., Shumilin~S., Vorobyov~M.} Scaling of supercomputer calculations on unstructered surface computational meshes. // Lobachevskii Journal of Mathematics, 2021, Vol.~42, No.~11, P.~2571-2579. DOI:~10.1134/S1995080221110202

% Общая память.
\bibitem{Section3IntroIntel}Рейтинг серверных процессоров Intel Xeon: сравнение производительности и выбор лучшего. // СерверМолл, 11.04.2025. URL:~https://servermall.ru/blog/reyting-servernykh-protsessorov-intel-xeon-sravnenie-proizvoditelnosti-i-vybor-luchshego
\bibitem{Section3IntroAMD}Intel такое и не снилось. AMD создала процессор для настольных ПК с поистине гигантским числом ядер. Конкурентов нет. // CNews, 21.05.2015. URL:~https://www.cnews.ru/news/top/2025-05-21\_intel\_takoe\_i\_ne\_snilosamd\_sozdala (дата обращения \StrDate)
\bibitem{Kuzminsky2022ARM}\textbf{Кузьминский~М.~Б.} Современные серверные ARM-процессоры для суперЭВМ: A64FX и другие. Начальные данные тестов производительности. // Программные системы: теория и приложения, 2022, Т.~13, №~1(52), С.~63-129. DOI:~10.25209/2079-3316-2022-13-1-63-129

% Разрешение конфликтов на общей памяти.
\bibitem{Gulicheva2024}\textbf{Гуличева~А., Рыбаков~А.} Реберная раскраска кубического графа в задаче распараллелирования расчетов на неструктурированной поверхностной расчетной сетке. // Программные продукты и системы, 2024, Т.~37(3), С.~374-383. DOI:~10.15827/0236-235X.142.374-383
\bibitem{Gilfanov2021Coloring}\textbf{Гильфанов~Л., Мигранов~С., Бикбов~А.} Распараллеливание решения задач с использованием раскраски графов. // Молодой ученый, 2021, №~5 (347), С.~4-6. URL:~https://moluch.ru/archive/347/78208/ (дата обращения \StrDate)
\bibitem{Vizing1964}\textbf{Визинг~В.} Об оценке хроматического класса р-графа. // сб. Дискретный анализ. -- Новосибирск: Институт математики СО АН СССР, 1964, Вып.~3, С.~25–30.
\bibitem{Vizing1965}\textbf{Визинг~В.} Критические графы с данным хроматическим классом. // сб. Дискретный анализ. -- Новосибирск: Институт математики СО АН СССР, 1965, Вып.~5, С.~9–17.
\bibitem{Soifer2009}\textbf{Soifer~A.} The mathematical coloring book. // 2009, Springer, 619~P.
\bibitem{Tait1880}\textbf{Tait~P.} Remarks on the colourings of maps. // Proceedings of the Royal Society of Edinburgh, 1880, Vol.~10, №~4, P.~501-503. DOI:~10.1017/S0370164600044229
\bibitem{Kurapov2018}\textbf{Курапов~С., Давидовский~М., Толок~А.} Визуальный алгоритм раскраски плоских графов. // Научная визуализация, 2018, Т.~10, №~3, С.~1-33. DOI:~10.26583/sv.10.3.01
\bibitem{Kurapov2020}\textbf{Курапов~С., Давидовский~М., Толок~А.} Алгебраические методы раскраски кубических графов. // Научная визуализация, 2020, Т.~12, №~2, С.~21-36. DOI:~10.26583/sv.12.2.03
\bibitem{Kurapov2020Mono}\textbf{Курапов~С., Давидовский~М.} Теорема о четырех красках. Теория, методы, алгоритмы. // Запорожье: Запорожский национальный университет, 2020, 94~С.

% Масштабируемость.
\bibitem{Vorobyov2020ParVec}\textbf{Воробьев~М., Рыбаков~А., Чопорняк~А.} Сравнение стратегий распараллеливания векторизованного римановского решателя с помощью OpenMP для микропроцессора Intel Xeon Phi KNL. // Труды НИИСИ РАН, 2020, Т.~10, №~5-6, С.~113-119. DOI:~10.25682/NIISI.2020.5\_6.0015
\bibitem{Vorobyov2020Scaling}\textbf{Воробьев~М., Рыбаков~А., Никсон~М.} Исследование масштабируемости плотных параллельных вычислений на микропроцессорах Intel. // Инжиниринг предприятий и управление знаниями (ИП \& УЗ-2020). Сборник научных трудов XXIII Международной научной конференции, 2020, С.~45-52.

%---------------------------------------------------------------------------------------------------
% Глава 5. - Векторизация.

% Обзорная часть.
\bibitem{Cebrian2019VecScal}\textbf{Cebrian~J., Natvig~L., Lahre~M.} Scalability analysis of AVX-512 extensions. // The Journal of Supercomputing, 2020, Vol.~76, P.~2082-2097. DOI:~10.1007/s11227-019-02840-7
% x86
\bibitem{IntelSDM2025}Intel 64 and IA-32 architectures software developer's manual. Combined volumes: 1, 2A, 2B, 2C, 2D, 3A, 3B, 3C, 3D, and 4. // Order number: 325462-087US, march 2025.
% ARM
\bibitem{Zhuykov2012VecARM}\textbf{Жуйков~Р., Плотников~Д., Варданян~М.} Автоматическая настройка оптимизационных преобразований компилятора GCC для платформы arm. // Труды ИСП РАН, 2012. URL:~https://cyberleninka.ru/article/n/avtomaticheskaya-nastroyka-optimizatsionnyh-preobrazovaniy-kompilyatora-gcc-dlya-platformy-arm (дата обращения \StrDate).
%\bibitem{Marquez2020VecARM}\textbf{M{\'a}rques~R., Sarmiento~A., S{\'a}nchez-Solano~S.} Speeding up elliptic curve arithmetic on ARM processors using NEON instructions. // Revista de Ingenieria Electr{\'o}nica, Autom{\'a}tica y Comunicaciones, 2020, Vol.~41, №~3, P.~1-20.
\bibitem{Stephens2017VecARM}\textbf{Stephens~N., Biles~S., Boetthcher~M., Eapen~J.} The ARM Scalable Vector Extension. // IEEE Micro, 2017, Vol.~37, Issue~2, P.~26-39. DOI:~10.1109/MM.2017.35
\bibitem{Okazaki2020A64FX}\textbf{Okazaki~R., Tabata~T., Sakashita~S., Kitamura~K. et al.} Supercomputer Fugaku CPU A64FX realizing high performance, high-density packaging, and low power consumption. // Fujitsu Technical Review, 2020. URL:~https://www.fujitsu.com/global/about/resources/publications/technicalreview/2020-03/article03.html?ysclid=mbawuuc4vr289719768
% Power
%\bibitem{Gschwind2016VecPower}\textbf{Gschwind~M.} Workload acceleration with the IBM Power vector-scalar architecture. // IBM Journal of Research and Development, 2016, Vol.~60, №~2/3, Paper~14. DOI:~10.1147/JRD.2016.2527418
\bibitem{Eisen2007VecPower}\textbf{Eisen~L., Ward~III~J., Tast~H.-W., M{\"a}ding~N. et al.} IBM Power6 accelerators: VMX and DFU. // IBM Journal of Research and Development, 2007, 51(6), P.~1-21. DOI:~10.1147/rd.516.0663
% Эльбрус
\bibitem{Ishin2011VecElbrus}\textbf{Ишин~П.~А.} Оптимизация преобразования Фурье под архитектуру Эльбрус. // Современные информационные технологии и ИТ-образование, 2011 №~7. URL:~https://cyberleninka.ru/article/n/optimizatsiya-preobrazovaniya-furie-pod-arhitekturu-elbrus (дата обращения: \StrDate).
\bibitem{Volkonsky2012VecElbrus}\textbf{Волконский~В.~Ю., Брегер~А.~В., Бучнев~А.~Ю., Грабежной~А.~В. et al.} Методы распараллеливания программ в оптимизирующем компиляторе. // Вопросы радиоэлектроники, 2012, Т.~4, №~3, С.~63-88.
% Китайкие
\bibitem{Bai2024VecLoongarch}\textbf{Bai~Z., Wu~Q., Cao~K., Sun~Y. et al.} Application of regional meteorology and air quality models based on the microprocessor without interlocked piped stages (MIPS) and LoongArch CPU platforms. // Geoscientific Model Development, 2024, Vol.~17(10), P.~4383-4399. DOI:~10.5194/gmd-17-4383-2024
\bibitem{Sun2023VecSunway}\textbf{Sun~Z., Wang~Z., Hua~M., Xiong~P. et al.} Accelerating ray tracing engine of BLENDER on the new Sunway architecture. // Engineering Reports, 2025, 7:e12789. DOI:~10.1002/eng2.12789

% 5.1 AVX-512
\bibitem{IntelIntrinsicsGuide}Intel Intrinsics Guide, https://alouettesu.github.io/Intrinsics/ (дата обращения \StrDate)
\bibitem{Jeffers2016KNL}\textbf{Jeffers~J., Reinders~J., Sodani~A.} Intel Xeon Phi processor high performance programming: Knights Landing edition. // Morgan Kaufmann, 2016, 662~P.

% Применение AVX-512.
\bibitem{Kulikov2019VecAstro}\textbf{Kulikov~I., Chernykh~I., Tutukov~A.} A new hydrodynamic code with explicit vectorization instructions optimizations that is dedicated to the numerical simulation of astrophysical gas flow. I. Numerical method, tests, and model problem. // The Astrophysical Journal Supplement Series, 2019, Vol.~243, No.~4, 15~P. DOI:~10.3847/1538-4365/ab2237
\bibitem{Glinting2019VecSwim}\textbf{Glinting~B., Mundani~R.-P.} Comparison of shallow water solvers: applications for dam-break and tsunami cases with reordering strategy for efficient vectorization on modern hardware. // Water, 2019, Vol.~11(4), No.~639, 31~P. DOI:~10.3390/w11040639
\bibitem{Yildirim2021VecCFD}\textbf{Yildirim~A., Mader~C., Martins~J.} Accelerating parallel CFD codes on modern vector processors using blockettes. // PASC’21: Proceedings of the Platform for Advanced Scientific Computing Conference, 2021. DOI:~10.1145/3468267.3470615
\bibitem{Rucci2020VecNBody}\textbf{Rucci~E., Moreno~E., Pousa~A., Chichizola~F.} Optimization of the N-body simulation on Intel’s architectures based on AVX-512 instruction set. // In book: Comminications in Computer and Information Science, 2020. DOI:~10.1007/978-3-030-48325-8\_3
\bibitem{Rucci2019VecSW}\textbf{Rucci~E., Garcia~C., Botella~G., De~Giusti~A.} SWIMM 2.0: Enhanced Smith-Waterman on Intel’s multicore and manycore architectures based on AVX-512 vector extensions. // International Journal of Parallel Programming, 2019, Vol.~47(17). DOI:~10.1007/s10766-018-0585-7
%\bibitem{Choi2023VecKorean}\textbf{Choi~Y., Choi~H., Chung~S.} AVX512Crypto: Parallel implementations of Korean block ciphers using AVX-512. // IEEE Access, 2023. DOI:~10.1109/ACCESS.2023.3278993
%\bibitem{Cheng2021VecCSIDH}\textbf{Cheng~H., Fotiadis~G., Gro{\ss}sch{\"a}dl~J., Ryan~P., R{\o}nne~P.} Batching CSIDH group actions using AVX-512. // IACR Transactions on Cryptographic Hardware and Embedded Systems, 2021, Vol.~2021, No.~4, P.~618-649. DOI:~10.46586/tches.v2021.i4.618-649
\bibitem{Blacher2022VecQuick}\textbf{Blacher~M., Giesen~J., Sanders~P., Wassenberg~J.} Vectorized and performance-portable Quicksort. // ArXiv, 2022, art.~2205.05982, P.~1–21. DOI:~10.48550/arXiv.2205.05982
\bibitem{Long2022VecSPD}\textbf{Long~S., Fan~X., Chao~L., Yi~L., Fan~S., Guo~X.-W., Yang~C.} VecDualSPHysics: A vectorized implementation of Smoothed Particle Hydrodynamics method for simulating fluid flows on multi-core processors. // Journal of Computational Physics, 2022, Vol.~463, art.~111234. DOI:~10.1016/j.jcp.2022.111234
\bibitem{PonteFernandez2022VecInteractions}\textbf{Ponte-Fern{\'a}ndez~C., Gonz{\'a}lez-Dom{\'i}nguez~J., Mart{\'i}n~M.~J.} A SIMD algorithm for the detection of epistatic interactions of any order. // Future Generation Comput. Sys., 2022, Vol.~132, P.~108–123. DOI:~10.1016/j.future.2022.02.009
\bibitem{Quisland2023VecSeries}\textbf{Quislant~R., Fernandez~I.} Time series analysis acceleration with advanced vectorization extensions. // The Journal of Supercomputing, 2023, Vol.~79, No.~9, P.~10178–10207. DOI:~10.1007/s11227-023-05060-2
\bibitem{Buhrow2022VecMult}\textbf{Buhrow~B., Gilbert~B., Haider~C.} Parallel modular multiplication using 512-bit advanced vector instructions. // Journal of Cryptographic Engineering, 2022, Vol.~12, P.~95–105. DOI:~10.1007/s13389-021-00256-9
\bibitem{Choi2022VecPIPO}\textbf{Choi~H., Seo~S.~C.} Efficient parallel implementations of PIPO Block Cipher on CPU and GPU. // IEEE Access, 2022, Vol.~10, P.~85995–86007. DOI:~10.1109/ACCESS.2022.3198707
\bibitem{Cheng2022VecSIKE}\textbf{Cheng~H., Fotiadis~G., Gro{\ss}sch{\"a}dl~J., Ryan~P.}. Highly vectorized SIKE for AVX-512. // IACR Transactions on Cryptographic Hardware and Embedded Sys., 2022, No.~2, P.~41–68. DOI:~10.46586/tches.v2022.i2.41-68
\bibitem{Sansone2023VecFourier}\textbf{Sansone~G., Cococcioni~M.} Experiments on speeding up the recursive fast Fourier transform by using AVX-512 SIMD instructions. // Proc. ApplePies. LNEE, 2023, Vol.~1036, P.~255–263. DOI:~10.1007/978-3-031-30333-3\_34
\bibitem{Edamatsu2023VecDiv}\textbf{Edamatsu~T., Takahashi~D.} Fast multiple-precision integer division using Intel AVX-512. // IEEE Transactions on Emerging Topics in Computing, 2023, Vol.~11, No.~1, P.~224–236. DOI:~10.1109/TETC.2022.3196147
\bibitem{Medakin2021VecPP}\textbf{Медакин~П., Никулин~Р., Авдеюк~О., Королева~И., Павлова~Е., Лемешкина~И.} Векторизация и распараллеливание метода «частица-частица». // Инженерный вестник Дона, 2021, No.~1. URL:~http://ivdon.ru/ru/magazine/archive/n1y2021/6800 (дата обращения: \StrDate)
\bibitem{Tayeb2023VecAuto}\textbf{Tayeb~H., Paillat~L., Bramas~B.} Autovesk: Automatic vectorization of unstructured static kernels by graph transformations. // ArXiv, 2023, art.~2301.01018, P.~1–21. DOI:~10.48550/arXiv.2301.01018
\bibitem{Laukemann2019VecAuto}\textbf{Laukemann~J., Hammer~J., Hager~G., Wellein~G.} Automatic throughput and critical path analysis of x86 and ARM assembly kernels. // Proc. IEEE/ACM PMBS, 2019, P.~1–6. DOI:~10.1109/PMBS49563.2019.00006
\bibitem{Kusswurm2022VecCpp}\textbf{Kusswurm~D.} Modern parallel programming with C++ and Assembly Language. X86 SIMD development using AVX, AVX2, and AVX-512. // CA, Apress Berkeley Publ., 2022, 633~ P. DOI:~10.1007/978-1-4842-7918-2

% Описание инструкций AVX-512.
\bibitem{Kalamkar2019VecBF16}\textbf{Kalamkar~D., Mudigere~D., Mellempudi~N., Das~D. et al.} A study of BFloat16 for deep learning training. // ArXiv, 2019, art.~1905.12322, P.~1-11. DOI:~10.48550/arXiv.1905.12322
\bibitem{Zhou2024VecVNNI}\textbf{Zhou~H., Han~Q., Shi~H., Zhang~Y., Yao~J.} Boost linear algebra computation performance via efficient VNNI utilization. // ASPLOS '24: Proceedings of the 29th ACM International Conference on Architectural Support for Programming Languages and Operating Systems, 2024, Vol.~3, P.~149-163. DOI:~10.1145/3620666.3651333
\bibitem{DiezCanas2021VecVP2Int}\textbf{D{\'i}ez-Ca{\~n}as~G.} Faster-than-native alternatives for x86 VP2INTERSECT instructions. // ArXiv, 2021, art.~2112.06342, P.~1-10. DOI:~10.48550/arXiv.2112.06342
\bibitem{Kovats2024VecAES}\textbf{Kovats~T., Rameshan~N., Karunarathe~K., Giannopoulos~I., Sebastian~A.} In-memory encryption using the advanced encryption standard. // Philosophical Transactions, 2025, Vol.~383, art.~20230396. DOI:~10.1098/rsta.2023.0396
\bibitem{Yoo2023VecGFNI}\textbf{Yoo~T.-H., Kivilinna~J., Cho~C.-H.} AVX-based acceleration of ARIA block cipher algorithm. // IEEE Access, 2023. DOI:~10.1109/ACCESS.2023.3298026
\bibitem{Volkonsky2003VecPred}\textbf{Волконский~В., Окунев~С.} Предикатное представление как основа оптимизации программы для архитектур с явно выраженной параллельностью. // Информационные технологии, 2003, №~4, С.~36–45.
\bibitem{Kim2013VecElb}\textbf{Ким~А., Перекатов~В., Ермаков~С.} Микропроцессоры и вычислительные комплексы семейства «Эльбрус», Питер, СПб., 2013, 273~С.

% 5.2 Выделение однотипных операций - матрицы малой размерности.
\bibitem{Bendersky2018VecMat2}\textbf{Бендерский~Л., Лещев~С., Рыбаков~А.} Векторизация операций над матрицами малой размерности для процессора Intel Xeon Phi Knights Landing. // Современные информационные технологии и ИТ-образование, 2018, Т.~14, №~1, С.~73-90. DOI:~10.25559/SITITO.14.201801.073-090
\bibitem{iparGithub}Parallelization samples for Intel microprocessors. // репозиторий github.com. https://github.com/r-aax/ipar (дата обращения \StrDate)
\bibitem{Bendersky2018VecMat1}\textbf{Бендерский~Л., Рыбаков~А., Шумилин~С.} Векторизация перемножения малоразмерных матриц специального вида с использованием инструкций AVX-512. // Современные информационные технологии и ИТ-образование, 2018, Т.~14, №~3, С.~594-602. DOI:~10.25559/SITITO.14.201803.594-602

% 5.3 - Введение понятия плоского цикла.
\bibitem{Shabanov2021VecCFG}\textbf{Шабанов~Б., Рыбаков~А., Чопорняк~А.} Оптимизации, применяемые к графу потока управления программы для повышения эффективности векторизации плоских циклов. // Труды НИИСИ РАН, 2021, Т.~11, №~2, С.~11-19. DOI:~10.25682/NIISI.2021.2.0002
\bibitem{Armstrong2013VecErlang}\textbf{Armstrong~J.} Programming Erlang. Software for a concurrent world. // The Pragmatic Programmers, 2013, 520~P.
\bibitem{Savin2020VecFlat}\textbf{Savin~G., Shabanov~B., Rybakov~A., Shumilin~S.} Vectorization of flat loops of arbitrary structure using instructions AVX-512. // Lobachevskii Journal of Mathematics, 2020, Vol.~41, No.~12, P.~2566-2574. DOI:~10.1134/S1995080220120331
\bibitem{Muchnick1997Compilers}\textbf{Muchnick~S.} Advanced compiler design and implementation. // Morgan Kaufmann Publishers, 1997.
\bibitem{Rybakov2013CGF}\textbf{Рыбаков~А.} Алгоритм создания случайных графов потока управления для анализа глобальных оптимизаций в компиляторе. // Parallel and Distributed Computing Systems PDSC 2013, Collection of scientific papers, P.~269-275.
\bibitem{Aho2006Compilers}\textbf{Aho~A., Lam~M., Sethi~R., Ulman~J.} Compilers: principles, techniques, and tools. // Prentice Hall, 2nd ed, 2006.
\bibitem{Chetverina2015Profile}\textbf{Четверина~О.} Методы коррекции профильной информации в процессе компиляции. // Труды ИСП РАН, 2015, Т.~27, Вып.~6, С.~49-63.
\bibitem{Rybakov2023VecIBM}\textbf{Рыбаков~А., Мещеряков~А.} Векторизация трехмерного метода погруженных границ для повышения эффективности расчетов на микропроцессорах Intel. // Программные продукты и системы, 2023, Т.~36, №~1, C.~130-143. DOI:~10.15827/0236-235X.141.130-143
\bibitem{ibmGithub}Immersed boundary methods vectorization. // репозиторий github.com. https://github.com/r-aax/ibm\_vec (дата обращения \StrDate)

% 5.4 - Векторизация с помощью локализации маловероятных регионов.
\bibitem{Rybakov2018VecBranch}\textbf{Рыбаков~А., Шумилин~С.} Векторизация сильно разветвленного управления с помощью инструкций AVX-512. // Труды НИИСИ РАН, 2018, Т.~8, №~4, С.~114-126. DOI:~10.25682/NIISI.2018.4.0014
\bibitem{Aubakirov1999Wake}\textbf{Аубакиров~Т., Желанников~А., Иванов~П., Ништ~М.} Спутные следы и их воздействие на летательные аппараты. // Моделирование на ЭВМ, Алматы, 1999, 280~С.
%\bibitem{Vyshinsky2006Wake}\textbf{Вышинский~В., Судаков~Г.} Вихревой след самолёта в турбулентной атмосфере. // Труды ЦАГИ, 2006, Вып.~2667, 155~С.
\bibitem{Babkin2008Wake}\textbf{Бабкин~В., Белоцерковский~А., Турчак~Л., Баранов~Н., Замятин~А., Каневский~М., Морозов~В., Пасекунов~И., Чижов~Н.} Системы обеспечения вихревой безопасности полетов летательных аппаратов. // М.: Наука, 2008, 373~С.
%\bibitem{Burluzky2014Wake}\textbf{Бурлуцкий~С.} Вопросы обеспечения вихревой безопасности аэропортов. // Системный анализ и логистика. Специальное научное издание, Вып. от 12 мая 2014 года, С.~37-40.
\bibitem{Rybakov2022VecGeom}\textbf{Рыбаков~А.} Векторизация программного кода, содержащего маловероятные регионы, в задачах вычислительной геометрии. // Современные информационные технологии и ИТ-образование, 2022, Т.~18, №~1, С.~28-38. DOI:~10.25559/SITITO.18.202201.28-38
%\bibitem{Ilbeyi2019}\textbf{Ilbeyi~B.} Co-optimizing hardware design and meta-tracing just-in-time compilation. // A disserta-tion for the degree of Doctor of Philosophy, Cornell University, 2019.

% 5.5 - Векторизация слияние под условием.
\bibitem{Rybakov2024VecComb}\textbf{Рыбаков~А.} Векторизация циклов с условными операциями с помощью комбинирования векторных масок. // Современные информационные технологии и ИТ-образование, 2024, Т.~20, №~3, С.~520-534. DOI:~10.25559/SITITO.020.202403.520-534

% 5.6 - Векторизация - комбинирование масок.
\bibitem{Rybakov2020VecMon}\textbf{Рыбаков~А., Чопорняк~А.} Повышение производительности векторного кода с помощью мониторинга плотности масок в векторных инструкциях. // Труды НИИСИ РАН, 2020, Т.~10, №~4, С.~40-47. DOI:~10.25682/NIISI.2020.4.0006
\bibitem{Toh2024VecRiemann}\textbf{Toh~Y.} Efficient non-iterative multi-point method for solving the Riemann problem. // Nonlinear Dynnamics, 2024, Vol.~112, P.~5439-5451. DOI:~10.1007/s11071-023-09229-5
%\bibitem{Zeng2021VecRiemann}\textbf{Zeng~Z., Feng~C., Yu~C. et al.} Linearized double-shock approximate Riemann solver for augmented linear elastic solid. // Numerical Mathematics Theory Methods and Applications, 2021, Vol.~15. DOI:~10.4208/nmtma.OA-2021-0021
%\bibitem{Lee2024VecGem}\textbf{Lee~S., Kim~Y., Nam~D. et al.} Gem5-AVX: Extension of the Gem5 simulator to support AVX instruction sets. // IEEE Access, 2024. DOI:~10.1109/ACCESS.2024.3359296
\bibitem{Rybakov2023VecShvindt}\textbf{Рыбаков~А., Швиндт~А.} Создание инструментария для векторизации тела плоского цикла с помощью векторных инструкций AVX-512. // Программные продукты и системы, 2023, Т.~36, №~4, С.~561-572. DOI:~10.15827/0236-235X.142.561-572

% 5.7 - Векторизация гнезд.
\bibitem{Rybakov2019VecInt}\textbf{Рыбаков~А.} Векторизация нахождения пересечения объемной и поверхностной сеток для микропроцессоров с поддержкой AVX-512. // Труды НИИСИ РАН, 2019, Т.~9, №~5, С.~5-14. DOI:~10.25682/NIISI.2019.5.0001
\bibitem{Rybakov2019VecRiem1}\textbf{Rybakov~A., Shumilin~S.} Vectorization of the Riemann solver using the AVX-512 instruction set. // Program Systems: Theory and Applications, 2019, Vol.~10, №~3(42), P.~41-58. DOI:~10.25209/2079-3316-2019-10-3-41-58
\bibitem{Rybakov2019VecRiem2}\textbf{Рыбаков~А., Шумилин~С.} Векторизация римановского решателя с использованием набора инструкций AVX-512. // Программные системы: Теория и приложения, 2019, Т.~10, №~3(42), С.~59-80. DOI:~10.25209/2079-3316-2019-10-3-59-80
%\bibitem{numericaGithub}NUMERICA. Hyperbolic Solvers. // репозиторий github.com. https://github.com/dasikasunder/NUMERICA (дата обращения \StrDate)
\bibitem{Krzikalla2026Vec}\textbf{Krzikalla~O., Wende~F., Hohnerbach~M.} Dynamic SIMD vector lane scheduling. // ISC High Performance 2016: High Performance Computing, Lectuer Notes Computer Science, 2016, Vol.~9945, P.~354–365. DOI:~10.1007/978-3-319-46079-6\_25
\bibitem{Bulat2015VecRiemann}\textbf{Булат~П., Волков~К.} Одномерные задачи газовой динамики и их решение при помощи разностных схем высокой разрешающей способности. // Научно-технический вестник информационных технологий, механики и оптики, 2015, Т.~15, №~4, С.~731–740. DOI:~ 10.17586/2226-1494-2015-15-4-731-740
\bibitem{Rybakov2019VecIrr}\textbf{Рыбаков~А., Шумилин~С.} Исследование эффективности векторизации гнезд циклов с нерегулярным числом итераций. // Программные системы: Теория и приложения, 2019, Т.~10, №~4(43), С.~77-96. DOI:~10.25209/2079-3316-2019-10-4-77-96
\bibitem{Shabanov2019VecSci}\textbf{Shabanov~B., Rybakov~A., Shumilin~S.} Vectorization of high-performance scientific calculations using AVX-512 instruction set. // Lobachevskii Journal of Mathematics, 2019, Vol.~40, No.~5, P.~580-598. DOI:~10.1134/S1995080219050196
\bibitem{Rybakov2018VecNest}\textbf{Рыбаков~А., Телегин~П., Шабанов~Б.} Проблемы векторизации гнезд циклов с использованием инструкций AVX-512. // Электронный научный журнал: Программные продукты, системы и алгоритмы, 2018, №~3, С.~1-11. DOI:~10.15827/2311-6749.28.314
\bibitem{MOVUPSintel}Using Intel AVX without Writing AVX. // https://software.intel.com/enus/articles/using-intel-avx-without-writing-avx (дата обращения \StrDate)
\bibitem{comboptGithub}Векторизация алгоритма пузырькового роста декомпозиции графа. // репозиторий github.com. https://github.com/r-aax/comb\_opt\_vect (дата обращения \StrDate)

%---------------------------------------------------------------------------------------------------

\end{thebibliography}