% Список литературы.
\newpage

\renewcommand{\baselinestretch}{1.0}
\begin{thebibliography}{99}
\addcontentsline{toc}{section}{Список литературы} % добавить в оглавление список литературы

% bank
% - https://github.com/tpn/pdfs

%---------------------------------------------------------------------------------------------------
% Введение.


%---------------------------------------------------------------------------------------------------
% Глава 1.

% Основные понятия.
% 
\bibitem{Chernikov1963}\textbf{Черников С.} Свертывание конечных систем линейных неравенств. // Доклады АН СССР, 1963, Т. 152, № 5, С. 1075-1078.

% Метод погруженной границы.
%
\bibitem{Mahesh2003}\textbf{Mahesh K., Constantinescu G., Moin P.} Simulating turbulent flows in complex geometries. // Proceedings of FEDSM2003 2003 4th ASME JSME Joint Fluids Engineering Conference, 2003. DOI: 10.1115/FEDSM2003-45337
%
\bibitem{Ye2020}\textbf{Ye H., Liu Y., Chen B., Liu Z., Zheng J., Pang Y., Chen J.} Hybrid grid generation for viscous flow simulation in complex geometries. // 2020. DOI: 10.21203/rs.3.rs-31698/v1
%
\bibitem{Wright2015}\textbf{Wright W., Struk P., Bartkus T., Addy G.} Recent advances in the LEWICE icing model. // SAE Technical Paper, 2015. DOI: 10.4271/2015-01-2094
%
\bibitem{BourgaultCote2017}\textbf{Bourgault-Côté S., Hasanzadeh K., Lavoie P., Laurendeau E.} Multi-layer methodologies for conservative ice growth. // 7th European Conference for Aeronautics and Aerospace Sciences (EUCASS), 2017. DOI: 10.13009/EUCASS2017-258
%
\bibitem{Tong2016}\textbf{Tong X., Thompson D., Arnoldus Q., Collins E., Luke E.} Three-dimensional surface evolution and mesh deformation for aircraft icing applications. // Journal of Aircraft, 2016. DOI: 10.2514/1.C033949
%
\bibitem{Abalakin2018}\textbf{Абалакин И., Жданова Н., Козубская Т.} Метод погруженных границ для численного моделирования невязких сжимаемых течений. // Журнал вычислительной математики и математической физики, 2018, Т. 58, № 9, С. 1462-1471. DOI: 10.31857/S004446690002525-8
%
\bibitem{Mori2008}\textbf{Mori Y., Peskin C.} Implicit second-order immersed boundary methods with boundary mass. // Comput. Methods Appl. Mech. Engrg, 2008, Vol. 197, P. 2049-2067. DOI: 10.1016/J.CMA.2007.05.028
%
\bibitem{Kim2004}\textbf{Kim J., Choi H.} An immersed-boundary finite-volume method for simulation of heat transfer in complex geometries. // KSME International Journal, 2004, Vol. 18, № 6, P. 1026-1035. DOI: 10.1007/BF02990875
%
\bibitem{Clarke1996}\textbf{Clarke D., Salas M., Hassan H.} Euler calculations for multielement airfoils using cartesian grids. // AIAA Journal, 1986, Vol. 24, № 3, P. 353-358. DOI: 10.2514/3.9273
%
\bibitem{Farrashkhalvat2003}\textbf{Farrashkhalvat M., Miles J.} Basic structured grid generation with an introduction to unstructured grid generation. // 2003, Butterworth Heinemann, 231 p.
%
\bibitem{Rybakov2017}\textbf{Рыбаков А.} Внутреннее представление и механизм межпроцессного обмена для блочно-структурированной сетки при выполнении расчетов на суперкомпьютере. // Программные системы: теория и приложения, 2017, Т. 8, Вып. 1, С. 121-134. DOI: 10.25209/2079-3316-2017-8-1-121-134
%
\bibitem{Savin2019}\textbf{Savin. G., Benderskiy L., Lyubimov D., Rybakov A.} RANS/ILES method optimization for effective calculations on supercomputer. // Lobachevskii Journal of Mathematics, 2019, Vol. 40, No. 5, P. 566-573. DOI: 10.1134/S1995080219050172
%
\bibitem{Giordano2019}\textbf{Giordano A., De Rango A., Rongo R., D'Ambrosio D., Spataro W.} A dynamic load balancing technique for parallel execution of structured grid models. // 2020, In: Sergeyev Y., Kvasov D. (eds) Numerical Computations: Theory and Algorithms. NUMTA 2019. Lecture Notes in Computer Science, Vol. 11973, Springer. DOI: 10.1007/978-3-030-39081-5\_25
%
\bibitem{Fadlun2000}\textbf{Fadlun E., Verzicco R., Orlandi P., Mohd-Yusof J.} Combined immersed-boundary finite-difference methods for three-dimensional complex flow simulations. // Journal of Computational Physics, 2000, Vol. 161, P. 35-60. DOI: 10.1006/jcph.2000.6484
%
\bibitem{Mittal2005}\textbf{Mittal R., Iaccarino G.} Immersed boundary methods. // Annual. Rev. Fluid Mech., 2005, Vol. 37, P. 239-261. DOI: 10.1146/annurev.fluid.37.061903.175743
%
\bibitem{Tseng2003}\textbf{Tseng Y.-H., Ferziger J.} A ghost-cell immersed boundary method for flow in complex geometry. // Journal of Computational Physics, 2003, Vol. 192, P. 593-623. DOI: 10.1016/j.jcp.2003.07.024
%
\bibitem{Peter2016}\textbf{Peter S., De A.} A parallel implementation of the ghost-cell immersed boundary method with application to stationary and moving boundary problems. // Sadhana, 2016, Vol. 41, № 4, P. 441-450. DOI: 10.1007/s12046-016-0484-9
%
\bibitem{Vinnikov2007}\textbf{Винников В., Ревизников Д.} Метод погруженной границы для расчета сверхзвукового обтекания затупленных тел на прямоугольных сетках. // Электронный журнал «Труды МАИ», 2007, № 27, 13 С.
%
\bibitem{Rybakov2019}\textbf{Рыбаков А.} Векторизация нахождения пересечения объемной и поверхностной сеток для микропроцессоров с поддержкой AVX-512. // Труды НИИСИ РАН, 2019, Т. 9, № 5, С. 5-14.
%
\bibitem{Wackers2011}\textbf{Wackers J., Deng G., Leroyer A., Queutey P., Visonneau M.} Adaptive grid refinement for hydrodynamic flows. // Computers \& Fluids, 2011, 55, P. 85-100. DOI: 10.1016/j.compfluid.2011.11.004
%
\bibitem{Zhou2014}\textbf{Zhou L., Yunjun Y., Anlong G., Weijiang Z.} Unstructured adaptive grid refinement for flow feature capture. // Procedia Engineering, 2014, 99, P. 477-483. DOI: 10.1016/j.proeng.2014.12.561
%
\bibitem{Plas2015}\textbf{van der Plas P., Veldman A., van der Heiden H., Luppes R.} Adaptive grid refinement for free-surface flow simulations in offshore applications. // Proceedings of the ASME 2015 34th International Conference on Ocean, Offshore and Arctic Engineering OMA2015, 2015. DOI: 10.1115/OMAE2015-42029
%
\bibitem{Smirnova2018}\textbf{Смирнова Н.} Сравнение схем с расщеплением потока для численного решения уравнения Эйлера сжимаемого газа. // Труды МФТИ, 2018, Т. 10, № 1, С. 122-141.

%---------------------------------------------------------------------------------------------------
% Глава 2.

%---------------------------------------------------------------------------------------------------
% Глава 3.

% Положения из теории графов.
%
\bibitem{Vizing1964}\textbf{Визинг В.} Об оценке хроматического класса р-графа. // сб. Дискретный анализ. -- Новосибирск: Институт математики СО АН СССР, 1964, Вып. 3, С. 25–30.
%
\bibitem{Vizing1965}\textbf{Визинг В.} Критические графы с данным хроматическим классом. // сб. Дискретный анализ. -- Новосибирск: Институт математики СО АН СССР, 1965, Вып. 5, С. 9–17.
%
\bibitem{Soifer2009}\textbf{Soifer A.} The mathematical coloring book. // 2009, Springer, 619 P.

%---------------------------------------------------------------------------------------------------
% Глава 4.

% Особенности AVX-512.
%
\bibitem{IntelSDM2025}Intel 64 and IA-32 architectures software developer's manual. Combined volumes: 1, 2A, 2B, 2C, 2D, 3A, 3B, 3C, 3D, and 4. // Order number: 325462-087US, march 2025.
%
\bibitem{Volkonsky2003}\textbf{Волконский В., Окунев С.} Предикатное представление как основа оптимизации программы для архитектур с явно выраженной параллельностью. // Информационные технологии, 2003, № 4, С. 36–45.
%
\bibitem{Kim2013}\textbf{Ким А., Перекатов В., Ермаков С.} Микропроцессоры и вычислительные комплексы семейства «Эльбрус», Питер, СПб., 2013, 273 С.
%
\bibitem{IntelIntrinsicsGuide}Intel Intrinsics Guide, https://alouettesu.github.io/Intrinsics/ (дата обращения 01.05.2025)

%---------------------------------------------------------------------------------------------------

\end{thebibliography}