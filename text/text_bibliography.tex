% Список литературы.
\newpage

\renewcommand{\baselinestretch}{1.0}
\begin{thebibliography}{99}
\addcontentsline{toc}{section}{Список литературы} % добавить в оглавление список литературы

% bank
% - https://github.com/tpn/pdfs

%---------------------------------------------------------------------------------------------------
% Введение.



%---------------------------------------------------------------------------------------------------
% Глава 1.



% Основные понятия.
% 
\bibitem{Chernikov1963}\textbf{Черников~С.} Свертывание конечных систем линейных неравенств. // Доклады АН СССР, 1963, Т. 152, № 5, С. 1075-1078.
%
\bibitem{Rybakov2017Flight}\textbf{Рыбаков~А.} Оптимизация задачи об определении конфликтов с опасными зонами движения летательных аппаратов для выполнения на Intel Xeon Phi. // Программные продукты и системы, 2017, Т.~30, №~3, С.~524-528.
DOI:~10.15827/0236-235X.119.3.524-528
%
% ==========



% Перестроение поверхности в двумерном случае.
%
\bibitem{Rybakov2019Geo2D}\textbf{Rybakov A., Shumilin S.} Approximate methods of the surface mesh deformation in two-dimensional cases. // Lobachevskii Journal of Mathematics, 2019, Vol.~40, No.~11, P.~1848-1852. DOI:~10.1134/S1995080219110258

\bibitem{Fortin2004Remesh2d}\textbf{Fortin~G., Ilinca~A., Laforte~J.-L., Brandi~V.} New roughness computation method and geometric accretion model for airfoil icing. // Journal of Aircraft, 20024, Vol.~41, P.~119–127. DOI:~10.2514/1.173
%
\bibitem{Kantorovich1984Func}\textbf{Канторович~Л., Акилов~Г.} Функциональный анализ. // Москва <<Наука>>, 1984, 752~С.
%
% ==========



% Перестроение в трехмерном случае.
%
\bibitem{Meshcheryakov2023GeoEvo}\textbf{Meshcheryakov~A., Rybakov A.} Evolution of the surface computational mesh in the ice accretion process. // Lobachevskii Journal of Mathematics, 2023, Vol.~44, No.~11, P.~361-378. DOI:~10.1134/S1995080223110367
%
\bibitem{Beaugendre2003Ice}\textbf{Beaugendre~H.}A PDE-based approach to in-flight ice accretion. // PhD Thesis (Dep. of Mech. Eng., McGill Univ., Montreal, Qu{\'e}bec, 2003).
%
\bibitem{Rybakov2023GeoRemesh}\textbf{Рыбаков~А.} Геометрическое перестроение расчетной сетки с помощью общей огибающей семейства сфер в задаче ледообразования. // Современные информационные технологии и ИТ-образование, 2023, Т.~19, №~2, С.~282-291. DOI:~10.25559/SITITO.019.202302.282-291
%
% ==========



% Устранение самопересечений.
%
\bibitem{Freylekhman2022GeoIntersect}\textbf{Freylekhman~S., Rybakov~A.} Self-intersection elimination for unstructured surface computational meshes. // Lobachevskii Journal of Mathematics, 2022, Vol.~43, No.~10, P.~2846-2852. DOI:~10.1134/S1995080222130133
%
% ==========



% Метод погруженной границы.
%
\bibitem{Mahesh2003}\textbf{Mahesh K., Constantinescu G., Moin P.} Simulating turbulent flows in complex geometries. // Proceedings of FEDSM2003 2003 4th ASME JSME Joint Fluids Engineering Conference, 2003. DOI: 10.1115/FEDSM2003-45337
%
\bibitem{Ye2020}\textbf{Ye H., Liu Y., Chen B., Liu Z., Zheng J., Pang Y., Chen J.} Hybrid grid generation for viscous flow simulation in complex geometries. // 2020. DOI: 10.21203/rs.3.rs-31698/v1
%
\bibitem{Wright2015}\textbf{Wright W., Struk P., Bartkus T., Addy G.} Recent advances in the LEWICE icing model. // SAE Technical Paper, 2015. DOI: 10.4271/2015-01-2094
%
\bibitem{BourgaultCote2017}\textbf{Bourgault-Côté S., Hasanzadeh K., Lavoie P., Laurendeau E.} Multi-layer methodologies for conservative ice growth. // 7th European Conference for Aeronautics and Aerospace Sciences (EUCASS), 2017. DOI: 10.13009/EUCASS2017-258
%
\bibitem{Tong2016}\textbf{Tong X., Thompson D., Arnoldus Q., Collins E., Luke E.} Three-dimensional surface evolution and mesh deformation for aircraft icing applications. // Journal of Aircraft, 2016. DOI: 10.2514/1.C033949
%
\bibitem{Abalakin2018}\textbf{Абалакин И., Жданова Н., Козубская Т.} Метод погруженных границ для численного моделирования невязких сжимаемых течений. // Журнал вычислительной математики и математической физики, 2018, Т. 58, № 9, С. 1462-1471. DOI: 10.31857/S004446690002525-8
%
\bibitem{Mori2008}\textbf{Mori Y., Peskin C.} Implicit second-order immersed boundary methods with boundary mass. // Comput. Methods Appl. Mech. Engrg, 2008, Vol. 197, P. 2049-2067. DOI: 10.1016/J.CMA.2007.05.028
%
\bibitem{Kim2004}\textbf{Kim J., Choi H.} An immersed-boundary finite-volume method for simulation of heat transfer in complex geometries. // KSME International Journal, 2004, Vol. 18, № 6, P. 1026-1035. DOI: 10.1007/BF02990875
%
\bibitem{Clarke1996}\textbf{Clarke D., Salas M., Hassan H.} Euler calculations for multielement airfoils using cartesian grids. // AIAA Journal, 1986, Vol. 24, № 3, P. 353-358. DOI: 10.2514/3.9273
%
\bibitem{Farrashkhalvat2003}\textbf{Farrashkhalvat M., Miles J.} Basic structured grid generation with an introduction to unstructured grid generation. // 2003, Butterworth Heinemann, 231 p.
%
\bibitem{Rybakov2017}\textbf{Рыбаков А.} Внутреннее представление и механизм межпроцессного обмена для блочно-структурированной сетки при выполнении расчетов на суперкомпьютере. // Программные системы: теория и приложения, 2017, Т. 8, Вып. 1, С. 121-134. DOI: 10.25209/2079-3316-2017-8-1-121-134
%
\bibitem{Savin2019}\textbf{Savin. G., Benderskiy L., Lyubimov D., Rybakov A.} RANS/ILES method optimization for effective calculations on supercomputer. // Lobachevskii Journal of Mathematics, 2019, Vol. 40, No. 5, P. 566-573. DOI: 10.1134/S1995080219050172
%
\bibitem{Giordano2019}\textbf{Giordano A., De Rango A., Rongo R., D'Ambrosio D., Spataro W.} A dynamic load balancing technique for parallel execution of structured grid models. // 2020, In: Sergeyev Y., Kvasov D. (eds) Numerical Computations: Theory and Algorithms. NUMTA 2019. Lecture Notes in Computer Science, Vol. 11973, Springer. DOI: 10.1007/978-3-030-39081-5\_25
%
\bibitem{Fadlun2000}\textbf{Fadlun E., Verzicco R., Orlandi P., Mohd-Yusof J.} Combined immersed-boundary finite-difference methods for three-dimensional complex flow simulations. // Journal of Computational Physics, 2000, Vol. 161, P. 35-60. DOI: 10.1006/jcph.2000.6484
%
\bibitem{Mittal2005}\textbf{Mittal R., Iaccarino G.} Immersed boundary methods. // Annual. Rev. Fluid Mech., 2005, Vol. 37, P. 239-261. DOI: 10.1146/annurev.fluid.37.061903.175743
%
\bibitem{Tseng2003}\textbf{Tseng Y.-H., Ferziger J.} A ghost-cell immersed boundary method for flow in complex geometry. // Journal of Computational Physics, 2003, Vol. 192, P. 593-623. DOI: 10.1016/j.jcp.2003.07.024
%
\bibitem{Peter2016}\textbf{Peter S., De A.} A parallel implementation of the ghost-cell immersed boundary method with application to stationary and moving boundary problems. // Sadhana, 2016, Vol. 41, № 4, P. 441-450. DOI: 10.1007/s12046-016-0484-9
%
\bibitem{Rybakov2020GeoIBM}\textbf{Рыбаков А.} Метод погруженной границы с использованием фиктивных ячеек в трехмерной постановке. // Современные информационные технологии и ИТ-образование, 2020, Т.~16, №~2, С.~321-330. DOI:~10.25559/SITITO.16.202002.321-330
%
\bibitem{Vinnikov2007}\textbf{Винников В., Ревизников Д.} Метод погруженной границы для расчета сверхзвукового обтекания затупленных тел на прямоугольных сетках. // Электронный журнал «Труды МАИ», 2007, № 27, 13 С.
%
\bibitem{Rybakov2019VecInt}\textbf{Рыбаков~А.} Векторизация нахождения пересечения объемной и поверхностной сеток для микропроцессоров с поддержкой AVX-512. // Труды НИИСИ РАН, 2019, Т.~9, №~5, С.~5-14. DOI:~10.25682/NIISI.2019.5.0001
%
\bibitem{Wackers2011}\textbf{Wackers J., Deng G., Leroyer A., Queutey P., Visonneau M.} Adaptive grid refinement for hydrodynamic flows. // Computers \& Fluids, 2011, 55, P. 85-100. DOI: 10.1016/j.compfluid.2011.11.004
%
\bibitem{Zhou2014}\textbf{Zhou L., Yunjun Y., Anlong G., Weijiang Z.} Unstructured adaptive grid refinement for flow feature capture. // Procedia Engineering, 2014, 99, P. 477-483. DOI: 10.1016/j.proeng.2014.12.561
%
\bibitem{Plas2015}\textbf{van der Plas P., Veldman A., van der Heiden H., Luppes R.} Adaptive grid refinement for free-surface flow simulations in offshore applications. // Proceedings of the ASME 2015 34th International Conference on Ocean, Offshore and Arctic Engineering OMA2015, 2015. DOI: 10.1115/OMAE2015-42029
%
\bibitem{Smirnova2018}\textbf{Смирнова Н.} Сравнение схем с расщеплением потока для численного решения уравнения Эйлера сжимаемого газа. // Труды МФТИ, 2018, Т. 10, № 1, С. 122-141.
%
% ==========



%---------------------------------------------------------------------------------------------------
% Глава 2.



%---------------------------------------------------------------------------------------------------
% Глава 3.



% Положения из теории графов.
%
\bibitem{Vizing1964}\textbf{Визинг В.} Об оценке хроматического класса р-графа. // сб. Дискретный анализ. -- Новосибирск: Институт математики СО АН СССР, 1964, Вып. 3, С. 25–30.
%
\bibitem{Vizing1965}\textbf{Визинг В.} Критические графы с данным хроматическим классом. // сб. Дискретный анализ. -- Новосибирск: Институт математики СО АН СССР, 1965, Вып. 5, С. 9–17.
%
\bibitem{Soifer2009}\textbf{Soifer A.} The mathematical coloring book. // 2009, Springer, 619 P.
%
% ==========



%---------------------------------------------------------------------------------------------------
% Глава 4.



% Обзорная часть.
%
\bibitem{Cebrian2019VecScal}\textbf{Cebrian~J., Natvig~L., Lahre~M.} Scalability analysis of AVX-512 extensions. // The Journal of Supercomputing, 2020, Vol.~76, P.~2082-2097. DOI:~10.1007/s11227-019-02840-7
%
\bibitem{Kulikov2019VecAstro}\textbf{Kulikov~I., Chernykh~I., Tutukov~A.} A new hydrodynamic code with explicit vectorization instructions optimizations that is dedicated to the numerical simulation of astrophysical gas flow. I. Numerical method, tests, and model problem. // The Astrophysical Journal Supplement Series, 2019, Vol.~243, No.~4, 15~P. DOI:~10.3847/1538-4365/ab2237
%
\bibitem{Glinting2019VecSwim}\textbf{Glinting~B., Mundani~R.-P.} Comparison of shallow water solvers: applications for dam-break and tsunami cases with reordering strategy for efficient vectorization on modern hardware. // Water, 2019, Vol.~11(4), No.~639, 31~P. DOI:~10.3390/w11040639
%
\bibitem{Yildirim2021VecCFD}\textbf{Yildirim~A., Mader~C., Martins~J.} Accelerating parallel CFD codes on modern vector processors using blockettes. // PASC’21: Proceedings of the Platform for Advanced Scientific Computing Conference, 2021.
DOI:~10.1145/3468267.3470615
%
\bibitem{Rucci2020VecNBody}\textbf{Rucci~E., Moreno~E., Pousa~A., Chichizola~F.} Optimization of the N-body simulation on Intel’s architectures based on AVX-512 instruction set. // In book: Comminications in Computer and Information Science, 2020.
DOI:~10.1007/978-3-030-48325-8\_3
%
\bibitem{Rucci2019VecSW}\textbf{Rucci~E., Garcia~C., Botella~G., De~Giusti~A.} SWIMM 2.0: Enhanced Smith-Waterman on Intel’s multicore and manycore architectures based on AVX-512 vector extensions. // International Journal of Parallel Programming,
2019, Vol.~47(17). DOI:~10.1007/s10766-018-0585-7
%
\bibitem{Choi2023VecKorean}\textbf{Choi~Y., Choi~H., Chung~S.} AVX512Crypto: Parallel implementations of Korean block ciphers using AVX-512. // IEEE Access, 2023. DOI:~10.1109/ACCESS.2023.3278993
%
\bibitem{Cheng2021VecCSIDH}\textbf{Cheng~H., Fotiadis~G., Gro{\ss}sch{\"a}dl~J., Ryan~P., R{\o}nne~P.} Batching CSIDH group actions using AVX-512. // IACR Transactions on Cryptographic Hardware and Embedded Systems, 2021, Vol.~2021, No.~4, P.~618-649. DOI:~10.46586/tches.v2021.i4.618-649
%
% ----------
%
\bibitem{Kusswurm2022VecCpp}\textbf{Kusswurm~D.} Modern parallel programming with C++ and Assembly Language. X86 SIMD development using AVX, AVX2, and AVX-512. // CA, Apress Berkeley Publ., 2022, 633~ P. DOI:~10.1007/978-1-4842-7918-2
%
\bibitem{Blacher2022VecQuick}\textbf{Blacher~M., Giesen~J., Sanders~P., Wassenberg~J.} Vectorized and performance-portable Quicksort. // ArXiv, 2022, art.~2205.05982, P.~1–21. DOI:~10.48550/arXiv.2205.05982
%
\bibitem{Long2022VecSPD}\textbf{Long~S., Fan~X., Chao~L., Yi~L., Fan~S., Guo~X.-W., Yang~C.} VecDualSPHysics: A vectorized implementation of Smoothed Particle Hydrodynamics method for simulating fluid flows on multi-core processors. // Journal of Computational Physics, 2022, Vol.~463, art.~111234. DOI:~10.1016/j.jcp.2022.111234
%
\bibitem{PonteFernandez2022VecInteractions}\textbf{Ponte-Fern{\'a}ndez~C., Gonz{\'a}lez-Dom{\'i}nguez~J., Mart{\'i}n~M.~J.} A SIMD algorithm for the detection of epistatic interactions of any order. // Future Generation Comput. Sys., 2022, Vol.~132, P.~108–123. DOI:~10.1016/j.future.2022.02.009
%
\bibitem{Quisland2023VecSeries}\textbf{Quislant~R., Fernandez~I.} Time series analysis acceleration with advanced vectorization extensions. // The Journal of Supercomputing, 2023, Vol.~79, No.~9, P.~10178–10207. DOI:~10.1007/s11227-023-05060-2
%
\bibitem{Buhrow2022VecMult}\textbf{Buhrow~B., Gilbert~B., Haider~C.} Parallel modular multiplication using 512-bit advanced vector instructions. // Journal of Cryptographic Engineering, 2022, Vol.~12, P.~95–105. DOI:~10.1007/s13389-021-00256-9
%
\bibitem{Choi2022VecPIPO}\textbf{Choi~H., Seo~S.~C.} Efficient parallel implementations of PIPO Block Cipher on CPU and GPU. // IEEE Access, 2022, Vol.~10, P.~85995–86007. DOI:~10.1109/ACCESS.2022.3198707
%
\bibitem{Cheng2022VecSIKE}\textbf{Cheng~H., Fotiadis~G., Gro{\ss}sch{\"a}dl~J., Ryan~P.}. Highly vectorized SIKE for AVX-512. // IACR Transactions on Cryptographic Hardware and Embedded Sys., 2022, No.~2, P.~41–68. DOI:~10.46586/tches.v2022.i2.41-68
%
\bibitem{Sansone2023VecFourier}\textbf{Sansone~G., Cococcioni~M.} Experiments on speeding up the recursive fast Fourier transform by using AVX-512 SIMD instructions. // Proc. ApplePies. LNEE, 2023, Vol.~1036, P.~255–263. DOI:~10.1007/978-3-031-30333-3\_34
%
\bibitem{Edamatsu2023VecDiv}\textbf{Edamatsu~T., Takahashi~D.} Fast multiple-precision integer division using Intel AVX-512. // IEEE Transactions on Emerging Topics in Computing, 2023, Vol.~11, No.~1, P.~224–236. DOI:~10.1109/TETC.2022.3196147
%
\bibitem{Medakin2021VecPP}\textbf{Медакин~П.О., Никулин~Р.Н., Авдеюк~О.А., Королева~И.Ю., Павлова~Е.С., Лемешкина~И.Г.} Векторизация и распараллеливание метода «частица-частица». // Инженерный вестник Дона, 2021, No.~1. URL:~http://ivdon.ru/ru/
magazine/archive/n1y2021/6800 (дата обращения: 01.05.2025)
%
\bibitem{Tayeb2023VecAuto}\textbf{Tayeb~H., Paillat~L., Bramas~B.} Autovesk: Automatic vectorization of unstructured static kernels by graph transformations. // ArXiv, 2023, art.~2301.01018, P.~1–21. DOI:~10.48550/arXiv.2301.01018
%
\bibitem{Laukemann2019VecAuto}\textbf{Laukemann~J., Hammer~J., Hager~G., Wellein~G.} Automatic throughput and critical path analysis of x86 and ARM assembly kernels. // Proc. IEEE/ACM PMBS, 2019, P.~1–6. DOI:~10.1109/PMBS49563.2019.00006
%
% ==========



% Описание инструкций AVX-512.
%
\bibitem{IntelSDM2025}Intel 64 and IA-32 architectures software developer's manual. Combined volumes: 1, 2A, 2B, 2C, 2D, 3A, 3B, 3C, 3D, and 4. // Order number: 325462-087US, march 2025.
%
\bibitem{Kalamkar2019VecBF16}\textbf{Kalamkar~D., Mudigere~D., Mellempudi~N., Das~D. et al.} A study of BFloat16 for deep learning training. // ArXiv, 2019, art.~1905.12322, P.~1-11. DOI:~10.48550/arXiv.1905.12322
%
\bibitem{Zhou2024VecVNNI}\textbf{Zhou~H., Han~Q., Shi~H., Zhang~Y., Yao~J.} Boost linear algebra computation performance via efficient VNNI utilization. // ASPLOS '24: Proceedings of the 29th ACM International Conference on Architectural Support for Programming Languages and Operating Systems, 2024, Vol.~3, P.~149-163. DOI:~10.1145/3620666.3651333
%
\bibitem{DiezCanas2021VecVP2Int}\textbf{D{\'i}ez-Ca{\~n}as~G.} Faster-than-native alternatives for x86 VP2INTERSECT instructions. // ArXiv, 2021, art.~2112.06342, P.~1-10. DOI:~10.48550/arXiv.2112.06342
%
\bibitem{Kovats2024VecAES}\textbf{Kovats~T., Rameshan~N., Karunarathe~K., Giannopoulos~I., Sebastian~A.} In-memory encryption using the advanced encryption standard. // Philosophical Transactions, 2025, Vol.~383, art.~20230396. DOI:~10.1098/rsta.2023.0396
%
\bibitem{Yoo2023VecGFNI}\textbf{Yoo~T.-H., Kivilinna~J., Cho~C.-H.} AVX-based acceleration of ARIA block cipher algorithm. // IEEE Access, 2023. DOI:~10.1109/ACCESS.2023.3298026
%
\bibitem{Volkonsky2003VecPred}\textbf{Волконский~В., Окунев~С.} Предикатное представление как основа оптимизации программы для архитектур с явно выраженной параллельностью. // Информационные технологии, 2003, №~4, С.~36–45.
%
\bibitem{Kim2013VecElb}\textbf{Ким~А., Перекатов~В., Ермаков~С.} Микропроцессоры и вычислительные комплексы семейства «Эльбрус», Питер, СПб., 2013, 273~С.
%
\bibitem{IntelIntrinsicsGuide}Intel Intrinsics Guide, https://alouettesu.github.io/Intrinsics/ (дата обращения 01.05.2025)
%
% ==========



% Выделение однотипных операций - матрицы малой размерности.
%
\bibitem{Bendersky2018VecMat2}\textbf{Бендерский~Л., Лещев~С., Рыбаков~А.} Векторизация операций над матрицами малой размерности для процессора Intel Xeon Phi Knights Landing. // Современные информационные технологии и ИТ-образование, 2018, Т.~14, №~1, С.~73-90. DOI:~10.25559/SITITO.14.201801.073-090
%
\bibitem{iparGithub}Parallelization samples for Intel microprocessors. // репозиторий github.com. https://github.com/r-aax/ipar (дата обращения 01.05.2025)
%
% ==========



% Потери производительности - матрицы специального вида.
%
\bibitem{Bendersky2018VecMat1}\textbf{Бендерский~Л., Рыбаков~А., Шумилин~С.} Векторизация перемножения малоразмерных матриц специального вида с использованием инструкций AVX-512. // Современные информационные технологии и ИТ-образование, 2018, Т.~14, №~3, С.~594-602. DOI:~10.25559/SITITO.14.201803.594-602
%
% ==========



% Введение понятия плоского цикла.
%
\bibitem{Shabanov2021VecCFG}\textbf{Шабанов~Б., Рыбаков~А., Чопорняк~А.} Оптимизации, применяемые к графу потока управления программы для повышения эффективности векторизации плоских циклов. // Труды НИИСИ РАН, 2021, Т.~11, №~2, С.~11-19. DOI:~10.25682/NIISI.2021.2.0002
%
\bibitem{Armstrong2013VecErlang}\textbf{Armstrong J.} Programming Erlang. Software for a concurrent world. // The Pragmatic Programmers, 2013, 520~P.
%
\bibitem{Savin2020VecFlat}\textbf{Savin~G., Shabanov~B., Rybakov~A., Shumilin~S.} Vectorization of flat loops of arbitrary structure using instructions AVX-512. // Lobachevskii Journal of Mathematics, 2020, Vol.~41, No.~12, P.~2566-2574. DOI:~10.1134/S1995080220120331
%
\bibitem{Muchnick1997Compilers}\textbf{Muchnick S.} Advanced compiler design and implementation. // Morgan Kaufmann Publishers, 1997.
%
\bibitem{Rybakov2013CGF}\textbf{Рыбаков А.} Алгоритм создания случайных графов потока управления для анализа глобальных оптимизаций в компиляторе. // Parallel and Distributed Computing Systems PDSC 2013, Collection of scientific papers, P.~269-275.
%
\bibitem{Aho2006Compilers}\textbf{Aho A., Lam M., Sethi R., Ulman J.} Compilers: principles, techniques, and tools. // Prentice Hall, 2nd ed, 2006.
%
\bibitem{Chetverina2015Profile}\textbf{Четверина О.} Методы коррекции профильной информации в процессе компиляции. // Труды ИСП РАН, 2015, Т.~27, Вып.~6, С.~49-63.
%
% ==========



% Метод погруженных границ - композиция плоских циклов.
%
\bibitem{Rybakov2023VecIBM}\textbf{Рыбаков~А., Мещеряков~А.} Векторизация трехмерного метода погруженных границ для повышения эффективности расчетов на микропроцессорах Intel. // Программные продукты и системы, 2023, Т.~36, №~1, C.~130-143. DOI:~10.15827/0236-235X.141.130-143
%
\bibitem{ibmGithub}Immersed boundary methods vectorization. // репозиторий github.com. https://github.com/r-aax/ibm\_vec (дата обращения 01.05.2025)
%
% ==========



% Векторизация с помощью локализации маловероятных регионов.
%
\bibitem{Rybakov2018VecBranch}\textbf{Рыбаков~А., Шумилин~С.} Векторизация сильно разветвленного управления с помощью инструкций AVX-512. // Труды НИИСИ РАН, 2018, Т.~8, №~4, С.~114-126. DOI:~10.25682/NIISI.2018.4.0014
%
\bibitem{Aubakirov1999Wake}\textbf{Аубакиров~Т., Желанников~А., Иванов~П., Ништ~М.} Спутные следы и их воздействие на летательные аппараты. // Моделирование на ЭВМ, Алматы, 1999, 280~С.
%
\bibitem{Vyshinsky2006Wake}\textbf{Вышинский~В., Судаков~Г.} Вихревой след самолёта в турбулентной атмосфере. // Труды ЦАГИ, 2006, Вып.~2667, 155~С.
%
\bibitem{Babkin2008Wake}\textbf{Бабкин~В., Белоцерковский~А., Турчак~Л., Баранов~Н., Замятин~А., Каневский~М., Морозов~В., Пасекунов~И., Чижов~Н.} Системы обеспечения вихревой безопасности полетов летательных аппаратов. // М.: Наука, 2008, 373~С.
%
\bibitem{Burluzky2014Wake}\textbf{Бурлуцкий С.} Вопросы обеспечения вихревой безопасности аэропортов. // Системный анализ и логистика. Специальное научное издание, Вып. от 12 мая 2014 года, С.~37-40.
%
\bibitem{Rybakov2022VecGeom}\textbf{Рыбаков~А.} Векторизация программного кода, содержащего маловероятные регионы, в задачах вычислительной геометрии. // Современные информационные технологии и ИТ-образование, 2022, Т.~18, №~1, С.~28-38. DOI:~10.25559/SITITO.18.202201.28-38
%
\bibitem{Ilbeyi2019}\textbf{Ilbeyi~B.} Co-optimizing hardware design and meta-tracing just-in-time compilation. // A disserta-tion for the degree of Doctor of Philosophy, Cornell University, 2019.
%
% ==========



% Векторизация - слияние под условием.
%
\bibitem{Rybakov2024VecComb}\textbf{Рыбаков~А.} Векторизация циклов с условными операциями с помощью комбинирования векторных масок. // Современные информационные технологии и ИТ-образование, 2024, Т.~20, №~3, С.~520-534. DOI:~10.25559/SITITO.020.202403.520-534
%
% ==========



% Векторизация - проверка масок.
%
%
% ==========



% Векторизация - комбинирование масок.
%
\bibitem{Rybakov2020VecMon}\textbf{Рыбаков~А., Чопорняк~А.} Повышение производительности векторного кода с помощью мониторинга плотности масок в векторных инструкциях. // Труды НИИСИ РАН, 2020, Т.~10, №~4, С.~40-47. DOI:~10.25682/NIISI.2020.4.0006
%
\bibitem{Toh2024VecRiemann}\textbf{Toh Y.} Efficient non-iterative multi-point method for solving the Riemann problem. // Nonlinear Dynnamics, 2024, Vol.~112, P.~5439-5451. DOI:~10.1007/s11071-023-09229-5
%
\bibitem{Zeng2021VecRiemann}\textbf{Zeng Z., Feng C., Yu C. et al.} Linearized double-shock approximate Riemann solver for augmented linear elastic solid. // Numerical Mathematics Theory Methods and Applications, 2021, Vol.~15. DOI:~10.4208/nmtma.OA-2021-0021
%
\bibitem{Lee2024VecGem}\textbf{Lee S., Kim Y., Nam D. et al.} Gem5-AVX: Extension of the Gem5 simulator to support AVX instruction sets. // IEEE Access, 2024. DOI:~10.1109/ACCESS.2024.3359296
%
\bibitem{Rybakov2023VecShvindt}\textbf{Рыбаков А., Швиндт А.} Создание инструментария для векторизации тела плоского цикла с помощью векторных инструкций AVX-512. // Программные продукты и системы, 2023, Т.~36, №~4, С.~561-572. DOI:~10.15827/0236-235X.142.561-572
%
% ==========



% Векторизация - тождества.
%
%
% ==========



% Векторизация - циклы - постоянное число итераций.
%
%
% ==========



% Векторизация - циклы - непостоянного число итераций.
%
\bibitem{Rybakov2019VecRiem1}\textbf{Rybakov~A., Shumilin~S.} Vectorization of the Riemann solver using the AVX-512 instruction set. // Program Systems: Theory and Applications, 2019, Vol.~10, №~3(42), P.~41-58. DOI:~10.25209/2079-3316-2019-10-3-41-58
%
\bibitem{Rybakov2019VecRiem2}\textbf{Рыбаков~А., Шумилин~С.} Векторизация римановского решателя с использованием набора инструкций AVX-512. // Программные системы: Теория и приложения, 2019, Т.~10, №~3(42), С.~59-80. DOI:~10.25209/2079-3316-2019-10-3-59-80
%
\bibitem{numericaGithub}NUMERICA. Hyperbolic Solvers. // репозиторий github.com. https://github.com/dasikasunder/NUMERICA (дата обращения 01.05.2025)
%
\bibitem{riemannvecGithub}Vectorized Riemann Solver. // репозиторий github.com. https://github.com/r-aax/riemann\_vec (дата обращения 01.05.2025)
%
\bibitem{Krzikalla2026Vec}\textbf{Krzikalla~O., Wende~F., Hohnerbach~M.} Dynamic SIMD vector lane scheduling. // ISC High Performance 2016: High Performance Computing, Lectuer Notes Computer Science, 2016, Vol.~9945, P.~354–365. DOI:~10.1007/978-3-319-46079-6\_25
%
\bibitem{Bulat2015VecRiemann}\textbf{Булат~П., Волков~К.} Одномерные задачи газовой динамики и их решение при помощи разностных схем высокой разрешающей способности. // Научно-технический вестник информационных технологий,
механики и оптики, 2015, Т.~15, №~4, С.~731–740. DOI:~ 10.17586/2226-1494-2015-15-4-731-740
%
% ==========



% Векторизация - циклы - нерегулярное число итераций.
%
\bibitem{Rybakov2019VecIrr}\textbf{Рыбаков~А., Шумилин~С.} Исследование эффективности векторизации гнезд циклов с нерегулярным числом итераций. // Программные системы: Теория и алгоритмы, 2019, Т.~10, №~4(43), С.~77-96. DOI:~10.25209/2079-3316-2019-10-4-77-96
%
%
\bibitem{Rybakov2018VecNest}\textbf{Рыбаков~А., Телегин~П., Шабанов~Б.} Проблемы векторизации гнезд циклов с использованием инструкций AVX-512. // Электронный научный журнал: Программные продукты, системы и алгоритмы, 2018, №~3, С.~1-11. DOI:~10.15827/2311-6749.28.314
%
\bibitem{Shabanov2019VecSci}\textbf{Shabanov~B., Rybakov~A., Shumilin~S.} Vectorization of high-performance scientific calculations using AVX-512 instruction set. // Lobachevskii Journal of Mathematics, 2019, Vol.~40, No.~5, P.~580-598. DOI:~10.1134/S1995080219050196
%
\bibitem{Knut1994}\textbf{Кнут~Д.} Искусство программирования. Т. 3: Сортировка и поиск. // Вильямс, М., 1994, 832 С.
%
\bibitem{seqHibbard}A168604, последовательность Хиббарда. // https://oeis.org/A168604 (дата обращения 01.05.2025)
%
\bibitem{seqPratt}A003586, последовательность Пратта. // https://oeis.org/A003586 (дата обращения 01.05.2025)
%
\bibitem{seqSadgwick}A033622, последовательность Седжвика. // https://oeis.org/A033622 (дата обращения 01.05.2025)
%
\bibitem{MOVUPSintel}Using Intel AVX without Writing AVX. // https://software.intel.com/enus/articles/using-intel-avx-without-writing-avx (дата обращения 01.05.2025)
%
% ==========



% Векторизация целочисленного программного контеста.
%
\bibitem{comboptGithub}Векторизация алгоритма пузырькового роста декомпозиции графа. // репозиторий github.com. https://github.com/r-aax/comb\_opt\_vect (дата обращения 01.05.2025)
%
% ==========



%---------------------------------------------------------------------------------------------------

\end{thebibliography}